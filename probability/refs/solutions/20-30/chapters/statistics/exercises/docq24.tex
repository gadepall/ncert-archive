The following python code computes the median .
	\begin{lstlisting}
	codes/statistics/exercises/q24.py
	\end{lstlisting}
	
	
	\begin{align}
	\text{Median} &= l + \frac{\frac{n}{2} -cf}{f}\times h
	\end{align}
	\begin{align}
	\text{n} = \sum f_{i} = 100 \implies \frac{n}{2} = 50\\
	\end{align}
	$\therefore$ 46-48 is the median class.\\
	Here l is the lower limit of the median class = 46\\
	h is the classinterval =2\\
	cf is the cumulative frequency of the class before median class = 14\\
	f is the frequency of the median class =14

	\begin{align}
	\text{Median} &= 46 + \frac{17.5 - 14}{14}\times 2\\
	\text{Median} &= 46 + 0.5 = 46.5
	\end{align}
	Hence median weight is 46.5
	\begin{table}[ht]
	\begin{center}
    	\input{./solutions/20-30/tables/statistics/exercises/stat_ex_q24_ans.tex}
	\caption{Frequency distribution of the weights of students}
	\label{table:stat_ex_q24_anstable10}
	\end{center}
	\end{table}

