
	\item Give five examples of data that you can collect from your day-to-day life.\\
	\item  Classify the data in Q.1 above as primary or secondary data.\\
	\item The blood groups of 30 students of Class VIII are recorded as follows:\\
A, B, O, O, AB, O, A, O, B, A, O, B, A, O, O,\\
A, AB, O, A, A, O, O, AB, B, A, O, B, A, B, O.\\
Represent this data in the form of a frequency distribution table. Which is the most common, and which is the rarest, blood group among these students?\\
\solution
Let $X,Y \in \cbrak{0,1}$ represent HIV with 1 being positive.  From the given information,
{\small
\begin{align}
\pr{X = 1|Y=1} &= \frac{9}{10},\pr{X = 0|Y=1} = \frac{1}{10}
\\
\pr{X = 1|Y=0} &= \frac{1}{100},
\pr{X = 0|Y=0} = \frac{99}{100}
\\
\pr{Y=1} &= \frac{1}{1000},
\pr{Y=0} = \frac{999}{1000}
\end{align}
}
Then, 
{\tiny
\begin{multline}
\pr{Y=1|X=1} 
\\
= \frac{\pr{X=1|Y=1}\pr{Y=1}}{\pr{X=1|Y=1}\pr{Y=1}+\pr{X=1|Y=0}\pr{Y=0}}
\\
=\frac{\frac{9}{10}\times \frac{1}{1000}}{\frac{9}{10}\times \frac{1}{1000}+\frac{1}{100}\times \frac{999}{1000}}
=\frac{10}{121}
\end{multline}
}

\item
	  The distance (in km) of 40 engineers from their residence to their place of work were found as follows: 5,3,10,20, 25, 11, 13, 7, 12, 31,19, 10, 12, 17, 18, 11, 32, 17, 16, 2,7,9,7,8,3,5,12, 15, 18 ,3,12, 14, 2,9,6, 15, 15, 7,6, 12.Construct a grouped frequency distribution table with class size 5 for the data given above taking the first interval as 0-5 (5 not included).
What main features do you observe from this tabular representation?\\
\solution
	  See Table 	\ref{table:stat_ex_q29_quetable6}.
Since the minimum =2 and maximum = 32, we take the intervals as 0-5,5-10 and so on upto 30-35. It can be observed that there are very few engineers whose homes are at more than or equal to 20km distance from their work place and most engineers have their workplace at distance of 0km to 15km from their homes.
	\begin{table}[ht]
	\begin{center}
    	\input{./solutions/20-30/tables/statistics/exercises/stat_ex_q29_que.tex}
	\caption{distance of engineers from their residence to their place of work }
	\label{table:stat_ex_q29_quetable6}
	\end{center}
	\end{table}

	


\item The relative humidity (in $\%$) of a certain city for a month of 30 days was as follows:\\
(i) Construct a grouped frequency distribution table with classes 84 - 86, 86 - 88, etc.\\
(ii) Which month or season do you think this data is about?\\
(iii) What is the range of this data?\\
\begin{table}[ht!]
\resizebox{\columnwidth}{!}{%
\begin{tabular}{ |c|c|c|c|c|c|c|c|c|c| } 
\hline
 98.1 &98.6 &99.2 &90.3 &86.5 &95.3 &92.9 &96.3 &94.2 &95.1  \\ 
 89.2 &92.3 &97.1 &93.5 &92.7 &95.1 &97.2 &93.3 &95.2 &97.3\\ 
 96.2 &92.1 &84.9 &90.2 &95.7 &98.3 &97.3 &96.1 &92.1 &89.0  \\ 
\hline
\end{tabular}
}
\caption{Relative Humidity}
\label{table:docq30statex}
\end{table}
\\
\solution
See Table 	\ref{table:stat_ex_q30_quetable7}
.  Since the minimum =84.9 and maximum = 99.2, we take the intervals as 84-86,86-88 and so on upto 98-100. As the relative humidity is high, the data is about rainy season.\\
\begin{align}
\text{Range} &= \text{Maximum value - Minimum value}\\
\text{Range} &= 99.2 - 84.9\\
\text{Range} &= 14.3
\end{align}

	\begin{table}[ht]
	\begin{center}
    	\input{./solutions/20-30/tables/statistics/exercises/stat_ex_q30_que.tex}
	\caption{Relative humidity in \%}
	\label{table:stat_ex_q30_quetable7}
	\end{center}
	\end{table}

\item The heights of 50 students, measured to the nearest centimetres, have been found to be as follows:\\
\resizebox{\columnwidth}{12pt}{%
\begin{tabular}{ |c|c|c|c|c|c|c|c|c|c| } 
 161 &150 &154 &165 &168 &161 &154 &162 &150 &151  \\ 
 162 &164 &171 &165 &158 &154 &156 &172 &160 &170\\ 
 153 &159 &161 &170 &162 &165 &166 &168 &165 &164  \\ 
 154 &152 &153 &156 &158 &162 &160 &161 &173 &166\\ 
 161 &159 &162 &167 &168 &159 &158 &153 &154 &159  \\ 
\end{tabular}\\%
}\\

(i) Represent the data given above by a grouped frequency distribution table, taking the class intervals as 160 - 165, 165 - 170, etc.\\
(ii) What can you conclude about their heights from the table?\\
\item 
A study was conducted to find out the concentration of sulphur dioxide in the air in parts per million (ppm) of a certain city. The data obtained for 30 days is listed in Table \ref{table:sulphurdioxide_concentration}

\begin{table}[ht!]
\centering
\input{./solutions/20-10/stat/tables/exer32/sulphur.tex}
\caption{Concentrations of sulphur dioxide in air in ppm for 30 days}
\label{table:sulphurdioxide_concentration}
\end{table}
\begin{enumerate}
\item Make a grouped frequency distribution table for this data with class intervals as 0.00-0.04, 0.04-0.08, and so on.
\item  For how many days, was the concentration of sulphur dioxide more than 0.11 parts per million?
\end{enumerate}
\solution
In general, the complex number $\myvec{a_1\\a_2}$ has the matrix representation
\begin{align}
\label{eq:3.4.1_Complex}
\myvec{a_1\\a_2} &= \myvec{a_1 & -a_2\\ a_2 & a_1}\myvec{1\\0}
\\
&= \vec{T}_a\myvec{1\\0}
\\
\implies \myvec{5\\-3}&=\myvec{5&3\\-3&5}\myvec{1\\0}
\end{align}
Then,
\begin{align}
\myvec{5\\-3}^3 &\triangleq\myvec{5&3\\-3&5}^3\myvec{1\\0}
\\
 &= \myvec{-10&198\\-198&-10} \myvec{1\\0}
\\
&=\myvec{-10\\-198}
\end{align}
The python code for above problem is
\begin{lstlisting}
codes/line/comp.py
\end{lstlisting}

\item 
\renewcommand{\theequation}{\theenumi}
\begin{enumerate}[label=\arabic*.,ref=\thesubsubsection.\theenumi]
\numberwithin{equation}{enumi}

\item Find the coordinates of points which divide the line segment joining A=$\myvec{-2\\2}$ , B= $\myvec{2\\8}$ into four equal parts.

\end{enumerate} 
\solution
In general, the complex number $\myvec{a_1\\a_2}$ has the matrix representation
\begin{align}
\label{eq:3.4.1_Complex}
\myvec{a_1\\a_2} &= \myvec{a_1 & -a_2\\ a_2 & a_1}\myvec{1\\0}
\\
&= \vec{T}_a\myvec{1\\0}
\\
\implies \myvec{5\\-3}&=\myvec{5&3\\-3&5}\myvec{1\\0}
\end{align}
Then,
\begin{align}
\myvec{5\\-3}^3 &\triangleq\myvec{5&3\\-3&5}^3\myvec{1\\0}
\\
 &= \myvec{-10&198\\-198&-10} \myvec{1\\0}
\\
&=\myvec{-10\\-198}
\end{align}
The python code for above problem is
\begin{lstlisting}
codes/line/comp.py
\end{lstlisting}

\item 
\renewcommand{\theequation}{\theenumi}
\begin{enumerate}[label=\arabic*.,ref=\thesubsubsection.\theenumi]
\numberwithin{equation}{enumi}

\item Find the coordinates of points which divide the line segment joining A=$\myvec{-2\\2}$ , B= $\myvec{2\\8}$ into four equal parts.

\end{enumerate} 
\solution
In general, the complex number $\myvec{a_1\\a_2}$ has the matrix representation
\begin{align}
\label{eq:3.4.1_Complex}
\myvec{a_1\\a_2} &= \myvec{a_1 & -a_2\\ a_2 & a_1}\myvec{1\\0}
\\
&= \vec{T}_a\myvec{1\\0}
\\
\implies \myvec{5\\-3}&=\myvec{5&3\\-3&5}\myvec{1\\0}
\end{align}
Then,
\begin{align}
\myvec{5\\-3}^3 &\triangleq\myvec{5&3\\-3&5}^3\myvec{1\\0}
\\
 &= \myvec{-10&198\\-198&-10} \myvec{1\\0}
\\
&=\myvec{-10\\-198}
\end{align}
The python code for above problem is
\begin{lstlisting}
codes/line/comp.py
\end{lstlisting}

\item 
\renewcommand{\theequation}{\theenumi}
\begin{enumerate}[label=\arabic*.,ref=\thesubsubsection.\theenumi]
\numberwithin{equation}{enumi}

\item Find the coordinates of points which divide the line segment joining A=$\myvec{-2\\2}$ , B= $\myvec{2\\8}$ into four equal parts.

\end{enumerate} 
\solution
In general, the complex number $\myvec{a_1\\a_2}$ has the matrix representation
\begin{align}
\label{eq:3.4.1_Complex}
\myvec{a_1\\a_2} &= \myvec{a_1 & -a_2\\ a_2 & a_1}\myvec{1\\0}
\\
&= \vec{T}_a\myvec{1\\0}
\\
\implies \myvec{5\\-3}&=\myvec{5&3\\-3&5}\myvec{1\\0}
\end{align}
Then,
\begin{align}
\myvec{5\\-3}^3 &\triangleq\myvec{5&3\\-3&5}^3\myvec{1\\0}
\\
 &= \myvec{-10&198\\-198&-10} \myvec{1\\0}
\\
&=\myvec{-10\\-198}
\end{align}
The python code for above problem is
\begin{lstlisting}
codes/line/comp.py
\end{lstlisting}


\item 
\renewcommand{\theequation}{\theenumi}
\begin{enumerate}[label=\arabic*.,ref=\thesubsubsection.\theenumi]
\numberwithin{equation}{enumi}

\item Find the coordinates of points which divide the line segment joining A=$\myvec{-2\\2}$ , B= $\myvec{2\\8}$ into four equal parts.

\end{enumerate} 
\solution
In general, the complex number $\myvec{a_1\\a_2}$ has the matrix representation
\begin{align}
\label{eq:3.4.1_Complex}
\myvec{a_1\\a_2} &= \myvec{a_1 & -a_2\\ a_2 & a_1}\myvec{1\\0}
\\
&= \vec{T}_a\myvec{1\\0}
\\
\implies \myvec{5\\-3}&=\myvec{5&3\\-3&5}\myvec{1\\0}
\end{align}
Then,
\begin{align}
\myvec{5\\-3}^3 &\triangleq\myvec{5&3\\-3&5}^3\myvec{1\\0}
\\
 &= \myvec{-10&198\\-198&-10} \myvec{1\\0}
\\
&=\myvec{-10\\-198}
\end{align}
The python code for above problem is
\begin{lstlisting}
codes/line/comp.py
\end{lstlisting}


\item 
\renewcommand{\theequation}{\theenumi}
\begin{enumerate}[label=\arabic*.,ref=\thesubsubsection.\theenumi]
\numberwithin{equation}{enumi}

\item Find the coordinates of points which divide the line segment joining A=$\myvec{-2\\2}$ , B= $\myvec{2\\8}$ into four equal parts.

\end{enumerate} 
\solution
In general, the complex number $\myvec{a_1\\a_2}$ has the matrix representation
\begin{align}
\label{eq:3.4.1_Complex}
\myvec{a_1\\a_2} &= \myvec{a_1 & -a_2\\ a_2 & a_1}\myvec{1\\0}
\\
&= \vec{T}_a\myvec{1\\0}
\\
\implies \myvec{5\\-3}&=\myvec{5&3\\-3&5}\myvec{1\\0}
\end{align}
Then,
\begin{align}
\myvec{5\\-3}^3 &\triangleq\myvec{5&3\\-3&5}^3\myvec{1\\0}
\\
 &= \myvec{-10&198\\-198&-10} \myvec{1\\0}
\\
&=\myvec{-10\\-198}
\end{align}
The python code for above problem is
\begin{lstlisting}
codes/line/comp.py
\end{lstlisting}


  \item 
\renewcommand{\theequation}{\theenumi}
\begin{enumerate}[label=\arabic*.,ref=\thesubsubsection.\theenumi]
\numberwithin{equation}{enumi}

\item Find the coordinates of points which divide the line segment joining A=$\myvec{-2\\2}$ , B= $\myvec{2\\8}$ into four equal parts.

\end{enumerate} 
\solution
In general, the complex number $\myvec{a_1\\a_2}$ has the matrix representation
\begin{align}
\label{eq:3.4.1_Complex}
\myvec{a_1\\a_2} &= \myvec{a_1 & -a_2\\ a_2 & a_1}\myvec{1\\0}
\\
&= \vec{T}_a\myvec{1\\0}
\\
\implies \myvec{5\\-3}&=\myvec{5&3\\-3&5}\myvec{1\\0}
\end{align}
Then,
\begin{align}
\myvec{5\\-3}^3 &\triangleq\myvec{5&3\\-3&5}^3\myvec{1\\0}
\\
 &= \myvec{-10&198\\-198&-10} \myvec{1\\0}
\\
&=\myvec{-10\\-198}
\end{align}
The python code for above problem is
\begin{lstlisting}
codes/line/comp.py
\end{lstlisting}


\item 
\renewcommand{\theequation}{\theenumi}
\begin{enumerate}[label=\arabic*.,ref=\thesubsubsection.\theenumi]
\numberwithin{equation}{enumi}

\item Find the coordinates of points which divide the line segment joining A=$\myvec{-2\\2}$ , B= $\myvec{2\\8}$ into four equal parts.

\end{enumerate} 
\solution
In general, the complex number $\myvec{a_1\\a_2}$ has the matrix representation
\begin{align}
\label{eq:3.4.1_Complex}
\myvec{a_1\\a_2} &= \myvec{a_1 & -a_2\\ a_2 & a_1}\myvec{1\\0}
\\
&= \vec{T}_a\myvec{1\\0}
\\
\implies \myvec{5\\-3}&=\myvec{5&3\\-3&5}\myvec{1\\0}
\end{align}
Then,
\begin{align}
\myvec{5\\-3}^3 &\triangleq\myvec{5&3\\-3&5}^3\myvec{1\\0}
\\
 &= \myvec{-10&198\\-198&-10} \myvec{1\\0}
\\
&=\myvec{-10\\-198}
\end{align}
The python code for above problem is
\begin{lstlisting}
codes/line/comp.py
\end{lstlisting}

\item 
\renewcommand{\theequation}{\theenumi}
\begin{enumerate}[label=\arabic*.,ref=\thesubsubsection.\theenumi]
\numberwithin{equation}{enumi}

\item Find the coordinates of points which divide the line segment joining A=$\myvec{-2\\2}$ , B= $\myvec{2\\8}$ into four equal parts.

\end{enumerate} 
\solution
In general, the complex number $\myvec{a_1\\a_2}$ has the matrix representation
\begin{align}
\label{eq:3.4.1_Complex}
\myvec{a_1\\a_2} &= \myvec{a_1 & -a_2\\ a_2 & a_1}\myvec{1\\0}
\\
&= \vec{T}_a\myvec{1\\0}
\\
\implies \myvec{5\\-3}&=\myvec{5&3\\-3&5}\myvec{1\\0}
\end{align}
Then,
\begin{align}
\myvec{5\\-3}^3 &\triangleq\myvec{5&3\\-3&5}^3\myvec{1\\0}
\\
 &= \myvec{-10&198\\-198&-10} \myvec{1\\0}
\\
&=\myvec{-10\\-198}
\end{align}
The python code for above problem is
\begin{lstlisting}
codes/line/comp.py
\end{lstlisting}

\item
\renewcommand{\theequation}{\theenumi}
\begin{enumerate}[label=\arabic*.,ref=\thesubsubsection.\theenumi]
\numberwithin{equation}{enumi}

\item Find the coordinates of points which divide the line segment joining A=$\myvec{-2\\2}$ , B= $\myvec{2\\8}$ into four equal parts.

\end{enumerate} 
\solution
In general, the complex number $\myvec{a_1\\a_2}$ has the matrix representation
\begin{align}
\label{eq:3.4.1_Complex}
\myvec{a_1\\a_2} &= \myvec{a_1 & -a_2\\ a_2 & a_1}\myvec{1\\0}
\\
&= \vec{T}_a\myvec{1\\0}
\\
\implies \myvec{5\\-3}&=\myvec{5&3\\-3&5}\myvec{1\\0}
\end{align}
Then,
\begin{align}
\myvec{5\\-3}^3 &\triangleq\myvec{5&3\\-3&5}^3\myvec{1\\0}
\\
 &= \myvec{-10&198\\-198&-10} \myvec{1\\0}
\\
&=\myvec{-10\\-198}
\end{align}
The python code for above problem is
\begin{lstlisting}
codes/line/comp.py
\end{lstlisting}


%\item The following table gives the distribution of students of two sections according to
%the marks obtained by them:\\
%
%\begin{tabular}{|c|c|c|c|c|}
%\hline
% \multicolumn{2}{c|}{Section A}  
%    &\multicolumn{2}{c|}{Section B} \\
%\cline{2-4}
%
% \textbf{Marks} &\textbf{Frequency} &\textbf{Marks} &\textbf{Frequency} \\
%\hline
%0-10 &3 &0-10 &5\\
%10-20 &9 &10-20 &19\\
%20-30 &17 &20-30 &15\\
%30-40 &12 &30-40 &10\\
%40-50 &9 &40-50 &1\\
%\hline
%
%\end{tabular}\\
%
%Represent the marks of the students of both the sections on the same graph by two frequency polygons. From the two polygons compare the performance of the two sections.\\
\item 
\renewcommand{\theequation}{\theenumi}
\begin{enumerate}[label=\arabic*.,ref=\thesubsubsection.\theenumi]
\numberwithin{equation}{enumi}

\item Find the coordinates of points which divide the line segment joining A=$\myvec{-2\\2}$ , B= $\myvec{2\\8}$ into four equal parts.

\end{enumerate} 
\\
\solution
In general, the complex number $\myvec{a_1\\a_2}$ has the matrix representation
\begin{align}
\label{eq:3.4.1_Complex}
\myvec{a_1\\a_2} &= \myvec{a_1 & -a_2\\ a_2 & a_1}\myvec{1\\0}
\\
&= \vec{T}_a\myvec{1\\0}
\\
\implies \myvec{5\\-3}&=\myvec{5&3\\-3&5}\myvec{1\\0}
\end{align}
Then,
\begin{align}
\myvec{5\\-3}^3 &\triangleq\myvec{5&3\\-3&5}^3\myvec{1\\0}
\\
 &= \myvec{-10&198\\-198&-10} \myvec{1\\0}
\\
&=\myvec{-10\\-198}
\end{align}
The python code for above problem is
\begin{lstlisting}
codes/line/comp.py
\end{lstlisting}

\item 
\renewcommand{\theequation}{\theenumi}
\begin{enumerate}[label=\arabic*.,ref=\thesubsubsection.\theenumi]
\numberwithin{equation}{enumi}

\item Find the coordinates of points which divide the line segment joining A=$\myvec{-2\\2}$ , B= $\myvec{2\\8}$ into four equal parts.

\end{enumerate} 
\\
\solution
In general, the complex number $\myvec{a_1\\a_2}$ has the matrix representation
\begin{align}
\label{eq:3.4.1_Complex}
\myvec{a_1\\a_2} &= \myvec{a_1 & -a_2\\ a_2 & a_1}\myvec{1\\0}
\\
&= \vec{T}_a\myvec{1\\0}
\\
\implies \myvec{5\\-3}&=\myvec{5&3\\-3&5}\myvec{1\\0}
\end{align}
Then,
\begin{align}
\myvec{5\\-3}^3 &\triangleq\myvec{5&3\\-3&5}^3\myvec{1\\0}
\\
 &= \myvec{-10&198\\-198&-10} \myvec{1\\0}
\\
&=\myvec{-10\\-198}
\end{align}
The python code for above problem is
\begin{lstlisting}
codes/line/comp.py
\end{lstlisting}

%\item The runs scored by two teams A and B on the first 60 balls in a cricket match are given below:\\
%
%\begin{tabular}{|c|c|c|}
%\hline
%\textbf{Number of balls} &\textbf{Team A}  &textbf{Team B}\\
%\hline
%1-6 &2 &5\\
%7-12 &1 &6\\
%13-18 &8 &2\\
%19-24 &9 &10\\
%25-30 &4 &5\\
%31-36 &5 &6\\
%37-42 &6 &3\\
%43-48 &10 &4\\
%49-54 &6 &8\\
%55-60 &2 &10\\
%\hline
%\end{tabular}\\
%
%Represent the data of both the teams on the same graph by frequency polygons.\\
%(Hint : First make the class intervals continuous.)\\
\item 
\renewcommand{\theequation}{\theenumi}
\begin{enumerate}[label=\arabic*.,ref=\thesubsubsection.\theenumi]
\numberwithin{equation}{enumi}

\item Find the coordinates of points which divide the line segment joining A=$\myvec{-2\\2}$ , B= $\myvec{2\\8}$ into four equal parts.

\end{enumerate} 
\\
\solution
In general, the complex number $\myvec{a_1\\a_2}$ has the matrix representation
\begin{align}
\label{eq:3.4.1_Complex}
\myvec{a_1\\a_2} &= \myvec{a_1 & -a_2\\ a_2 & a_1}\myvec{1\\0}
\\
&= \vec{T}_a\myvec{1\\0}
\\
\implies \myvec{5\\-3}&=\myvec{5&3\\-3&5}\myvec{1\\0}
\end{align}
Then,
\begin{align}
\myvec{5\\-3}^3 &\triangleq\myvec{5&3\\-3&5}^3\myvec{1\\0}
\\
 &= \myvec{-10&198\\-198&-10} \myvec{1\\0}
\\
&=\myvec{-10\\-198}
\end{align}
The python code for above problem is
\begin{lstlisting}
codes/line/comp.py
\end{lstlisting}

%\item A random survey of the number of children of various age groups playing in a park was found as follows:\\
%
%\begin{tabular}{|c|c|}
%\hline
%\textbf{Age (in years)} &\textbf{No.of Children}\\
%\hline
%1-2 &5\\
%2-3 &3\\
%3-5 &6\\
%5-7 &12\\
%7-10 &9\\
%10-15 &10\\
%15-17 &4\\
%\hline
%\end{tabular}\\
%
%Draw a histogram to represent the data above.\\
\item 
\renewcommand{\theequation}{\theenumi}
\begin{enumerate}[label=\arabic*.,ref=\thesubsubsection.\theenumi]
\numberwithin{equation}{enumi}

\item Find the coordinates of points which divide the line segment joining A=$\myvec{-2\\2}$ , B= $\myvec{2\\8}$ into four equal parts.

\end{enumerate} 
\solution
In general, the complex number $\myvec{a_1\\a_2}$ has the matrix representation
\begin{align}
\label{eq:3.4.1_Complex}
\myvec{a_1\\a_2} &= \myvec{a_1 & -a_2\\ a_2 & a_1}\myvec{1\\0}
\\
&= \vec{T}_a\myvec{1\\0}
\\
\implies \myvec{5\\-3}&=\myvec{5&3\\-3&5}\myvec{1\\0}
\end{align}
Then,
\begin{align}
\myvec{5\\-3}^3 &\triangleq\myvec{5&3\\-3&5}^3\myvec{1\\0}
\\
 &= \myvec{-10&198\\-198&-10} \myvec{1\\0}
\\
&=\myvec{-10\\-198}
\end{align}
The python code for above problem is
\begin{lstlisting}
codes/line/comp.py
\end{lstlisting}

%\item 100 surnames were randomly picked up from a local telephone directory and a frequency distribution of the number of letters in the English alphabet in the surnames was found as follows:\\
%
%\begin{tabular}{|c|c|}
%\hline
%\textbf{No.of letters} &\textbf{No.of Surnames}\\
%\hline
%1-4 &6\\
%4-6 &30\\
%6-8 &44\\
%8-12 &16\\
%12-20 &4\\
%\hline
%\end{tabular}\\
%
%(i) Draw a histogram to depict the given information.\\
%(ii) Write the class interval in which the maximum number of surnames lie.\\
\item 
\renewcommand{\theequation}{\theenumi}
\begin{enumerate}[label=\arabic*.,ref=\thesubsubsection.\theenumi]
\numberwithin{equation}{enumi}

\item Find the coordinates of points which divide the line segment joining A=$\myvec{-2\\2}$ , B= $\myvec{2\\8}$ into four equal parts.

\end{enumerate} 
\solution
In general, the complex number $\myvec{a_1\\a_2}$ has the matrix representation
\begin{align}
\label{eq:3.4.1_Complex}
\myvec{a_1\\a_2} &= \myvec{a_1 & -a_2\\ a_2 & a_1}\myvec{1\\0}
\\
&= \vec{T}_a\myvec{1\\0}
\\
\implies \myvec{5\\-3}&=\myvec{5&3\\-3&5}\myvec{1\\0}
\end{align}
Then,
\begin{align}
\myvec{5\\-3}^3 &\triangleq\myvec{5&3\\-3&5}^3\myvec{1\\0}
\\
 &= \myvec{-10&198\\-198&-10} \myvec{1\\0}
\\
&=\myvec{-10\\-198}
\end{align}
The python code for above problem is
\begin{lstlisting}
codes/line/comp.py
\end{lstlisting}

\item 
\renewcommand{\theequation}{\theenumi}
\begin{enumerate}[label=\arabic*.,ref=\thesubsubsection.\theenumi]
\numberwithin{equation}{enumi}

\item Find the coordinates of points which divide the line segment joining A=$\myvec{-2\\2}$ , B= $\myvec{2\\8}$ into four equal parts.

\end{enumerate} 
\solution
In general, the complex number $\myvec{a_1\\a_2}$ has the matrix representation
\begin{align}
\label{eq:3.4.1_Complex}
\myvec{a_1\\a_2} &= \myvec{a_1 & -a_2\\ a_2 & a_1}\myvec{1\\0}
\\
&= \vec{T}_a\myvec{1\\0}
\\
\implies \myvec{5\\-3}&=\myvec{5&3\\-3&5}\myvec{1\\0}
\end{align}
Then,
\begin{align}
\myvec{5\\-3}^3 &\triangleq\myvec{5&3\\-3&5}^3\myvec{1\\0}
\\
 &= \myvec{-10&198\\-198&-10} \myvec{1\\0}
\\
&=\myvec{-10\\-198}
\end{align}
The python code for above problem is
\begin{lstlisting}
codes/line/comp.py
\end{lstlisting}

\item 
\renewcommand{\theequation}{\theenumi}
\begin{enumerate}[label=\arabic*.,ref=\thesubsubsection.\theenumi]
\numberwithin{equation}{enumi}

\item Find the coordinates of points which divide the line segment joining A=$\myvec{-2\\2}$ , B= $\myvec{2\\8}$ into four equal parts.

\end{enumerate} 
\solution
In general, the complex number $\myvec{a_1\\a_2}$ has the matrix representation
\begin{align}
\label{eq:3.4.1_Complex}
\myvec{a_1\\a_2} &= \myvec{a_1 & -a_2\\ a_2 & a_1}\myvec{1\\0}
\\
&= \vec{T}_a\myvec{1\\0}
\\
\implies \myvec{5\\-3}&=\myvec{5&3\\-3&5}\myvec{1\\0}
\end{align}
Then,
\begin{align}
\myvec{5\\-3}^3 &\triangleq\myvec{5&3\\-3&5}^3\myvec{1\\0}
\\
 &= \myvec{-10&198\\-198&-10} \myvec{1\\0}
\\
&=\myvec{-10\\-198}
\end{align}
The python code for above problem is
\begin{lstlisting}
codes/line/comp.py
\end{lstlisting}

\item 
\renewcommand{\theequation}{\theenumi}
\begin{enumerate}[label=\arabic*.,ref=\thesubsubsection.\theenumi]
\numberwithin{equation}{enumi}

\item Find the coordinates of points which divide the line segment joining A=$\myvec{-2\\2}$ , B= $\myvec{2\\8}$ into four equal parts.

\end{enumerate} 
\solution
In general, the complex number $\myvec{a_1\\a_2}$ has the matrix representation
\begin{align}
\label{eq:3.4.1_Complex}
\myvec{a_1\\a_2} &= \myvec{a_1 & -a_2\\ a_2 & a_1}\myvec{1\\0}
\\
&= \vec{T}_a\myvec{1\\0}
\\
\implies \myvec{5\\-3}&=\myvec{5&3\\-3&5}\myvec{1\\0}
\end{align}
Then,
\begin{align}
\myvec{5\\-3}^3 &\triangleq\myvec{5&3\\-3&5}^3\myvec{1\\0}
\\
 &= \myvec{-10&198\\-198&-10} \myvec{1\\0}
\\
&=\myvec{-10\\-198}
\end{align}
The python code for above problem is
\begin{lstlisting}
codes/line/comp.py
\end{lstlisting}

\item 
\renewcommand{\theequation}{\theenumi}
\begin{enumerate}[label=\arabic*.,ref=\thesubsubsection.\theenumi]
\numberwithin{equation}{enumi}

\item Find the coordinates of points which divide the line segment joining A=$\myvec{-2\\2}$ , B= $\myvec{2\\8}$ into four equal parts.

\end{enumerate} 
\\
\solution
In general, the complex number $\myvec{a_1\\a_2}$ has the matrix representation
\begin{align}
\label{eq:3.4.1_Complex}
\myvec{a_1\\a_2} &= \myvec{a_1 & -a_2\\ a_2 & a_1}\myvec{1\\0}
\\
&= \vec{T}_a\myvec{1\\0}
\\
\implies \myvec{5\\-3}&=\myvec{5&3\\-3&5}\myvec{1\\0}
\end{align}
Then,
\begin{align}
\myvec{5\\-3}^3 &\triangleq\myvec{5&3\\-3&5}^3\myvec{1\\0}
\\
 &= \myvec{-10&198\\-198&-10} \myvec{1\\0}
\\
&=\myvec{-10\\-198}
\end{align}
The python code for above problem is
\begin{lstlisting}
codes/line/comp.py
\end{lstlisting}

\item 
\renewcommand{\theequation}{\theenumi}
\begin{enumerate}[label=\arabic*.,ref=\thesubsubsection.\theenumi]
\numberwithin{equation}{enumi}

\item Find the coordinates of points which divide the line segment joining A=$\myvec{-2\\2}$ , B= $\myvec{2\\8}$ into four equal parts.

\end{enumerate} 
\solution
In general, the complex number $\myvec{a_1\\a_2}$ has the matrix representation
\begin{align}
\label{eq:3.4.1_Complex}
\myvec{a_1\\a_2} &= \myvec{a_1 & -a_2\\ a_2 & a_1}\myvec{1\\0}
\\
&= \vec{T}_a\myvec{1\\0}
\\
\implies \myvec{5\\-3}&=\myvec{5&3\\-3&5}\myvec{1\\0}
\end{align}
Then,
\begin{align}
\myvec{5\\-3}^3 &\triangleq\myvec{5&3\\-3&5}^3\myvec{1\\0}
\\
 &= \myvec{-10&198\\-198&-10} \myvec{1\\0}
\\
&=\myvec{-10\\-198}
\end{align}
The python code for above problem is
\begin{lstlisting}
codes/line/comp.py
\end{lstlisting}

%\item The following number of goals were scored by a team in a series of 10 matches:\\
%2, 3, 4, 5, 0, 1, 3, 3, 4, 3\\
%Find the mean, median and mode of these scores.\\
%\item In a mathematics test given to 15 students, the following marks (out of 100) are recorded:\\
%41, 39, 48, 52, 46, 62, 54, 40, 96, 52, 98, 40, 42, 52, 60\\               
%Find the mean, median and mode of this data.\\
%\item The following observations have been arranged in ascending order. If the median of the data is 63, find the value of x.\\
%29, 32, 48, 50, x, x + 2, 72, 78, 84, 95\\
%\item Find the mode of \\
%14, 25, 14, 28, 18, 17, 18, 14, 23, 22, 14, 18.\\
%\item Find the mean salary of 60 workers of a factory from the following table:\\
%\begin{tabular}{|c|c|}
%\hline
%\textbf{Salary (in \rupee)} &\textbf{No.of Workers}\\
%\hline
%3000 &16\\
%4000 &12\\
%5000 &10\\
%6000 &8\\
%7000 &6\\
%8000 &4\\
%9000 &3\\
%10000 &1\\
%\hline
%\textbf{Total} &60\\
%\hline
%\end{tabular}\\

%\item Give one example of a situation in which\\
%(i) the mean is an appropriate measure of central tendency.\\
%(ii) the mean is not an appropriate measure of central tendency but the median is an
%appropriate measure of central tendency.\\



