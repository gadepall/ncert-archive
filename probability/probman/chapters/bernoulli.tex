\renewcommand{\theequation}{\theenumi}
\renewcommand{\thefigure}{\theenumi}
\begin{enumerate}[label=\thesection.\arabic*.,ref=\thesection.\theenumi]
\numberwithin{equation}{enumi}
\numberwithin{figure}{enumi}
\numberwithin{table}{enumi}
%
\item Find the probability of getting a head when a coin is tossed once. Also
find the probability of getting a tail.
\\
\solution  
Let the random variable be $X\in \cbrak{0,1}$.  Then
\begin{align}
\pr{X=0}=\pr{X=1}=\frac{1}{2}
\end{align}
The following code simulates the event for 100 coin tosses
\begin{lstlisting}
codes/bernoulli/coin.py
\end{lstlisting}
\item A jar contains 24 marbles, some are green and others are blue. If a marble is drawn at random from the jar, the probability that it is green is
$\frac{2}{3}$. Find the number of blue balls (marbles) in the jar.\\
\solution Let random variable $X\in\{0,1\}$ denote the outcomes of the experiment of drawing a marble from a jar as shown in Table \ref{table:bernoulli}
%
\begin{table}
\centering
\caption{}
\renewcommand{\theequation}{\theenumi}
\renewcommand{\thefigure}{\theenumi}
\begin{enumerate}[label=\thesection.\arabic*.,ref=\thesection.\theenumi]
\numberwithin{equation}{enumi}
\numberwithin{figure}{enumi}
%
\item Find the probability of getting a head when a coin is tossed once. Also
find the probability of getting a tail.
\\
\solution  Let the random variable be $X\in \cbrak{0,1}$.  Then
\begin{align}
\pr{X=0}=\pr{X=1}=\frac{1}{2}
\end{align}
The following code simulates the event for 100 coin tosses
\begin{lstlisting}
codes/bernoulli/coin.py
\end{lstlisting}
\item A jar contains 24 marbles, some are green and others are blue. If a marble is drawn at random from the jar, the probability that it is green is
$\frac{2}{3}$. Find the number of blue balls (marbles) in the jar.\\
%\solution
%
%Total number of marbles $=24$
%
%Let $G$ be the event that marble drawn out randomly is green. 
%
%Since, probability of marble being green, $P(G)=\frac{\mbox{Number of green marbles}}{\mbox{Total number of marbles}}$. 
%
%We have, 
%\begin{align}
%\mbox{Number of green marbles}&=P(G)*\mbox{Total number of marbles}\nonumber\\
%&=\frac{2}{3}*24=16.
%\label{eq:green_marble}
%\end{align}
%
%So, from Eq.~\eqref{eq:green_marble} we have \begin{align}
%\mbox{No. of blue marbles}&=\mbox{Total no. of marbles}-\mbox{No. of green marbles}\nonumber\\
%&=24-16=8.
%\label{eq:blue_marble}
%\end{align}


\end{enumerate}



\label{table:bernoulli}
\end{table}
%
From the given information,
\begin{align}
\label{eq:bernoulli_x=1}
P(X=1)&=\frac{2}{3}
\\
n\brak{X=0}+
n\brak{X=1} &= 24
\label{eq:bernoulli_sum}
\end{align}
%
Thus, from \eqref{eq:bernoulli_x=1} 
\begin{align}
 P(X=0)&=1-P(X=1)\nonumber\\
 &=1-\frac{2}{3}=\frac{1}{3}.
\end{align}
%
%Since, we know 
$\because$
\begin{equation}
 P(X=0)=\frac{n\brak{X=0} }{n\brak{X=0}+
n\brak{X=1}},
 \label{eq:blue_marb}
\end{equation} 
from \eqref{eq:blue_marb} and \eqref{eq:bernoulli_sum},
%So, using Eq.~\eqref{eq:blue_marb} we have, 
\begin{align}
n\brak{X=0}  &=P(X=0)\cbrak{n\brak{X=0}+n\brak{X=1}}
\\
&=\frac{1}{3}\times 24=8.
\label{eq:green_marble}
\end{align}
% 
The following code generates the number of blue marbles 
\begin{lstlisting}
codes/bernoulli/bernoulli.py
\end{lstlisting}


\end{enumerate}


