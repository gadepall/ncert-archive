Two dice, one blue and one grey, are thrown at the same time.   The event defined by the sum of the two numbers appearing on the top of the dice can have 11 possible outcomes 2, 3, 4, 5, 6, 6, 8, 9, 10, 11 and 12.  A student argues that each of these outcomes has a probability $\frac{1}{11}$.  Do you agree with this argument?  Justify your answer.

\renewcommand{\theequation}{\theenumi}
\renewcommand{\thefigure}{\theenumi}
\begin{enumerate}[label=\thesubsection.\arabic*.,ref=\thesubsection.\theenumi]
\numberwithin{equation}{enumi}
\numberwithin{figure}{enumi}

%\begin{abstract}
%We show that the problem of finding the probability of the sum of numbers appearing on top of two dice thrown at the same time can be solved used concepts in signal processing.  
%\begin{keywords}
%conditional probability, convolution, $Z$-transform
%\end{keywords}\bigskip
%
%
%\end{abstract}
%

\item  {\em The Uniform Distribution: }Let $X_i \in \cbrak{1,2,3,4,5,6}, i = 1,2,$ be the random variables representing the outcome for each die.  Assuming the dice to be fair, the probability mass function (pmf) is expressed as 
\begin{align}
\label{eq:dice_pmf_xi}
p_{X_i}(n) = \pr{X_i = n} = 
\begin{cases}
\frac{1}{6} & 1 \le n \le 6
\\
0 & otherwise
\end{cases}
\end{align}
The desired outcome is
\begin{align}
\label{eq:dice_xdef}
X &= X_1 + X_2,
\\
\implies X &\in \cbrak{1,2,\dots,12}
\end{align}
%
The objective is to show that
\begin{align}
p_X(n) \ne \frac{1}{11}
\label{eq:dice_wrong}
\end{align}
\item {\em Convolution: }
From \eqref{eq:dice_xdef},
\begin{align}
p_X(n) &= \pr{X_1 + X_2 = n} = \pr{X_1  = n -X_2}
\\
&= \sum_{k}^{}\pr{X_1  = n -k | X_2 = k}p_{X_2}(k)
\label{eq:dice_x_sum}
\end{align}
after unconditioning.  $\because X_1$ and $X_2$ are independent,
\begin{multline}
\pr{X_1  = n -k | X_2 = k} 
\\
= \pr{X_1  = n -k} = p_{X_1}(n-k)
\label{eq:dice_x1_indep}
\end{multline}
From \eqref{eq:dice_x_sum} and \eqref{eq:dice_x1_indep},
\begin{align}
p_X(n) = \sum_{k}^{}p_{X_1}(n-k)p_{X_2}(k) = p_{X_1}(n)*p_{X_2}(n)
\label{eq:dice_x_conv}
\end{align}
where $*$ denotes the convolution operation. 
%\cite{proakis_dsp}.  
Substituting from \eqref{eq:dice_pmf_xi}
in \eqref{eq:dice_x_conv},
\begin{align}
p_X(n) = \frac{1}{6}\sum_{k=1}^{6}p_{X_1}(n-k)= \frac{1}{6}\sum_{k=n-6}^{n-1}p_{X_1}(k)
\label{eq:dice_x_conv_x1}
\end{align}
\begin{align}
\because p_{X_1}(k) &= 0, \quad k \le 1, k \ge 6.
\end{align}
From \eqref{eq:dice_x_conv_x1},
%
\begin{align}
p_X(n) &= 
\begin{cases}
0 & n < 1
\\
\frac{1}{6}\sum_{k=1}^{n-1}p_{X_1}(k) &  1 \le n-1 \le  6
\\
\frac{1}{6}\sum_{k=n-6}^{6}p_{X_1}(k) & 1 < n-6 \le 6
\\
0 & n > 12
\end{cases}
\label{eq:dice_x_conv_cond}
\end{align}
Substituting from \eqref{eq:dice_pmf_xi} in \eqref{eq:dice_x_conv_cond},
\begin{align}
p_X(n) &= 
\begin{cases}
0 & n < 1
\\
\frac{n-1}{36} &  2 \le n \le  7
\\
\frac{13-n}{36} & 7 < n \le 12
\\
0 & n > 12
\end{cases}
\label{eq:dice_x_conv_final}
\end{align}
satisfying \eqref{eq:dice_wrong}.
\item {\em The $Z$-transform: }
The $Z$-transform of $p_X(n)$ is defined as 
%\cite{proakis_dsp}
\begin{align}
P_X(z) = \sum_{n = -\infty}^{\infty}p_X(n)z^{-n}, \quad z \in \mathbb{C}
\label{eq:dice_xz}
\end{align}
%
From \eqref{eq:dice_pmf_xi} and \eqref{eq:dice_xz}, 
\begin{align}
P_{X_1}(z) =P_{X_2}(z) &= \frac{1}{6}\sum_{n = 1}^{6}z^{-n}
\\
&=\frac{z^{-1}\brak{1-z^{-6}}}{6\brak{1-z^{-1}}}, \quad \abs{z} > 1
\label{eq:dice_xiz}
\end{align}
upon summing up the geometric progression.  
\begin{align}
\because p_X(n) &= p_{X_1}(n)*p_{X_2}(n),
\\
P_X(z) &= P_{X_1}(z)P_{X_2}(z)
\label{eq:dice_xzprod_def}
\end{align}
The above property follows from Fourier analysis and is fundamental to signal processing. 
%\cite{proakis_dsp}. 
From \eqref{eq:dice_xiz} and \eqref{eq:dice_xzprod_def},
\begin{align}
P_X(z) &= \cbrak{\frac{z^{-1}\brak{1-z^{-6}}}{6\brak{1-z^{-1}}}}^2
\\
&= \frac{1}{36}\frac{z^{-2}\brak{1-2z^{-6}+z^{-12}}}{\brak{1-z^{-1}}^2}
\label{eq:dice_xzprod}
\end{align}
Using the fact that 
%\cite{proakis_dsp}
\begin{align}
p_X(n-k) &\system{Z}P_X(z)z^{-k},
\\
nu(n)&\system{Z} \frac{z^{-1}}{\brak{1-z^{-1}}^2}
\end{align}
after some algebra, it can be shown that
%{\tiny
\begin{multline}
\frac{1}{36}\lsbrak{\brak{n-1}u(n-1) - 2 \brak{n-7}u(n-7)}
\\
\rsbrak{ +\brak{n-13}u(n-13)}
\\
\system{Z}
\frac{1}{36}\frac{z^{-2}\brak{1-2z^{-6}+z^{-12}}}{\brak{1-z^{-1}}^2}
\label{eq:dice_xz_closed}
\end{multline}
%}

where 
\begin{align}
u(n) =
\begin{cases}
1 & n \ge 0
\\
0 & n < 0
\end{cases}
\end{align}

From \eqref{eq:dice_xz}, \eqref{eq:dice_xzprod} and \eqref{eq:dice_xz_closed}
\begin{multline}
p_{X}(n) = \frac{1}{36}\lsbrak{\brak{n-1}u(n-1) 
}
\\
\rsbrak{- 2 \brak{n-7}u(n-7)+\brak{n-13}u(n-13)}
\end{multline}
which is the same as \eqref{eq:dice_x_conv_final}.  Note that  \eqref{eq:dice_x_conv_final} can be obtained from \eqref{eq:dice_xz_closed} using contour integration as well.
% \cite{proakis_dsp}.  

\item 
The experiment of rolling the dice was simulated using Python for 10000 samples.  These were generated using Python libraries for uniform distribution. The frequencies for each outcome were then used to compute the resulting pmf, which  is plotted in Figure \ref{fig:dice}.  The theoretical pmf obtained in \eqref{eq:dice_x_conv_final} is plotted for comparison.  
%
\begin{figure}[!ht]
\centering
\includegraphics[width=\columnwidth]{./figs/sum/pmf.png}
\caption{Plot of $p_X(n)$.  Simulations are close to the analysis. }
\label{fig:dice}
\end{figure}
\item The python code is available in 
\begin{lstlisting}
/codes/sum/dice.py
\end{lstlisting}

%\item 
%We have shown how a simple problem of throwing a dice can be used for learning not just probability but concepts in  signal processing as well.  Inversion of the $Z$-transform for finding the pmf using contour integration, though not discussed here, opens a window to complex analysis too.  Note that the solutions that are provided  can be easily understood using high school math like arithmetic and geometric sums.  Thus, school students can use a bit of college level math to obtain simpler solutions to their problems.  College students can be exposed to advanced mathematics through simple problems from high school texts.  In addition, both can learn how to verify theoretical results through computer simulations.

%\bibliographystyle{tMES}
%\bibliography{school}
%
%\end{document}
\end{enumerate}
