 In the given question,
\begin{enumerate}
\item The sample size = Total number of possibilities(S)=25
\\
The possibilities are shown in the below table \ref{table:10-1_exer7table1}
\begin{table}[ht!]
\centering
\input{./solutions/10-1/prob/tables/prob7_inp.tex}
\caption{Input Values}
\label{table:10-1_exer7table1}	
\end{table}
Event size=Both same day=5
\\
Possibilities are given in table \ref{table:10-1_exer7table2}
\begin{table}[ht!]
\centering
\input{./solutions/10-1/prob/tables/prob7_1.tex}
\caption{Event Values}
\label{table:10-1_exer7table2}	
\end{table}
Probability =
\begin{align}
P=\frac{1}{5}
\end{align}
\end{enumerate}
\begin{enumerate}
\item Event size = On consequitive days=8
\\
Possibilities are given in the table \ref{table:10-1_exer7table3}
\begin{table}[ht!]
\centering
\input{./solutions/10-1/prob/tables/prob7_2.tex}
\caption{Event Values}
\label{table:10-1_exer7table3}	
\end{table}
Probability =
\begin{align}
P=\frac{8}{25}
\end{align}
\end{enumerate}
\begin{enumerate}
\item Event size= On different days=20
\\
Possibilities are given in the table \ref{table:10-1_exer7table4}
\begin{table}[ht!]
\centering
\input{./solutions/10-1/prob/tables/prob7_3.tex}
\caption{Event Values}
\label{table:10-1_exer7table4}	
\end{table}
Probability =
\begin{align}
P=\frac{4}{5}
\end{align}
\end{enumerate}
