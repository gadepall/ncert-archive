The question can be seen as choosing a number first from 1 to 6 numbers and then choosing one more from the remaining 5 numbers, Let $X_1$ be the $1^{st}$ numbers drawn randomly from 1 to 6 and $X_2$be the $2^{nd}$ number drawn from remaining and $X = \text{max } (X_1,X_2)$
%
\begin{align}
& Pr(X_1=n_1)= \begin{cases}
\dfrac{1}{6},  \text{ if } 1 \leq n_1 \leq 6\\
0,  \text{  otherwise }
\end{cases}\\
& Pr(X_2=n_2)= \begin{cases}
\dfrac{1}{5},  \text{ if } 1 \leq n_2 \leq 6 \text{ and }n_2 \neq n_1\\
0,  \text{  otherwise }
\end{cases}
\end{align}

let max $(X_1,X_2)=i$ and $Pr(X=i)$ denotes the probability that $X = \text{max } (X_1,X_2)=i$
 \begin{multline}
Pr(X=i)=Pr(X_1=i\text{ and }X_2<i)\\
 +Pr(X_2=i\text{ and }X_1<i) \label{eqn_(0.0.1)}
\end{multline}
 

since choosing of $X_1,X_2$ are independent events, so we can write 
$$Pr(X_1 \text{ and }X_2)=Pr(X_1)Pr(X_2)$$
Substituting this in \eqref{eqn_(0.0.1)} gives us
\begin{multline}
Pr(X=i)=Pr(X_1=i)Pr(X_2<i)+\\
Pr(X_2=i)Pr(X_1<i)
\end{multline}

\begin{align}
& \implies Pr(X=i)=\dfrac{1}{6}\times \dfrac{(i-1)}{5}+\dfrac{(i-1)}{6} \times\dfrac{1}{5}\\
& \implies Pr(X=n)=\dfrac{(i-1)}{15}
\end{align}

The expectation value of X represented by E(X) is given by
$$E(X)=\sum_{i=1}^{6} Pr(X=i)\times i$$
\begin{align}
& \implies E(X)=\sum_{i=1}^{6} \dfrac{(i-1)}{15}\times i\\
& \implies E(X)=\sum_{i=1}^{6} \dfrac{(i^2-i)}{15}\\
& \implies E(X)=\dfrac{1}{15} \sum_{i=1}^{6} i^2-\dfrac{1}{15}\sum_{i=1}^{6} i\\
& \implies E(X)=\dfrac{1}{15} \times 91-\dfrac{1}{15} \times 21\\
& \implies E(X)= \textbf{4.6667}
\end{align}
