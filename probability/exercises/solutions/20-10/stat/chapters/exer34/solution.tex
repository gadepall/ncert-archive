The 50 digits appearing after the decimal point of $\pi$ are tabulated as follows:\\
\begin{table}[ht!]
\centering
\input{./solutions/20-10/stat/tables/exer34/inp34.tex}
\caption{The 50 decimal places of $\pi$}
\label{table:input_34}
\end{table}
The frequency distribution table for the digits 0-9 is given in table \ref{table:output_34}
\begin{table}[ht!]
\centering
\input{./solutions/20-10/stat/tables/exer34/out34.tex}
\caption{Frequency distribution table for the numbers in \ref{table:input_34}}
\label{table:output_34}
\end{table}
\\From the table \ref{table:output_34},\\
0 is the least frequent digit \\
3,9 are the most frequent digits.\\
The python code for above problem is 
\begin{lstlisting}
./solutions/20-10/stat/codes/exer34.py
\end{lstlisting}
