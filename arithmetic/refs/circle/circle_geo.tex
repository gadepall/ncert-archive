\renewcommand{\theequation}{\theenumi}
\begin{enumerate}[label=\arabic*.,ref=\thesubsection.\theenumi]
\numberwithin{equation}{enumi}
\item Find the coordinates of a point $\vec{A}$, where $AB$ is the diameter of a circle whose centre is \myvec{2,-3} and $\vec{B} = \myvec{1\\4}$.
\item Find the centre $O$f a circle passing through the points \myvec{6\\-6}, \myvec{3\\-7} and  \myvec{3\\3}.
\item Sketch the circles with 
\begin{enumerate}
\item centre \myvec{0\\2} and radius 2
\item centre \myvec{-2\\32} and radius 4
\item centre $\myvec{\frac{1}{2}\\ \frac{1}{4}}$ and radius $\frac{1}{12}$.
\item centre \myvec{1\\1} and radius $\sqrt{2}$.
\item centre \myvec{-a\\-b} and radius $\sqrt{a^2-b^2}$.
\end{enumerate}
\item 
\item Sketch the circles with equation
\begin{enumerate}
\item $\norm{\vec{x}-\myvec{5\\-3}}^2 = 36$
\item $\vec{x}^T\vec{x}-\myvec{4\\8}\vec{x} -45= 0$
\item $\vec{x}^T\vec{x}-\myvec{8\\-10}\vec{x} -12= 0$
\item $2\vec{x}^T\vec{x}-\myvec{1\\0}\vec{x} = 0$
\end{enumerate}
%
\item Find the equation of the circle passing through the points \myvec{4\\1} and \myvec{6\\5} and whose centre is on the line $\myvec{4 & 1}\vec{x} = 16$.
\item Find the equation of the circle passing through the points \myvec{2\\3} and \myvec{–1\\1} and whose centre is on the line $\myvec{1 & -3}\vec{x} = 11$.
\item Find the equation of the circle with radius 5 whose centre lies on x-axis and passes through the point \myvec{2\\3}.
\item Find the equation of the circle passing through \myvec{0\\0} and making intercepts a and b on the coordinate axes.
\item Find the equation of a circle with centre \myvec{2\\2} and passes through the point \myvec{4\\5}. 
\item  Does the point \myvec{–2.5\\ 3.5} lie inside, outside or on the circle $\vec{x}^T\vec{x} = 25$?
\item Find the locus of all the unit vectors in the xy-plane.
%
\item Find the points on the curve $\vec{x}^T\vec{x}-2\myvec{1 & 0}\vec{x} -3 =0$  at which the tangents are parallel to the x-axis.
%
\item  Find the area of the region in the first quadrant enclosed by x-axis, line $\myvec{1 & -\sqrt{3}}\vec{x} =0$ and the circle $\vec{x}^T\vec{x}=4$.
%
\item Find the area lying in the first quadrant and bounded by the circle $\vec{x}^T\vec{x}=4$ and the lines $x = 0$ and $x = 2$.
%
\item Find the area of the circle $4\vec{x}^T\vec{x}=9$.
\item  Find the area bounded by curves $\norm{\vec{x}-\myvec{1\\0}} = 1$ and $\norm{\vec{x}}=1$
\item Find the smaller area enclosed by the circle $\vec{x}^T\vec{x}=4$ and the line $\myvec{1 & 1}\vec{x} = 2$.
\item The sum of the perimeter of a circle and square is $k$, where $k$ is some constant. Prove that the sum of their areas is least when the side of square is double the radius of the circle.
\item A window is in the form of a rectangle surmounted by a semicircular opening. The total perimeter of the window is 10 m. Find the dimensions of the window to admit maximum light through the whole opening.
%
\item If 
$
\brak{x-a}^2+\brak{y-b}^2 = c^2,
$
for some $c > 0$, prove that 
\begin{align}
\frac{\brak{1+y_2}^\frac{3}{2}}{y_2}
\end{align}
%
is a constant independent of $a$ and $b$.
%
\item Form the differential equation of the family of circles touching the y-axis at origin.
\item Form the differential equation of the family of circles having centre on y-axis and radius 3 units.
\end{enumerate}
