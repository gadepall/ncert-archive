The  following Python code generates Fig. \ref{fig:tri_geo_8_tri_sss_py}
%
\begin{lstlisting}
codes/triangle.py
\end{lstlisting}
%\renewcommand{\theequation}{\theenumi}
%\begin{enumerate}[label=\thesection.\arabic*.,ref=\thesection.\theenumi]
%\numberwithin{equation}{enumi}
%\item The figure for the triangle obtained in the question looks like Fig. \ref{fig:tri_geo_8_tri_right_angle}.
%%with angles $\phase{ A},\phase{ C}$ and $\phase{ B}$ and sides $a, b$ and $c$.  The unique feature of this triangle is $\phase{ C}$ which is defined to be $90\degree$.
%with vertices A,B and C
%
%
%%\renewcommand{\thefigure}{\theenumi.\arabic{figure}}
%\begin{figure}[!ht]
%\centering
%\resizebox{\columnwidth}{!}{
\begin{tikzpicture}

\draw (0,0) node[anchor=north]{$B$}
  -- (4,8) node[anchor=west]{$A$}
  -- (8,0) node[anchor=north]{$C$}
  -- cycle;
\draw (4,0) node[anchor=north]{$E$}
  -- (2,4) node[anchor=east]{$D$}
  -- (6,4) node[anchor=west]{$F$}
  -- cycle;
\end{tikzpicture}

}
%\caption{Scalene Triangle by Latex-Tikz}
%\label{fig:tri_geo_8_tri_right_angle}	
%\end{figure}
%%
%%
%%\renewcommand{\thefigure}{\theenumi}
%%
%\item List the design parameters for construction
%\label{const:table1}
%\\
%\solution See Table. \ref{table:table1}. 
%%
%\begin{table}[ht!]
%\centering
%%\begin{tabular}{ |p{3cm}|p{3cm}|  }
%%\hline
%% \multicolumn{2}{|c|}{Initial Input Values.} \\
%%\hline
%%a & 4\\
%%\hline
%%b & 3\\
%%\hline
%%$\phase{(ACB)$ & $90^{\circ}$ \\
%%\hline
%%\end{tabular}
%\input{./solutions/8/tables/inp.tex}
%\caption{To construct $\triangle ABC$}
%\label{table:table1}	
%\end{table}
%
%%\item
%%	For simplicity, let the greek letter $\alpha = \phase{ B$.  We have the following definitions.
%%\begin{equation}
%%\label{eq:tri_trig_defs}
%%\begin{matrix}
%	%\sin \theta = \frac{b}{c} & 	\cos \theta = \frac{a}{c} \\
%	%\tan \theta = \frac{c}{a} & \cot \theta = \frac{1}{\tan \theta} \\
%	%\csc \theta = \frac{1}{\sin \theta} & \sec \theta = \frac{1}{\cos \theta}
%	%\end{equation}
%%
%\item Find the coordinates of the various points in Fig. \ref{fig:tri_geo_8_tri_right_angle}
%\label{const:tri_right_angle}
%\\
%
%\solution 
From the given information, 
%$\triangle ABC$ are 
\begin{align}
\label{eq:constr_a}
\vec{A} &= \myvec{4\\-6} 
\\
\vec{B} &= \myvec{3\\-2}, 
\label{eq:constr_b}
\\
\vec{C} &= \myvec{5\\2}
\label{eq:constr_c}
\end{align}
$\because \vec{M}$ is the midpoint of $AB$,
\begin{align}
\vec{M}= \frac{\vec{A}+\vec{B}}{2} = \frac{1}{2}\myvec{7\\-8}
\label{eq:constr_m}
\end{align}

$\because \vec{N}$ is the midpoint of $BC$,
\begin{align}
\vec{N}= \frac{\vec{B}+\vec{C}}{2} = \frac{1}{2}\myvec{8\\0}
\label{eq:constr_n}
\end{align}

$\because \vec{P}$ is the midpoint of $CA$,
\begin{align}
\vec{P}= \frac{\vec{C}+\vec{A}}{2} = \frac{1}{2}\myvec{9\\-4}
\label{eq:constr_p}
\end{align}
%
%Also, $\vec{M}$ is given to be the midpoint of $C$.  Hence, 
%\begin{align}
%\vec{M}&= \frac{\vec{C}+\vec{D}}{2}
%\\
%\implies \vec{D} &= 2 \vec{M} - \vec{C} = \myvec{a\\b}
%\label{eq:constr_d}
%\end{align}
%
%The values are listed in 
%%\item List the  derived values.
%%\label{const:table2}
%%\\
%%\solution See  
%Table. \ref{table:table2} 
%\begin{table}[ht!]
%\centering
%\begin{tabular}{ |p{3cm}|p{3cm}|  }
%\hline
% \multicolumn{2}{|c|}{Derived Values.} \\
%\hline
%$\vec{M}$ & $$\begin{pmatrix}3.5\\-4\end{pmatrix}$$\\						
%\hline
%$\vec{N}$ & $$\begin{pmatrix}4\\0\end{pmatrix} $$\\
%\hline
%$\vec{P}$ & $$\begin{pmatrix}4.5\\-2\end{pmatrix} $$\\
%\hline
%\end{tabular}
%\caption{To get Mid-points of $\triangle ABC$}
%\label{table:table2}
%\end{table}
%
%\item Draw Fig. \ref{fig:tri_geo_8_tri_right_angle}.	
%\\
%\solution 

The  following Python code verifies the determinant values.

\begin{lstlisting}
codes/determinant_check.py
\end{lstlisting}

\begin{figure}[!ht]
\centering
\includegraphics[width=\columnwidth]{./solutions/8/figs/triangle.eps}
\caption{Triangle generated using python}
\label{fig:tri_geo_8_tri_sss_py}
\end{figure}

%
%and the equivalent latex-tikz code generating Fig. \ref{fig:tri_geo_8_tri_right_angle} is 
%\begin{lstlisting}
%figs/triangle.tex
%\end{lstlisting}
%
%The above latex code can be compiled as a standalone document as
%\begin{lstlisting}
%figs/triangle_fig.tex
%\end{lstlisting}

%

%

%
%

%\end{enumerate}


