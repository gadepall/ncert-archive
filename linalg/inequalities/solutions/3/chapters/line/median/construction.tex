\renewcommand{\theequation}{\theenumi}
\begin{enumerate}[label=\thesection.\arabic*.,ref=\thesection.\theenumi]
\numberwithin{equation}{enumi}

\item The vertices of $\triangle ABC$ are taken as follows:
\label{const:table1}
\\
\solution See Table. \ref{table:table1} 
%
\begin{table}[ht!]
\centering
\input{./tables/median/inp.tex}
\caption{To construct triangle ABC}
\label{table:table1}	
\end{table}

%\item

\item The midpoints of sides are given as
\begin{align}
\vec{D} &= \frac{\vec{B}+\vec{C}}{2}
\label{eq:pointd}
\\
\vec{E} &= \frac{\vec{A}+\vec{C}}{2}
\\
\vec{F} &= \frac{\vec{A}+\vec{B}}{2}
\end{align}
\item Also Centroid divides median in ratio 2:1.So by section formulae $\vec{O}$ is given as
\begin{align}
\vec{O} &= \frac{2\vec{D}+\vec{A}}{3}
\label{eq:centr}
\end{align}

The derived coordinates are listed in \ref{table:table2}. 
%
\begin{table}[ht!]
\centering
\input{./tables/median/out.tex}
\caption{Derived coordinates of triangle ABC}
\label{table:table2}	
\end{table}

\end{enumerate}
