%\renewcommand{\theequation}{\theenumi}
%\begin{enumerate}[label=\arabic*.,ref=\thesection.\theenumi]
%\numberwithin{equation}{enumi}
% \renewcommand{\thefigure}{\theenumi}
% \numberwithin{figure}{enumi}
%
\item Draw Fig. \ref{fig:tri_right_angle} for $a = 4, c =3$.
\label{const:tri_right_angle}
%
\begin{figure}[!ht]
\centering
\resizebox{\columnwidth}{!}{\input{./figs/triangle/tri_right_angle.tex}}
\caption{Right Angled Triangle}
\label{fig:tri_right_angle}	
\end{figure}
\\
\solution The vertices of $\triangle ABC$ are 
\begin{align}
\vec{A} = \myvec{0\\c} = \myvec{0\\3}, \vec{B} = \myvec{0\\0}, \vec{C} = \myvec{a\\0}=\myvec{4\\0}
\end{align}
%
The python code for  Fig. \ref{fig:tri_right_angle} is
\begin{lstlisting}
codes/triangle/tri_right_angle.py
\end{lstlisting}
%
and the equivalent latex-tikz code is
%
\begin{lstlisting}
figs/triangle/tri_right_angle.tex
\end{lstlisting}
%
The above latex code can be compiled as a standalone document as
%
\begin{lstlisting}
figs/triangle/tri_right_angle_alone.tex
\end{lstlisting}
%

\item Draw Fig. \ref{fig:tri_polar} for $a = 4, c =3$.
\label{const:tri_polar}
%
\\
\solution 
 The vertex  $\vec{A}$ can  be expressed  in {\em polar coordinate form} as
%\label{prob:tri_polar}
%
\begin{align}
\vec{A} = b\myvec{\cos \theta\\  \sin \theta} 
\end{align}
%
where
\begin{align}
b = \sqrt{a^2+c^2} = 5, \tan \theta = \frac{3}{4}
\end{align}
%The vertices of $\triangle ABC$ are 
%\begin{align}
%\vec{A} = \myvec{a\\c} = \myvec{4\\3}, \vec{B} = \myvec{a\\0}  = \myvec{4\\0}, \vec{C} = \myvec{0\\0}.
%\end{align}
%
The python code for  Fig. \ref{fig:tri_polar} is
\begin{lstlisting}
codes/triangle/tri_polar.py
\end{lstlisting}
%
and the equivalent latex-tikz code is
%
\begin{lstlisting}
figs/triangle/tri_polar.tex
\end{lstlisting}
\begin{figure}[!ht]
\centering
\resizebox{\columnwidth}{!}{\input{./figs/triangle/tri_polar.tex}}
\caption{Right Angled Triangle}
\label{fig:tri_polar}	
\end{figure}
%
\item Draw Fig. \ref{fig:tri_sss} with $a=6$, $b=5$  and $c=4$.  
\label{const:tri_sss}
\begin{figure}[!ht]
	\begin{center}
			\resizebox{\columnwidth}{!}{\input{./figs/triangle/tri_sss.tex}}
	\end{center}
	\caption{}
	\label{fig:tri_sss}	
\end{figure}
\\
\solution Let the vertices of  $\triangle ABC$ and $\vec{D}$ be 
\begin{align}
\label{eq:tri_basic}
\vec{A} = \myvec{p\\q}, \vec{B} = \myvec{0\\0}, \vec{C} = \myvec{a\\0}, \vec{D} = \myvec{p\\0}
\end{align}
%

Then
\begin{align}
\label{eq:c_tricoord}
AB &= \norm{\vec{A}-\vec{B}}^2 = \norm{\vec{A}}^2  = c^2 \quad \because \vec{B} = \vec{0}
\\
\label{eq:a_tricoord}
BC &= \norm{\vec{C}-\vec{B}}^2 = \norm{\vec{C}}^2  = a^2
\\
AC &= \norm{\vec{A}-\vec{C}}^2 =    b^2
\label{eq:b_tricoord}
\end{align}
%
From \eqref{eq:b_tricoord},
\begin{align}
b^2 &=\norm{\vec{A}-\vec{C}}^2 = \norm{\vec{A}-\vec{C}}^T\norm{\vec{A}-\vec{C}}  
\\
&= \vec{A}^T\vec{A}+\vec{C}^T\vec{C}-\vec{A}^T\vec{C} - \vec{C}^T\vec{A} 
\\
&= \norm{\vec{A}}^2 + \norm{\vec{C}}^2 - 2\vec{A}^T\vec{C} \quad \brak{\because \vec{A}^T\vec{C} = \vec{C}^T\vec{A} } 
\label{eq:tri_const_norm_ac}
\\
&= a^2+c^2-2ap
\end{align}
%
yielding
\begin{align}
p&= \frac{a^2+c^2-b^2}{2a}
\end{align}
%
From \eqref{eq:c_tricoord}, 
\begin{align}
\norm{\vec{A}}^2 &= c^2 = p^2+q^2
\\
\implies q&= \pm \sqrt{c^2-p^2}
\end{align}
%
The python code for  Fig. \ref{fig:tri_sss} is
\begin{lstlisting}
codes/triangle/tri_sss.py
\end{lstlisting}
%
and the equivalent latex-tikz code is
%
\begin{lstlisting}
figs/triangle/tri_sss.tex
\end{lstlisting}

\item Construct a triangle of sides $a=4$, $b=5$  and $c=6$.  
\\
\solution
Data from the given question is available in Table \ref{constr/circ/61/tab:table1}:
%
\begin{table}[!ht]
\begin{center}
\begin{tabular}{ | m{2cm} | m{1.5cm}| m{2cm} | m{1.5cm} |} 
\hline
& Symbols & Circle1  \\
\hline
Centre & $\vec{C}$ & \myvec{0\\0}  \\ 
\hline
Radius & $r$& 3.4 \\ 
\hline
\end{tabular}
\end{center}
\caption{Input values}
\label{constr/circ/61/tab:table1}
\end{table}
%
Let 
\begin{align}
      \vec{P} &= r \myvec{\cos \theta_1\\ \sin \theta_1} =  \myvec{1.7\\2.9}
      \\
      \vec{Q} &= r \myvec{\cos \theta_2\\ \sin \theta_2} =\myvec{-2.4\\2.4}
 \end{align}
 Then the perpendicular bisector of $PQ$ passes through 
 \begin{align}
    \vec{M} = \frac{\vec{P}+\vec{Q}}{2} 
\end{align}
and has normal vector 
\begin{align}
    \vec{n} = \vec{P}-\vec{Q}
\end{align}
resulting in the equation
\begin{align}
    \brak{\vec{P}-\vec{Q}}^\top\brak{\vec{x}-\frac{\vec{P}+\vec{Q}}{2} } = 0
    \implies \brak{\vec{P}-\vec{Q}}^\top\vec{x} = 0
\end{align}
%
after simplification.  It is obvious that $\vec{O}$ satisfies the above equuation as can be verified in Fig.     \ref{constr/circ/61/fig: perpenicular bisector of the chord passes through the center}.
%
\begin{figure}[ht]
    \centering
    \includegraphics[width=\columnwidth]{solutions/circle/61/FIG.3.png}
    \caption{perpendicular bisector of the chord passes through the center}
    \label{constr/circ/61/fig: perpenicular bisector of the chord passes through the center}
\end{figure}








\item Construct an isosceles triangle whose base is $a=8$cm and altitude $AD=h=4$cm 
\\
\solution
Data from the given question is available in Table \ref{constr/circ/61/tab:table1}:
%
\begin{table}[!ht]
\begin{center}
\begin{tabular}{ | m{2cm} | m{1.5cm}| m{2cm} | m{1.5cm} |} 
\hline
& Symbols & Circle1  \\
\hline
Centre & $\vec{C}$ & \myvec{0\\0}  \\ 
\hline
Radius & $r$& 3.4 \\ 
\hline
\end{tabular}
\end{center}
\caption{Input values}
\label{constr/circ/61/tab:table1}
\end{table}
%
Let 
\begin{align}
      \vec{P} &= r \myvec{\cos \theta_1\\ \sin \theta_1} =  \myvec{1.7\\2.9}
      \\
      \vec{Q} &= r \myvec{\cos \theta_2\\ \sin \theta_2} =\myvec{-2.4\\2.4}
 \end{align}
 Then the perpendicular bisector of $PQ$ passes through 
 \begin{align}
    \vec{M} = \frac{\vec{P}+\vec{Q}}{2} 
\end{align}
and has normal vector 
\begin{align}
    \vec{n} = \vec{P}-\vec{Q}
\end{align}
resulting in the equation
\begin{align}
    \brak{\vec{P}-\vec{Q}}^\top\brak{\vec{x}-\frac{\vec{P}+\vec{Q}}{2} } = 0
    \implies \brak{\vec{P}-\vec{Q}}^\top\vec{x} = 0
\end{align}
%
after simplification.  It is obvious that $\vec{O}$ satisfies the above equuation as can be verified in Fig.     \ref{constr/circ/61/fig: perpenicular bisector of the chord passes through the center}.
%
\begin{figure}[ht]
    \centering
    \includegraphics[width=\columnwidth]{solutions/circle/61/FIG.3.png}
    \caption{perpendicular bisector of the chord passes through the center}
    \label{constr/circ/61/fig: perpenicular bisector of the chord passes through the center}
\end{figure}








\item In $\triangle ABC$,  given that $a+b+c = 11, \angle B = 45^{\degree}$ and $\angle C = 45^{\degree}$, 
find 
$a,b,c$ and sketch the triangle.
\\
\solution
Data from the given question is available in Table \ref{constr/circ/61/tab:table1}:
%
\begin{table}[!ht]
\begin{center}
\begin{tabular}{ | m{2cm} | m{1.5cm}| m{2cm} | m{1.5cm} |} 
\hline
& Symbols & Circle1  \\
\hline
Centre & $\vec{C}$ & \myvec{0\\0}  \\ 
\hline
Radius & $r$& 3.4 \\ 
\hline
\end{tabular}
\end{center}
\caption{Input values}
\label{constr/circ/61/tab:table1}
\end{table}
%
Let 
\begin{align}
      \vec{P} &= r \myvec{\cos \theta_1\\ \sin \theta_1} =  \myvec{1.7\\2.9}
      \\
      \vec{Q} &= r \myvec{\cos \theta_2\\ \sin \theta_2} =\myvec{-2.4\\2.4}
 \end{align}
 Then the perpendicular bisector of $PQ$ passes through 
 \begin{align}
    \vec{M} = \frac{\vec{P}+\vec{Q}}{2} 
\end{align}
and has normal vector 
\begin{align}
    \vec{n} = \vec{P}-\vec{Q}
\end{align}
resulting in the equation
\begin{align}
    \brak{\vec{P}-\vec{Q}}^\top\brak{\vec{x}-\frac{\vec{P}+\vec{Q}}{2} } = 0
    \implies \brak{\vec{P}-\vec{Q}}^\top\vec{x} = 0
\end{align}
%
after simplification.  It is obvious that $\vec{O}$ satisfies the above equuation as can be verified in Fig.     \ref{constr/circ/61/fig: perpenicular bisector of the chord passes through the center}.
%
\begin{figure}[ht]
    \centering
    \includegraphics[width=\columnwidth]{solutions/circle/61/FIG.3.png}
    \caption{perpendicular bisector of the chord passes through the center}
    \label{constr/circ/61/fig: perpenicular bisector of the chord passes through the center}
\end{figure}








\item Draw $\triangle ABC$ with $a = 6, c = 5$ and $\angle B = 60 \degree$. 
\\
\solution
Data from the given question is available in Table \ref{constr/circ/61/tab:table1}:
%
\begin{table}[!ht]
\begin{center}
\begin{tabular}{ | m{2cm} | m{1.5cm}| m{2cm} | m{1.5cm} |} 
\hline
& Symbols & Circle1  \\
\hline
Centre & $\vec{C}$ & \myvec{0\\0}  \\ 
\hline
Radius & $r$& 3.4 \\ 
\hline
\end{tabular}
\end{center}
\caption{Input values}
\label{constr/circ/61/tab:table1}
\end{table}
%
Let 
\begin{align}
      \vec{P} &= r \myvec{\cos \theta_1\\ \sin \theta_1} =  \myvec{1.7\\2.9}
      \\
      \vec{Q} &= r \myvec{\cos \theta_2\\ \sin \theta_2} =\myvec{-2.4\\2.4}
 \end{align}
 Then the perpendicular bisector of $PQ$ passes through 
 \begin{align}
    \vec{M} = \frac{\vec{P}+\vec{Q}}{2} 
\end{align}
and has normal vector 
\begin{align}
    \vec{n} = \vec{P}-\vec{Q}
\end{align}
resulting in the equation
\begin{align}
    \brak{\vec{P}-\vec{Q}}^\top\brak{\vec{x}-\frac{\vec{P}+\vec{Q}}{2} } = 0
    \implies \brak{\vec{P}-\vec{Q}}^\top\vec{x} = 0
\end{align}
%
after simplification.  It is obvious that $\vec{O}$ satisfies the above equuation as can be verified in Fig.     \ref{constr/circ/61/fig: perpenicular bisector of the chord passes through the center}.
%
\begin{figure}[ht]
    \centering
    \includegraphics[width=\columnwidth]{solutions/circle/61/FIG.3.png}
    \caption{perpendicular bisector of the chord passes through the center}
    \label{constr/circ/61/fig: perpenicular bisector of the chord passes through the center}
\end{figure}








\item Draw $\triangle ABC$ with $a = 7, \angle B = 45\degree$ and $\angle A = 105 \degree$. 
\\
\solution
Data from the given question is available in Table \ref{constr/circ/61/tab:table1}:
%
\begin{table}[!ht]
\begin{center}
\begin{tabular}{ | m{2cm} | m{1.5cm}| m{2cm} | m{1.5cm} |} 
\hline
& Symbols & Circle1  \\
\hline
Centre & $\vec{C}$ & \myvec{0\\0}  \\ 
\hline
Radius & $r$& 3.4 \\ 
\hline
\end{tabular}
\end{center}
\caption{Input values}
\label{constr/circ/61/tab:table1}
\end{table}
%
Let 
\begin{align}
      \vec{P} &= r \myvec{\cos \theta_1\\ \sin \theta_1} =  \myvec{1.7\\2.9}
      \\
      \vec{Q} &= r \myvec{\cos \theta_2\\ \sin \theta_2} =\myvec{-2.4\\2.4}
 \end{align}
 Then the perpendicular bisector of $PQ$ passes through 
 \begin{align}
    \vec{M} = \frac{\vec{P}+\vec{Q}}{2} 
\end{align}
and has normal vector 
\begin{align}
    \vec{n} = \vec{P}-\vec{Q}
\end{align}
resulting in the equation
\begin{align}
    \brak{\vec{P}-\vec{Q}}^\top\brak{\vec{x}-\frac{\vec{P}+\vec{Q}}{2} } = 0
    \implies \brak{\vec{P}-\vec{Q}}^\top\vec{x} = 0
\end{align}
%
after simplification.  It is obvious that $\vec{O}$ satisfies the above equuation as can be verified in Fig.     \ref{constr/circ/61/fig: perpenicular bisector of the chord passes through the center}.
%
\begin{figure}[ht]
    \centering
    \includegraphics[width=\columnwidth]{solutions/circle/61/FIG.3.png}
    \caption{perpendicular bisector of the chord passes through the center}
    \label{constr/circ/61/fig: perpenicular bisector of the chord passes through the center}
\end{figure}








\item $\triangle ABC$ is right angled at $\vec{B}$.  If $a = 12$ and $b+c = 18$, find $b,c$ and draw the triangle.
\\
\solution
Data from the given question is available in Table \ref{constr/circ/61/tab:table1}:
%
\begin{table}[!ht]
\begin{center}
\begin{tabular}{ | m{2cm} | m{1.5cm}| m{2cm} | m{1.5cm} |} 
\hline
& Symbols & Circle1  \\
\hline
Centre & $\vec{C}$ & \myvec{0\\0}  \\ 
\hline
Radius & $r$& 3.4 \\ 
\hline
\end{tabular}
\end{center}
\caption{Input values}
\label{constr/circ/61/tab:table1}
\end{table}
%
Let 
\begin{align}
      \vec{P} &= r \myvec{\cos \theta_1\\ \sin \theta_1} =  \myvec{1.7\\2.9}
      \\
      \vec{Q} &= r \myvec{\cos \theta_2\\ \sin \theta_2} =\myvec{-2.4\\2.4}
 \end{align}
 Then the perpendicular bisector of $PQ$ passes through 
 \begin{align}
    \vec{M} = \frac{\vec{P}+\vec{Q}}{2} 
\end{align}
and has normal vector 
\begin{align}
    \vec{n} = \vec{P}-\vec{Q}
\end{align}
resulting in the equation
\begin{align}
    \brak{\vec{P}-\vec{Q}}^\top\brak{\vec{x}-\frac{\vec{P}+\vec{Q}}{2} } = 0
    \implies \brak{\vec{P}-\vec{Q}}^\top\vec{x} = 0
\end{align}
%
after simplification.  It is obvious that $\vec{O}$ satisfies the above equuation as can be verified in Fig.     \ref{constr/circ/61/fig: perpenicular bisector of the chord passes through the center}.
%
\begin{figure}[ht]
    \centering
    \includegraphics[width=\columnwidth]{solutions/circle/61/FIG.3.png}
    \caption{perpendicular bisector of the chord passes through the center}
    \label{constr/circ/61/fig: perpenicular bisector of the chord passes through the center}
\end{figure}







%\item Draw $\triangle ABC$ if $AB = 3, AC = 5$ and $\angle C = 30 \degree$.
\item In $\triangle ABC$,  $a = 8, \angle B = 45^{\degree}$ and $c-b = 3.5$.
Sketch $\triangle ABC$.
\\
\solution
Data from the given question is available in Table \ref{constr/circ/61/tab:table1}:
%
\begin{table}[!ht]
\begin{center}
\begin{tabular}{ | m{2cm} | m{1.5cm}| m{2cm} | m{1.5cm} |} 
\hline
& Symbols & Circle1  \\
\hline
Centre & $\vec{C}$ & \myvec{0\\0}  \\ 
\hline
Radius & $r$& 3.4 \\ 
\hline
\end{tabular}
\end{center}
\caption{Input values}
\label{constr/circ/61/tab:table1}
\end{table}
%
Let 
\begin{align}
      \vec{P} &= r \myvec{\cos \theta_1\\ \sin \theta_1} =  \myvec{1.7\\2.9}
      \\
      \vec{Q} &= r \myvec{\cos \theta_2\\ \sin \theta_2} =\myvec{-2.4\\2.4}
 \end{align}
 Then the perpendicular bisector of $PQ$ passes through 
 \begin{align}
    \vec{M} = \frac{\vec{P}+\vec{Q}}{2} 
\end{align}
and has normal vector 
\begin{align}
    \vec{n} = \vec{P}-\vec{Q}
\end{align}
resulting in the equation
\begin{align}
    \brak{\vec{P}-\vec{Q}}^\top\brak{\vec{x}-\frac{\vec{P}+\vec{Q}}{2} } = 0
    \implies \brak{\vec{P}-\vec{Q}}^\top\vec{x} = 0
\end{align}
%
after simplification.  It is obvious that $\vec{O}$ satisfies the above equuation as can be verified in Fig.     \ref{constr/circ/61/fig: perpenicular bisector of the chord passes through the center}.
%
\begin{figure}[ht]
    \centering
    \includegraphics[width=\columnwidth]{solutions/circle/61/FIG.3.png}
    \caption{perpendicular bisector of the chord passes through the center}
    \label{constr/circ/61/fig: perpenicular bisector of the chord passes through the center}
\end{figure}








\item In $\triangle ABC$,  $a = 6, \angle B = 60^{\degree}$ and $b-c = 2$. 
Sketch $\triangle ABC$.
Data from the given question is available in Table \ref{constr/circ/61/tab:table1}:
%
\begin{table}[!ht]
\begin{center}
\begin{tabular}{ | m{2cm} | m{1.5cm}| m{2cm} | m{1.5cm} |} 
\hline
& Symbols & Circle1  \\
\hline
Centre & $\vec{C}$ & \myvec{0\\0}  \\ 
\hline
Radius & $r$& 3.4 \\ 
\hline
\end{tabular}
\end{center}
\caption{Input values}
\label{constr/circ/61/tab:table1}
\end{table}
%
Let 
\begin{align}
      \vec{P} &= r \myvec{\cos \theta_1\\ \sin \theta_1} =  \myvec{1.7\\2.9}
      \\
      \vec{Q} &= r \myvec{\cos \theta_2\\ \sin \theta_2} =\myvec{-2.4\\2.4}
 \end{align}
 Then the perpendicular bisector of $PQ$ passes through 
 \begin{align}
    \vec{M} = \frac{\vec{P}+\vec{Q}}{2} 
\end{align}
and has normal vector 
\begin{align}
    \vec{n} = \vec{P}-\vec{Q}
\end{align}
resulting in the equation
\begin{align}
    \brak{\vec{P}-\vec{Q}}^\top\brak{\vec{x}-\frac{\vec{P}+\vec{Q}}{2} } = 0
    \implies \brak{\vec{P}-\vec{Q}}^\top\vec{x} = 0
\end{align}
%
after simplification.  It is obvious that $\vec{O}$ satisfies the above equuation as can be verified in Fig.     \ref{constr/circ/61/fig: perpenicular bisector of the chord passes through the center}.
%
\begin{figure}[ht]
    \centering
    \includegraphics[width=\columnwidth]{solutions/circle/61/FIG.3.png}
    \caption{perpendicular bisector of the chord passes through the center}
    \label{constr/circ/61/fig: perpenicular bisector of the chord passes through the center}
\end{figure}







\item Draw $\triangle ABC$,  given that $a+b+c = 11, \angle B = 30^{\degree}$ and $\angle C = 90^{\degree}$.
\\
\solution
Data from the given question is available in Table \ref{constr/circ/61/tab:table1}:
%
\begin{table}[!ht]
\begin{center}
\begin{tabular}{ | m{2cm} | m{1.5cm}| m{2cm} | m{1.5cm} |} 
\hline
& Symbols & Circle1  \\
\hline
Centre & $\vec{C}$ & \myvec{0\\0}  \\ 
\hline
Radius & $r$& 3.4 \\ 
\hline
\end{tabular}
\end{center}
\caption{Input values}
\label{constr/circ/61/tab:table1}
\end{table}
%
Let 
\begin{align}
      \vec{P} &= r \myvec{\cos \theta_1\\ \sin \theta_1} =  \myvec{1.7\\2.9}
      \\
      \vec{Q} &= r \myvec{\cos \theta_2\\ \sin \theta_2} =\myvec{-2.4\\2.4}
 \end{align}
 Then the perpendicular bisector of $PQ$ passes through 
 \begin{align}
    \vec{M} = \frac{\vec{P}+\vec{Q}}{2} 
\end{align}
and has normal vector 
\begin{align}
    \vec{n} = \vec{P}-\vec{Q}
\end{align}
resulting in the equation
\begin{align}
    \brak{\vec{P}-\vec{Q}}^\top\brak{\vec{x}-\frac{\vec{P}+\vec{Q}}{2} } = 0
    \implies \brak{\vec{P}-\vec{Q}}^\top\vec{x} = 0
\end{align}
%
after simplification.  It is obvious that $\vec{O}$ satisfies the above equuation as can be verified in Fig.     \ref{constr/circ/61/fig: perpenicular bisector of the chord passes through the center}.
%
\begin{figure}[ht]
    \centering
    \includegraphics[width=\columnwidth]{solutions/circle/61/FIG.3.png}
    \caption{perpendicular bisector of the chord passes through the center}
    \label{constr/circ/61/fig: perpenicular bisector of the chord passes through the center}
\end{figure}








\item Construct $\triangle xyz$ where $xy = 4.5, yz = 5$ and $zx = 6$.
\\
\solution
Data from the given question is available in Table \ref{constr/circ/61/tab:table1}:
%
\begin{table}[!ht]
\begin{center}
\begin{tabular}{ | m{2cm} | m{1.5cm}| m{2cm} | m{1.5cm} |} 
\hline
& Symbols & Circle1  \\
\hline
Centre & $\vec{C}$ & \myvec{0\\0}  \\ 
\hline
Radius & $r$& 3.4 \\ 
\hline
\end{tabular}
\end{center}
\caption{Input values}
\label{constr/circ/61/tab:table1}
\end{table}
%
Let 
\begin{align}
      \vec{P} &= r \myvec{\cos \theta_1\\ \sin \theta_1} =  \myvec{1.7\\2.9}
      \\
      \vec{Q} &= r \myvec{\cos \theta_2\\ \sin \theta_2} =\myvec{-2.4\\2.4}
 \end{align}
 Then the perpendicular bisector of $PQ$ passes through 
 \begin{align}
    \vec{M} = \frac{\vec{P}+\vec{Q}}{2} 
\end{align}
and has normal vector 
\begin{align}
    \vec{n} = \vec{P}-\vec{Q}
\end{align}
resulting in the equation
\begin{align}
    \brak{\vec{P}-\vec{Q}}^\top\brak{\vec{x}-\frac{\vec{P}+\vec{Q}}{2} } = 0
    \implies \brak{\vec{P}-\vec{Q}}^\top\vec{x} = 0
\end{align}
%
after simplification.  It is obvious that $\vec{O}$ satisfies the above equuation as can be verified in Fig.     \ref{constr/circ/61/fig: perpenicular bisector of the chord passes through the center}.
%
\begin{figure}[ht]
    \centering
    \includegraphics[width=\columnwidth]{solutions/circle/61/FIG.3.png}
    \caption{perpendicular bisector of the chord passes through the center}
    \label{constr/circ/61/fig: perpenicular bisector of the chord passes through the center}
\end{figure}








\item Draw an equilateral triangle of side $5.5$.
\\
\solution
Data from the given question is available in Table \ref{constr/circ/61/tab:table1}:
%
\begin{table}[!ht]
\begin{center}
\begin{tabular}{ | m{2cm} | m{1.5cm}| m{2cm} | m{1.5cm} |} 
\hline
& Symbols & Circle1  \\
\hline
Centre & $\vec{C}$ & \myvec{0\\0}  \\ 
\hline
Radius & $r$& 3.4 \\ 
\hline
\end{tabular}
\end{center}
\caption{Input values}
\label{constr/circ/61/tab:table1}
\end{table}
%
Let 
\begin{align}
      \vec{P} &= r \myvec{\cos \theta_1\\ \sin \theta_1} =  \myvec{1.7\\2.9}
      \\
      \vec{Q} &= r \myvec{\cos \theta_2\\ \sin \theta_2} =\myvec{-2.4\\2.4}
 \end{align}
 Then the perpendicular bisector of $PQ$ passes through 
 \begin{align}
    \vec{M} = \frac{\vec{P}+\vec{Q}}{2} 
\end{align}
and has normal vector 
\begin{align}
    \vec{n} = \vec{P}-\vec{Q}
\end{align}
resulting in the equation
\begin{align}
    \brak{\vec{P}-\vec{Q}}^\top\brak{\vec{x}-\frac{\vec{P}+\vec{Q}}{2} } = 0
    \implies \brak{\vec{P}-\vec{Q}}^\top\vec{x} = 0
\end{align}
%
after simplification.  It is obvious that $\vec{O}$ satisfies the above equuation as can be verified in Fig.     \ref{constr/circ/61/fig: perpenicular bisector of the chord passes through the center}.
%
\begin{figure}[ht]
    \centering
    \includegraphics[width=\columnwidth]{solutions/circle/61/FIG.3.png}
    \caption{perpendicular bisector of the chord passes through the center}
    \label{constr/circ/61/fig: perpenicular bisector of the chord passes through the center}
\end{figure}








\item Draw $\triangle PQR$ with $PQ = 4, QR = 3.5$ and $PR = 4$.  What type of triangle is this?
\\
\solution
Data from the given question is available in Table \ref{constr/circ/61/tab:table1}:
%
\begin{table}[!ht]
\begin{center}
\begin{tabular}{ | m{2cm} | m{1.5cm}| m{2cm} | m{1.5cm} |} 
\hline
& Symbols & Circle1  \\
\hline
Centre & $\vec{C}$ & \myvec{0\\0}  \\ 
\hline
Radius & $r$& 3.4 \\ 
\hline
\end{tabular}
\end{center}
\caption{Input values}
\label{constr/circ/61/tab:table1}
\end{table}
%
Let 
\begin{align}
      \vec{P} &= r \myvec{\cos \theta_1\\ \sin \theta_1} =  \myvec{1.7\\2.9}
      \\
      \vec{Q} &= r \myvec{\cos \theta_2\\ \sin \theta_2} =\myvec{-2.4\\2.4}
 \end{align}
 Then the perpendicular bisector of $PQ$ passes through 
 \begin{align}
    \vec{M} = \frac{\vec{P}+\vec{Q}}{2} 
\end{align}
and has normal vector 
\begin{align}
    \vec{n} = \vec{P}-\vec{Q}
\end{align}
resulting in the equation
\begin{align}
    \brak{\vec{P}-\vec{Q}}^\top\brak{\vec{x}-\frac{\vec{P}+\vec{Q}}{2} } = 0
    \implies \brak{\vec{P}-\vec{Q}}^\top\vec{x} = 0
\end{align}
%
after simplification.  It is obvious that $\vec{O}$ satisfies the above equuation as can be verified in Fig.     \ref{constr/circ/61/fig: perpenicular bisector of the chord passes through the center}.
%
\begin{figure}[ht]
    \centering
    \includegraphics[width=\columnwidth]{solutions/circle/61/FIG.3.png}
    \caption{perpendicular bisector of the chord passes through the center}
    \label{constr/circ/61/fig: perpenicular bisector of the chord passes through the center}
\end{figure}








\item Construct $\triangle ABC$ such that $AB = 2.5, BC = 6$ and $AC = 6.5$.  Find $\angle B$.
\\
\solution
Data from the given question is available in Table \ref{constr/circ/61/tab:table1}:
%
\begin{table}[!ht]
\begin{center}
\begin{tabular}{ | m{2cm} | m{1.5cm}| m{2cm} | m{1.5cm} |} 
\hline
& Symbols & Circle1  \\
\hline
Centre & $\vec{C}$ & \myvec{0\\0}  \\ 
\hline
Radius & $r$& 3.4 \\ 
\hline
\end{tabular}
\end{center}
\caption{Input values}
\label{constr/circ/61/tab:table1}
\end{table}
%
Let 
\begin{align}
      \vec{P} &= r \myvec{\cos \theta_1\\ \sin \theta_1} =  \myvec{1.7\\2.9}
      \\
      \vec{Q} &= r \myvec{\cos \theta_2\\ \sin \theta_2} =\myvec{-2.4\\2.4}
 \end{align}
 Then the perpendicular bisector of $PQ$ passes through 
 \begin{align}
    \vec{M} = \frac{\vec{P}+\vec{Q}}{2} 
\end{align}
and has normal vector 
\begin{align}
    \vec{n} = \vec{P}-\vec{Q}
\end{align}
resulting in the equation
\begin{align}
    \brak{\vec{P}-\vec{Q}}^\top\brak{\vec{x}-\frac{\vec{P}+\vec{Q}}{2} } = 0
    \implies \brak{\vec{P}-\vec{Q}}^\top\vec{x} = 0
\end{align}
%
after simplification.  It is obvious that $\vec{O}$ satisfies the above equuation as can be verified in Fig.     \ref{constr/circ/61/fig: perpenicular bisector of the chord passes through the center}.
%
\begin{figure}[ht]
    \centering
    \includegraphics[width=\columnwidth]{solutions/circle/61/FIG.3.png}
    \caption{perpendicular bisector of the chord passes through the center}
    \label{constr/circ/61/fig: perpenicular bisector of the chord passes through the center}
\end{figure}








\item Construct $\triangle DEF$ such that $DE = 5, DF = 3$ and $\angle D = 90\degree$.
\\
\solution
Data from the given question is available in Table \ref{constr/circ/61/tab:table1}:
%
\begin{table}[!ht]
\begin{center}
\begin{tabular}{ | m{2cm} | m{1.5cm}| m{2cm} | m{1.5cm} |} 
\hline
& Symbols & Circle1  \\
\hline
Centre & $\vec{C}$ & \myvec{0\\0}  \\ 
\hline
Radius & $r$& 3.4 \\ 
\hline
\end{tabular}
\end{center}
\caption{Input values}
\label{constr/circ/61/tab:table1}
\end{table}
%
Let 
\begin{align}
      \vec{P} &= r \myvec{\cos \theta_1\\ \sin \theta_1} =  \myvec{1.7\\2.9}
      \\
      \vec{Q} &= r \myvec{\cos \theta_2\\ \sin \theta_2} =\myvec{-2.4\\2.4}
 \end{align}
 Then the perpendicular bisector of $PQ$ passes through 
 \begin{align}
    \vec{M} = \frac{\vec{P}+\vec{Q}}{2} 
\end{align}
and has normal vector 
\begin{align}
    \vec{n} = \vec{P}-\vec{Q}
\end{align}
resulting in the equation
\begin{align}
    \brak{\vec{P}-\vec{Q}}^\top\brak{\vec{x}-\frac{\vec{P}+\vec{Q}}{2} } = 0
    \implies \brak{\vec{P}-\vec{Q}}^\top\vec{x} = 0
\end{align}
%
after simplification.  It is obvious that $\vec{O}$ satisfies the above equuation as can be verified in Fig.     \ref{constr/circ/61/fig: perpenicular bisector of the chord passes through the center}.
%
\begin{figure}[ht]
    \centering
    \includegraphics[width=\columnwidth]{solutions/circle/61/FIG.3.png}
    \caption{perpendicular bisector of the chord passes through the center}
    \label{constr/circ/61/fig: perpenicular bisector of the chord passes through the center}
\end{figure}








\item Construct an isosceles triangle in which the lengths of the equal sides is 6.5 and the angle between them is $110\degree$.
\\
\solution
Data from the given question is available in Table \ref{constr/circ/61/tab:table1}:
%
\begin{table}[!ht]
\begin{center}
\begin{tabular}{ | m{2cm} | m{1.5cm}| m{2cm} | m{1.5cm} |} 
\hline
& Symbols & Circle1  \\
\hline
Centre & $\vec{C}$ & \myvec{0\\0}  \\ 
\hline
Radius & $r$& 3.4 \\ 
\hline
\end{tabular}
\end{center}
\caption{Input values}
\label{constr/circ/61/tab:table1}
\end{table}
%
Let 
\begin{align}
      \vec{P} &= r \myvec{\cos \theta_1\\ \sin \theta_1} =  \myvec{1.7\\2.9}
      \\
      \vec{Q} &= r \myvec{\cos \theta_2\\ \sin \theta_2} =\myvec{-2.4\\2.4}
 \end{align}
 Then the perpendicular bisector of $PQ$ passes through 
 \begin{align}
    \vec{M} = \frac{\vec{P}+\vec{Q}}{2} 
\end{align}
and has normal vector 
\begin{align}
    \vec{n} = \vec{P}-\vec{Q}
\end{align}
resulting in the equation
\begin{align}
    \brak{\vec{P}-\vec{Q}}^\top\brak{\vec{x}-\frac{\vec{P}+\vec{Q}}{2} } = 0
    \implies \brak{\vec{P}-\vec{Q}}^\top\vec{x} = 0
\end{align}
%
after simplification.  It is obvious that $\vec{O}$ satisfies the above equuation as can be verified in Fig.     \ref{constr/circ/61/fig: perpenicular bisector of the chord passes through the center}.
%
\begin{figure}[ht]
    \centering
    \includegraphics[width=\columnwidth]{solutions/circle/61/FIG.3.png}
    \caption{perpendicular bisector of the chord passes through the center}
    \label{constr/circ/61/fig: perpenicular bisector of the chord passes through the center}
\end{figure}







\item Construct $\triangle ABC$ given that $\angle A = 60\degree, \angle B = 30\degree$ and $AB = 5.8$.
\\
\solution
From the given information, 
\begin{align}
\angle C = 90^{\degree}
\end{align}
Hence, 
\begin{align}
\vec{A}&=\myvec{0\\c\sin{B}}\\
              &=\myvec{0\\2.9}\\
\vec{B} &=\myvec{c\cos{B}\\0}\\
               &=\myvec {5.02294\\0}\\
\vec{C} &=\myvec{0\\0}
\end{align}
which are used to draw $\triangle ABC$ in Fig. \ref{triangle/20/fig:triangle ABC}.
%
\begin{figure}[!ht]
\centering
\includegraphics[width=\columnwidth]{solutions/triangle/20/Figure01.png}
\caption{$\triangle ABC$}
\label{triangle/20/fig:triangle ABC}
\end{figure}    

\item Construct  $\triangle LMN$ right angled at $M$ such that $LN = 5$ and $MN = 3$.
 \\
 \solution
Data from the given question is available in Table \ref{constr/circ/61/tab:table1}:
%
\begin{table}[!ht]
\begin{center}
\begin{tabular}{ | m{2cm} | m{1.5cm}| m{2cm} | m{1.5cm} |} 
\hline
& Symbols & Circle1  \\
\hline
Centre & $\vec{C}$ & \myvec{0\\0}  \\ 
\hline
Radius & $r$& 3.4 \\ 
\hline
\end{tabular}
\end{center}
\caption{Input values}
\label{constr/circ/61/tab:table1}
\end{table}
%
Let 
\begin{align}
      \vec{P} &= r \myvec{\cos \theta_1\\ \sin \theta_1} =  \myvec{1.7\\2.9}
      \\
      \vec{Q} &= r \myvec{\cos \theta_2\\ \sin \theta_2} =\myvec{-2.4\\2.4}
 \end{align}
 Then the perpendicular bisector of $PQ$ passes through 
 \begin{align}
    \vec{M} = \frac{\vec{P}+\vec{Q}}{2} 
\end{align}
and has normal vector 
\begin{align}
    \vec{n} = \vec{P}-\vec{Q}
\end{align}
resulting in the equation
\begin{align}
    \brak{\vec{P}-\vec{Q}}^\top\brak{\vec{x}-\frac{\vec{P}+\vec{Q}}{2} } = 0
    \implies \brak{\vec{P}-\vec{Q}}^\top\vec{x} = 0
\end{align}
%
after simplification.  It is obvious that $\vec{O}$ satisfies the above equuation as can be verified in Fig.     \ref{constr/circ/61/fig: perpenicular bisector of the chord passes through the center}.
%
\begin{figure}[ht]
    \centering
    \includegraphics[width=\columnwidth]{solutions/circle/61/FIG.3.png}
    \caption{perpendicular bisector of the chord passes through the center}
    \label{constr/circ/61/fig: perpenicular bisector of the chord passes through the center}
\end{figure}







\item Construct  $\triangle PQR$ right angled at $Q$ such that $QR = 8$ and $PR = 10$.
\\
\solution
Let 
%
\begin{align}
\vec{Q} = \myvec{0\\0}, \vec{R} = \myvec{8\\0}, \vec{P} = \myvec{0\\p}
\end{align}
%
Then,
\begin{align}
\norm{\vec{P}-\vec{R}}^2 &= ({\vec{P}-\vec{R}})^T({\vec{P}-\vec{R}})
\\
&= \norm{\vec{P}}^2 + \norm{\vec{R}}^2 
\\
\because \vec{P}^T\vec{R} &= \vec{R}^T\vec{P}, \vec{R}^T\vec{P}=0
\\
&= p^2 + 64 = 10^2
 \\
\implies p &=\pm 6
\end{align}
Since positive area is considered here,only $p$ = 6 is taken into consideration. Thus, 
\begin{align}
    \vec{P} = \myvec{0\\6}
    \end{align}
    and the desired traingle is plotted in Fig. \ref{constr/24/fig:right_angle_triangle}
%
\begin{figure}[!ht]
\centering
\includegraphics[width=\columnwidth]{solutions/24/Figure1.png}
\caption{Right Angle $\triangle PQR$}
\label{constr/24/fig:right_angle_triangle}	
\end{figure}

\item Construct  right angled $\triangle $ whose hypotenuse  is 6 and one of the legs is 4.
\\
\solution
From the given information,
%
\begin{align}
\angle{C} = \ang{60}
\end{align}
%
Using the sine formula, 
%
\begin{align}
c &= b \brak{\frac{\sin{C}}{\sin{B}}} 
\\
&= 3.3915
\end{align}
%
the vertices of $\triangle ABC$ are
\begin{align}
\vec{A} = \myvec{0 \\ 0},
\vec{B} = c\myvec{\cos{\ang{70}} \\ \sin{\ang{70}}},
\vec{C} = \myvec{3 \\ 0}
\end{align}
and  plotted in Fig. \ref{constr/tri/27/3/fig:triangle ABC}.
%
\begin{figure}[ht]
    \centering
    \includegraphics[width=\columnwidth]{solutions/triangle/27/3/Triangle_ABC.PNG}
    \caption{Plot of $\triangle ABC$}
    \label{constr/tri/27/3/fig:triangle ABC}
\end{figure}



\item Construct  an isosceles right angled $\triangle ABC$ right angled at $C$ such $AC = 6$.
\\
\solution

$\because \triangle ABC$ is isosceles, its vertices are
\begin{align}
\vec{C} = \myvec{0\\0},
\vec{A} = \myvec{6\\0}, 
\vec{B} = \myvec{0\\6}
\end{align}
which are used to plot the desired triangle in Fig. \ref{constr/26/fig:right_angle_triangle}.	
%
\begin{figure}[!ht]
\centering
\includegraphics[width=\columnwidth]{solutions/26/diagram-1.png}
\caption{Isosceles Right Angle $\triangle ABC$}
\label{constr/26/fig:right_angle_triangle}	
\end{figure}



\item Construct $\triangle PQR$, given that $PQ = 3, QR = 5.5$ and $\angle PQR = 60 \degree$.
\\
\solution
\begin{align}
    \because \vec{P} = r\myvec{\cos Q\\ \sin Q}, \vec{Q} = \myvec{0\\0}, \vec{R} = \myvec{p\\0}
    \end{align}
    from the given information, 
    \begin{align}
    \vec{P} = 3\myvec{\cos60\\ \sin60} = \frac{3}{2}\myvec{1\\\sqrt3},  \vec{Q} = \myvec{0\\0},  \vec{R} = \myvec{5.5\\0}
    \end{align}
and plotted in Fig.              \ref{constr/july/1Figure}.
    
    
\begin{figure}[!ht]
\centering
         \includegraphics[width= \columnwidth]{solutions/july/2/1/figure1.png}
         \caption{The Constructed triangle}
         \label{constr/july/1Figure}
\end{figure}






%
\item Construct $\triangle ABC$  with $BC = 7.5, AC = 5$ and $\angle C = 60\degree$.
\\
\solution
Let 
%
\begin{align}
\vec{Q} = \myvec{0\\0}, \vec{R} = \myvec{8\\0}, \vec{P} = \myvec{0\\p}
\end{align}
%
Then,
\begin{align}
\norm{\vec{P}-\vec{R}}^2 &= ({\vec{P}-\vec{R}})^T({\vec{P}-\vec{R}})
\\
&= \norm{\vec{P}}^2 + \norm{\vec{R}}^2 
\\
\because \vec{P}^T\vec{R} &= \vec{R}^T\vec{P}, \vec{R}^T\vec{P}=0
\\
&= p^2 + 64 = 10^2
 \\
\implies p &=\pm 6
\end{align}
Since positive area is considered here,only $p$ = 6 is taken into consideration. Thus, 
\begin{align}
    \vec{P} = \myvec{0\\6}
    \end{align}
    and the desired traingle is plotted in Fig. \ref{constr/24/fig:right_angle_triangle}
%
\begin{figure}[!ht]
\centering
\includegraphics[width=\columnwidth]{solutions/24/Figure1.png}
\caption{Right Angle $\triangle PQR$}
\label{constr/24/fig:right_angle_triangle}	
\end{figure}

\item Construct $\triangle XYZ$ if $XY = 6, \angle X = 30\degree$ and $\angle Y = 100 \degree$.
\\
\solution
Let
%
\begin{align}
 XY=z,  YZ=x,  XZ=y
\end{align}
From the given information,
%
\begin{align}
y&= z\brak{\frac{\sin{Y}}{\sin{Z}}} 
&= 6 \brak{\frac{\sin{\ang{100}}}{\sin{\ang{50}}}}
&= 7.7134
\end{align}
and the vertices are
\begin{align}
\vec{X} &= \myvec{\ 0\\ 0}, \vec{Y} = z\myvec{ \cos X\si{\degree}\\ \sin X\si{\degree}} ,\vec{Z} =\myvec{y \\ 0} \\
\\
\implies \vec{X} &=\myvec{0 \\ 0},
\vec{Y} =6\myvec{\cos{\ang{30}} \\ \sin{\ang{30}}} &= \myvec{3\sqrt{3} \\ 3 }, 
\vec{Z} =\myvec{7.7134 \\ 0}
\end{align}
and plotted in Fig. \ref{july/2/3/Figure}.
%
\begin{figure}[!b]
  \centering
            \includegraphics[width=\columnwidth]{solutions/july/2/3/Figure_1.png}
            \caption{Constructed Triangle}
            \label{july/2/3/Figure}
 \end{figure}
 %
 



\item If $AC = 7, \angle A = 60\degree$ and $\angle B = 50 \degree$, can you draw the triangle?
%
\\
\solution
From the given information, 
%
\begin{align}
\angle{C} = \ang{70}
\end{align}
and 
\begin{align}
  a&= b\brak{\frac{\sin{A}}{\sin{B}}} 
\\
  &= 7.913611
  \end{align}
%  
Thus, the coordinates are 
\begin{align}
  \vec{C} =\myvec{0 \\ 0},
  \vec{B} =a\myvec{\cos{\ang{70}} \\ \sin{\ang{70}}},
  \vec{A} =\myvec{7 \\ 0}
  \end{align}
and plotted in Fig. \ref{july/2/4/fig:triangle}.	  
%
\begin{figure}[!h]
\centering
\includegraphics[width=\columnwidth]{solutions/july/2/4/Figure 1.jpeg}
\caption{$\triangle ABC$}
\label{july/2/4/fig:triangle}	
\end{figure}




\item Construct $\triangle PQR$ if $PQ = 5, \angle Q = 105 \degree$ and $\angle R = 40 \degree$.
\\
Data from the given question is available in Table \ref{constr/circ/61/tab:table1}:
%
\begin{table}[!ht]
\begin{center}
\begin{tabular}{ | m{2cm} | m{1.5cm}| m{2cm} | m{1.5cm} |} 
\hline
& Symbols & Circle1  \\
\hline
Centre & $\vec{C}$ & \myvec{0\\0}  \\ 
\hline
Radius & $r$& 3.4 \\ 
\hline
\end{tabular}
\end{center}
\caption{Input values}
\label{constr/circ/61/tab:table1}
\end{table}
%
Let 
\begin{align}
      \vec{P} &= r \myvec{\cos \theta_1\\ \sin \theta_1} =  \myvec{1.7\\2.9}
      \\
      \vec{Q} &= r \myvec{\cos \theta_2\\ \sin \theta_2} =\myvec{-2.4\\2.4}
 \end{align}
 Then the perpendicular bisector of $PQ$ passes through 
 \begin{align}
    \vec{M} = \frac{\vec{P}+\vec{Q}}{2} 
\end{align}
and has normal vector 
\begin{align}
    \vec{n} = \vec{P}-\vec{Q}
\end{align}
resulting in the equation
\begin{align}
    \brak{\vec{P}-\vec{Q}}^\top\brak{\vec{x}-\frac{\vec{P}+\vec{Q}}{2} } = 0
    \implies \brak{\vec{P}-\vec{Q}}^\top\vec{x} = 0
\end{align}
%
after simplification.  It is obvious that $\vec{O}$ satisfies the above equuation as can be verified in Fig.     \ref{constr/circ/61/fig: perpenicular bisector of the chord passes through the center}.
%
\begin{figure}[ht]
    \centering
    \includegraphics[width=\columnwidth]{solutions/circle/61/FIG.3.png}
    \caption{perpendicular bisector of the chord passes through the center}
    \label{constr/circ/61/fig: perpenicular bisector of the chord passes through the center}
\end{figure}







\item Can you construct $\triangle DEF$ such that $EF = 7.2, \angle E = 110\degree$ and $\angle F = 180\degree$?



%\end{enumerate}

