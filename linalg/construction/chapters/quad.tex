%\renewcommand{\theequation}{\theenumi}
%\begin{enumerate}[label=\arabic*.,ref=\thesection.\theenumi]
%\numberwithin{equation}{enumi}
	%
%
% \renewcommand{\theequation}{\theenumi}
% \numberwithin{figure}{enumi}

\item Construct parallelogram $ABCD$ 	in Fig. \ref{fig:pgm_sas}	
given that  $BC = 5, AB = 6, \angle C = 85 \degree$.
\begin{figure}[!ht]
	\begin{center}
		\resizebox{\columnwidth}{!}{\input{./figs/quad/pgm_sas.tex}}
	\end{center}
	\caption{Parallelogram Properties}
	\label{fig:pgm_sas}	
\end{figure}
%
\\
\solution $BD$ is found using the cosine formula and $\triangle BDC$ is drawn using the approach in Construction \ref{const:tri_sss} with 
%
\begin{align}
\vec{B} = \myvec{0\\0},
\vec{C} = \myvec{5\\0},
\end{align}
%
Since the diagonals bisect each other, 
%
\begin{align}
\vec{O} &= \frac{\vec{B}+\vec{D}}{2}
\\
\vec{A} &= 2\vec{O} - \vec{C}.
\end{align}
%
$AB$ and $AD$ are then joined to complete the $\parallel$gm.
The python code for  Fig. \ref{fig:pgm_sas} is
\begin{lstlisting}
codes/quad/pgm_sas.py
\end{lstlisting}
%
and 
The equivalent latex-tikz code is
%
\begin{lstlisting}
figs/quad/pgm_sas.tex
\end{lstlisting}
%

\item Draw the $\parallel$gm $ABCD$ in 	Fig. \ref{fig:pgm_sss}	
with $BC = 6, CD = 4.5$ and $BD=7.5$.  Show that it is a rectangle.
\label{const:pgm_sss}
%
\begin{figure}[!ht]
	\begin{center}
		\resizebox{\columnwidth}{!}{\input{./figs/quad/pgm_sss.tex}}
	\end{center}
	\caption{Rectangle}
	\label{fig:pgm_sss}	
\end{figure}
\\
\solution It is easy to verify that 
%Using the approach in Construction\ref{const:tri_sss}, $\triangle BCD$ is drawn with
%
\begin{align}
BD^2=BC^2+C^2
\end{align}
%
Hence, using Baudhayana theorem, 
%
\begin{align}
\angle BCD = 90\degree
\end{align}
%
and  $ABCD$ is a rectangle.
\begin{align}
\vec{A} = \myvec{0\\4.5}
\vec{B} = \myvec{0\\0}
\vec{C} = \myvec{6\\0}
\vec{D} = \myvec{6\\4}
\end{align}
%
The python code for  Fig. \ref{fig:pgm_sss} is
\begin{lstlisting}
codes/quad/pgm_sss.py
\end{lstlisting}
%
and the equivalent latex-tikz code is
%
\begin{lstlisting}
figs/quad/pgm_sss.tex
\end{lstlisting}
%
%
%
%
%
\item Draw the rhombus $BEST$ with $BE = 4.5$ and $ET = 6$. 
\begin{figure}[!ht]
	\begin{center}
		\resizebox{\columnwidth}{!}{\input{./figs/quad/rhom_sss.tex}}
	\end{center}
	\caption{Rhombus}
	\label{fig:rhom_sss}	
\end{figure}
\\
\solution The coordinates of the various points in Fig. \ref{fig:rhom_sss} are obtained as
%
\begin{align}
\vec{O} = \myvec{0\\0},
\vec{B} = \myvec{0\\-4.5}
\\
\vec{E} = \myvec{3\\0},
\vec{S} = \myvec{4.5\\0},
\vec{T} = \myvec{0\\-3}
\end{align}
%
\item A square is a rectangle whose sides are equal.  Draw a square of side 4.5.
\\
\solution The coordinates of the various points in Fig. \ref{fig:square} are obtained as
%
\begin{align}
\vec{A} = \myvec{0\\4.5}
\\
\vec{B} = \myvec{0\\0},
\vec{C} = \myvec{4.5\\0},
\vec{D} = \myvec{4.5\\4.5}
\vec{O} = \frac{\vec{B}+\vec{C}}{2}
%
\end{align}
%
\begin{figure}[!ht]
	\begin{center}
		\resizebox{\columnwidth}{!}{\input{./figs/quad/square.tex}}
	\end{center}
	\caption{Square}
	\label{fig:square}	
\end{figure}
\item With the same centre $\vec{O}$,  draw two circles of radii 4 and 2.5
\\
\solution
From the given information, the centre 

\begin{align}
    \vec{c} = \myvec{2\\2} \implies   \vec {u}= -\vec{c} = \myvec{-2\\-2 }
\end{align}

Using the general equation of a circle and substituting $\vec{x} = \myvec{4\\5}$,

\begin{align}
    \label{quadforms/july/2/2/eq:1}
    \begin{split}
\vec{x^\top}\vec{x}+2\vec{u^\top}\vec{x}+f&=0 
\\
\implies     \myvec{4 \ 5}\myvec{4\\5}+\myvec{-4 \ -4}\myvec{4\\5}+f&=0
\\
\implies f+\myvec{41}+\myvec{-36} &= 0 
\\
\text{or, }  f &= -5
    \end{split}
\end{align}
%
Hence , the equation of the circle is,
\begin{align}
    \vec{x^\top}\vec{x}+\myvec{-4 \ -4 }\vec{x}-5=0
\end{align}
%
The radius of the circle is then given by 
\begin{align}
r = \sqrt{\vec{u^\top}\vec{u}-f}
\implies r = \sqrt{13}
\end{align}
%
The above results are verified in Fig.     \ref{quadforms/july/2/2/Figure}


\begin{figure}[!h]
    \centering
    \includegraphics[width=\columnwidth]{solutions/july/2/2/Figures/Figure.png}
    \caption{Plot of the required circle}
    \label{quadforms/july/2/2/Figure}
\end{figure}

\item Let $\vec{A}$ and $\vec{B}$ be the centres of two circles of equal radii 3 such that each one of them passes through the centre of the other.  Let them intersect at $\vec{C}$ and $\vec{D}$.  Is $AB \perp CD$?
\\
\solution
The centers and radii of the two circles without any loss of generality are given in Table \ref{constr/55/tab:table1}
%
\begin{table}[!ht]
\begin{center}
\begin{tabular}{ | m{2cm} | m{2cm} | m{2cm} |} 
\hline
 & Circle 1 & Circle 2 \\
\hline
Centre  & $\vec{A}$=\myvec{0\\0} & $\vec{B}$=\myvec{3\\0} \\ 
\hline
Radius & $r_{1}=r_{2}=3$  \\ 
\hline
\end{tabular}
\end{center}
\caption{Input values}
\label{constr/55/tab:table1}

\end{table}

% The choice for $\vec{A}$ and $\vec{B}$ is valid as:
% \begin{align}
% \norm{\vec{B}-\vec{A}} = \norm{\vec{A}-\vec{B}}=\norm{\vec{B}}  = 3 \quad \brak{\because \vec{A}=0}
% \end{align}

Let 
\begin{align}
\vec{u}=\myvec{\cos \theta\\  \sin \theta},  \theta \in \sbrak{0,2\pi}.
\end{align}

Then on Circle 1  and Circle 2 are  given by 
\begin{align}
\vec{x}&=\vec{A}+r\vec{u}
\\
\vec{x}&=\vec{B}+r\vec{u}
\end{align}

Fig. \ref{constr/55/fig:circle} is plotted using the above equations.
%
\begin{figure}[!ht]
\centering
\includegraphics[width=\columnwidth]{solutions/55/Figures/figure3.png}
\caption{Circles with their points of intersection}
\label{constr/55/fig:circle}	
\end{figure}
Fig. \ref{constr/55/fig:circle} 

The general equation of Circle 1 is given by 
\begin{align}
    \norm{\vec{x}-\vec{A}}^2 &= r^2
    \\
\vec{x}^{\top}\vec{x} - 2\vec{A}^{\top}\vec{x} + \norm{\vec{A}}^2 -r_{1}^2 &= 0\label{constr/55/eq:14}
\end{align}
Similarly, for Circle 2,
\begin{align}
\vec{x}^{\top}\vec{x} - 2\vec{B}^{\top}\vec{x} + \norm{\vec{B}}^2 -r_{2}^2 = 0\label{constr/55/eq:15}
\end{align}
Subtracting \eqref{constr/55/eq:15} from \eqref{constr/55/eq:14},
\begin{align}
2\vec{B}^{\top}\vec{x}=\norm{\vec{B}}^2
\\
\myvec{1&0}\vec{x}=\frac{3}{2}
\end{align}
which can be expressed as
\begin{align}
\vec{x} &=\frac{1}{2}\myvec{3\\ 0} + \lambda \myvec{0\\1}\label{constr/55/eq:19}\\
&=\vec{q}+\lambda\vec{m} \text{ where}\label{constr/55/eq:20}\\
\vec{q}&=\myvec{1.5\\ 0},
\vec{m}=\myvec{0\\1}
\end{align}
Substituting \eqref{constr/55/eq:20} in \eqref{constr/55/eq:14}
\begin{align}
\norm{\vec{x}}^2=r^2\quad \brak{\because \vec{A}=0}
\\
\norm{\vec{q}+\lambda\vec{m}}^2=r^2\\
(\vec{q}+\lambda \vec{m})^{\top}(\vec{q}+\lambda \vec{m})=r^2\\
\implies \vec{q}^{\top}(\vec{q}+\lambda \vec{m})+\lambda \vec{m}^{\top}(\vec{q}+\lambda \vec{m})=r^2
\\
\implies \norm{\vec{q}}^2+\lambda\vec{q}^{\top}\vec{m}+\lambda\vec{m}^{\top}\vec{q}+\lambda^2\norm{\vec{m}}^2=r^2
\\
\implies \norm{\vec{q}}^2+2\lambda\vec{q}^{\top}\vec{m}+\lambda^2\norm{\vec{m}}^2=r^2 
% \\
% \implies \lambda(\lambda\norm{\vec{m}}^2+2\vec{q}^{\top}\vec{m})=r^2-\norm{\vec{q}}^2\\
% \implies\lambda^2\norm{\vec{m}}^2=9-\norm{\vec{q}}^2
\\
\implies \lambda=\pm \sqrt{\frac{9-\norm{\vec{q}}^2}{\norm{\vec{m}}^2}} \quad \because \vec{q}^{\top}\vec{m} = 0
% \lambda^2=6.75\\
% \lambda=+\sqrt{6.75},-\sqrt{6.75}
\end{align}
Substituting the value of $\lambda$ in \eqref{constr/55/eq:20},
\begin{align}
%\vec{x}=\vec{q}+\lambda\vec{m}\\
\vec{C}&=\vec{q}+\lambda\vec{m}\\
\vec{D}&=\vec{q}-\lambda\vec{m} \\
\implies (\vec{A}-\vec{B})^{\top}(\vec{C}-\vec{D})
&=2\myvec{-3&0}\myvec{0\\\sqrt{6.75}}
\\
&=0
\\
\implies AB\perp CD
\end{align}


% We have $\vec{C}$ and $\vec{D}$ as points of intersection and $r_{1}=r_{2}$.So,
% \begin{align}
% \norm{\vec{C}-\vec{A}}^2 = \norm{\vec{C}-\vec{B}}^2
% \end{align}
% \begin{align}
% \implies(\vec{C}-\vec{A})^{\top}(\vec{C}-\vec{A})=(\vec{C}-\vec{B})^{\top}(\vec{C}-\vec{B})
% \\
% \implies \vec{A}^{\top}\vec{C}-\vec{B}^{\top}\vec{C}=\vec{C}^{\top}\vec{B}-\vec{C}^{\top}\vec{A}+\norm{\vec{A}}^2-\norm{\vec{B}}^2 
% \end{align}
% \begin{align}
% \implies2\times \vec{A}^{\top}\vec{C}-2\times\vec{B}^{\top}\vec{C}=\norm{\vec{A}}^2-\norm{\vec{B}}^2 \label{constr/55/eq:1}
% \end{align}
% Similarly,using:
% \begin{align}
% \norm{\vec{D}-\vec{A}}^2 = \norm{\vec{D}-\vec{B}}^2
% \end{align}
% We get:
% \begin{align}
% 2\times \vec{A}^{\top}\vec{D}-2\times\vec{B}^{\top}\vec{D}=\norm{\vec{A}}^2-\norm{\vec{B}}^2 \label{constr/55/eq:2}
% \end{align}

% Subtracting equation \ref{constr/55/eq:2} from equation \ref{constr/55/eq:1}:
% \begin{align}
% 2\times(\vec{A}^{\top}-\vec{B}^{\top})(\vec{C}-\vec{D})=0
% \\
% \implies (\vec{A}-\vec{B})^{\top}(\vec{C}-\vec{D})=0
% \\
% \implies AB\perp CD
% \end{align}


\item Construct a tangent to a circle of radius 4 units from a point on the concentric circle of radius 6 
units.
\\
\solution 
The given information is summarised in Table \ref{constr/56/tab}.
\begin{table}[!ht]
\begin{center}
    \resizebox{\columnwidth}{!}{
\begin{tabular}{ | c | c| c| c |} 
\hline
& Symbols & Circle1 & Circle2 \\
\hline
Centre & $\vec{O}$ & \myvec{0\\0} & \myvec{0\\0} \\ 
\hline
Radius & $r_{1}$,$r_{2}$ & 4 & 6\\ 
\hline
\end{tabular}
}
\end{center}
\caption{}
\label{constr/56/tab}
\end{table}
See Fig. \ref{fig:constr/56/Tangent}. Let P be a point on Circle 2 with radius 6.  Then 
\begin{align}
\vec{P} = \myvec{6\\0}
\end{align}
Let $PQ$ and $PR$  be tangents from point $\vec{P}$ on circle with radius 6 to the points $\vec{Q}$ and $\vec{R}$ on circle with radius 4 .
% \begin{align}
% \because OQ \perp QP,
% \end{align}
Now,
\begin{align}
(\vec{O}-\vec{Q})^T (\vec{Q}-\vec{P}) &= 0 \quad \brak{\because OQ \perp QP }
% \\
% \vec{Q}^T(\vec{Q}-\vec{P}) &= 0 \quad \brak{\because \vec{O}=\myvec{0\\0}}
% \\
% \vec{Q}^T \vec{Q} - \vec{Q}^T \vec{P} &= 0  
% \\
% \norm{\vec{Q}}^2 &= \vec{Q}^T \vec{P}
% \\
% \norm{\vec{Q}}^2 &= \vec{P}^T \vec{Q}  \quad \brak{\because \vec{Q}^T \vec{P} = \vec{P}^T \vec{Q}}
\\
\implies \vec{P}^T \vec{Q} &= 16 \quad \brak{\because \norm{\vec{Q}}^2 = 16}
% \\
% \myvec{6&0} \vec{Q} &= 16 \quad \brak{\because \vec{P} = \myvec{6\\0} }
\\
\text{or, }\myvec{1&0} \vec{Q} &= \frac{8}{3}
\\
\implies \vec{Q} &= \myvec{\frac{8}{3}\\0} + \lambda \myvec{0\\1} \label{constr/56/2.0.11} 
\\
&=\vec{q}+\lambda\vec{m}
\\
\text{where }\vec{q}&=\myvec{\frac{8}{3}\\ 0},\vec{m}=\myvec{0\\1}
\end{align}
We know,
\begin{align}
\norm{\vec{q}+\lambda\vec{m}}^2&=r_1^2
\\
(\vec{q}+\lambda \vec{m})^T(\vec{q}+\lambda \vec{m})&=r_1^2
\\
\lambda^2&=\frac{r_1^2-\norm{\vec{q}}^2}{\norm{\vec{m}}^2}
\\
\lambda &= \pm 2.98
\end{align}

Substituting the above in \eqref{constr/56/2.0.11},
%
\begin{align}
\vec{Q}=\myvec{\frac{8}{3}\\2.98},
\vec{R}=\myvec{\frac{8}{3}\\-2.98}
\end{align}
The circels as well as the tangents are plotted in Fig.     \ref{fig:constr/56/Tangent}
%
\begin{figure}[ht]
    \centering
    \includegraphics[width=\columnwidth]{solutions/circle/56/TANGENT.png}
    \caption{Tangent lines to circle of radius 4 units.}
    \label{fig:constr/56/Tangent}
\end{figure}    




\item Draw a circle with centre $\vec{C}$ and radius 3.4.  Draw any chord.  Construct the perpendicular bisector of the chord and examine if it passes through $\vec{C}$.
\\
\solution
Data from the given question is available in Table \ref{constr/circ/61/tab:table1}:
%
\begin{table}[!ht]
\begin{center}
\begin{tabular}{ | m{2cm} | m{1.5cm}| m{2cm} | m{1.5cm} |} 
\hline
& Symbols & Circle1  \\
\hline
Centre & $\vec{C}$ & \myvec{0\\0}  \\ 
\hline
Radius & $r$& 3.4 \\ 
\hline
\end{tabular}
\end{center}
\caption{Input values}
\label{constr/circ/61/tab:table1}
\end{table}
%
Let 
\begin{align}
      \vec{P} &= r \myvec{\cos \theta_1\\ \sin \theta_1} =  \myvec{1.7\\2.9}
      \\
      \vec{Q} &= r \myvec{\cos \theta_2\\ \sin \theta_2} =\myvec{-2.4\\2.4}
 \end{align}
 Then the perpendicular bisector of $PQ$ passes through 
 \begin{align}
    \vec{M} = \frac{\vec{P}+\vec{Q}}{2} 
\end{align}
and has normal vector 
\begin{align}
    \vec{n} = \vec{P}-\vec{Q}
\end{align}
resulting in the equation
\begin{align}
    \brak{\vec{P}-\vec{Q}}^\top\brak{\vec{x}-\frac{\vec{P}+\vec{Q}}{2} } = 0
    \implies \brak{\vec{P}-\vec{Q}}^\top\vec{x} = 0
\end{align}
%
after simplification.  It is obvious that $\vec{O}$ satisfies the above equuation as can be verified in Fig.     \ref{constr/circ/61/fig: perpenicular bisector of the chord passes through the center}.
%
\begin{figure}[ht]
    \centering
    \includegraphics[width=\columnwidth]{solutions/circle/61/FIG.3.png}
    \caption{perpendicular bisector of the chord passes through the center}
    \label{constr/circ/61/fig: perpenicular bisector of the chord passes through the center}
\end{figure}







\item Construct a quadrilateral $ABCD$ such that $AB=5, \angle A = 50\degree, AC = 4, BD = 5$ and $AD = 6$.
\\
\solution 
\begin{align}
AB&=\myvec{(2 \times 1) +(4 \times (-2)) &(2 \times 3) +(4 \times 5)\\(3 \times 1) +(2 \times (-2)) &(3 \times 3) +(2 \times 5)}\\
&=\myvec{-6 &26\\-1 &19}
\end{align}

\item Can you construct a quadrilateral $PQRS$ with $PQ=3, RS=3, PS=7.5, PR=8$ and $SQ=4$?
\\
\solution 
%
If solution exists for the given
system of equations then they
said to be consistent, otherwise they are
inconsistent.
The above equations can be expressed as the matrix equation
\begin{align}
\myvec{1 & 2\\2 & 3} \vec{x} = \myvec{2\\3}
\end{align}
%
The augmented matrix for the above equation and row reducing as follows
\begin{align}
\myvec{1 & 2 & 2 \\ 2 & 3 & 3}  \xleftrightarrow[]{R_2\rightarrow R_2-2R_1} \myvec{1 & 2 & 2 \\ 0 & -1 & -1}\\
\xleftrightarrow[]{R_1\rightarrow R_1+2R_2}
\myvec{1 & 0 & 0\\0 & -1 & -1}\label{aug/2/70/a}\\
\implies\text{Rank}\myvec{1&2\\2&3}=\text{Rank}\myvec{1&2&2\\2&3&3}=2
\end{align}
Here, $Rank(A)=Rank(A|B)$. Therefore, the system is consistent. Also, there exist a unique solution as $Rank(A)=n$ (number of unknown).\\ 
From equation \eqref{aug/2/70/a}, we get:
\begin{align}
    \vec{x}=\myvec{0\\1}
\end{align}
Plotting the lines and the intersection point in Fig.\ref{aug/2/70/b}
\begin{figure}[htp]
\centering
\includegraphics[width=\columnwidth]{solutions/aug/2/70/a_2.png}
\caption{Lines and their intersection denoting the solution}
\label{aug/2/70/b}
\end{figure}
%
$\therefore$ The given system of equation is consistent with unique solution of,
$$\myvec{x\\y}=\myvec{0\\1}$$


\item Draw $GOLD$ such that $OL=7.5, GL=6, GD=6, LD = 5, OD = 10$.
\\
\solution 
Data from the given question is available in Table \ref{constr/circ/61/tab:table1}:
%
\begin{table}[!ht]
\begin{center}
\begin{tabular}{ | m{2cm} | m{1.5cm}| m{2cm} | m{1.5cm} |} 
\hline
& Symbols & Circle1  \\
\hline
Centre & $\vec{C}$ & \myvec{0\\0}  \\ 
\hline
Radius & $r$& 3.4 \\ 
\hline
\end{tabular}
\end{center}
\caption{Input values}
\label{constr/circ/61/tab:table1}
\end{table}
%
Let 
\begin{align}
      \vec{P} &= r \myvec{\cos \theta_1\\ \sin \theta_1} =  \myvec{1.7\\2.9}
      \\
      \vec{Q} &= r \myvec{\cos \theta_2\\ \sin \theta_2} =\myvec{-2.4\\2.4}
 \end{align}
 Then the perpendicular bisector of $PQ$ passes through 
 \begin{align}
    \vec{M} = \frac{\vec{P}+\vec{Q}}{2} 
\end{align}
and has normal vector 
\begin{align}
    \vec{n} = \vec{P}-\vec{Q}
\end{align}
resulting in the equation
\begin{align}
    \brak{\vec{P}-\vec{Q}}^\top\brak{\vec{x}-\frac{\vec{P}+\vec{Q}}{2} } = 0
    \implies \brak{\vec{P}-\vec{Q}}^\top\vec{x} = 0
\end{align}
%
after simplification.  It is obvious that $\vec{O}$ satisfies the above equuation as can be verified in Fig.     \ref{constr/circ/61/fig: perpenicular bisector of the chord passes through the center}.
%
\begin{figure}[ht]
    \centering
    \includegraphics[width=\columnwidth]{solutions/circle/61/FIG.3.png}
    \caption{perpendicular bisector of the chord passes through the center}
    \label{constr/circ/61/fig: perpenicular bisector of the chord passes through the center}
\end{figure}







\item Construct $ABCD $, where $AB = 4, BC = 5, Cd = 6.5, \angle B = 105 \degree$ and $\angle C = 80\degree$.
\\
\solution 

Let 
    \begin{align}
    &\angle B= 105\degree=\theta \label{quad/42/eq1}
    \\
    &\angle C= 80\degree=\alpha \label{quad/42/eq2}
    \\
    &\norm{\vec{A}-\vec{B}} =4=p \label{quad/42/eq3}
    \\
    &\norm{\vec{C}-\vec{B}} =5=q \label{quad/42/eq4}
    \\
     &\norm{\vec{D}-\vec{C}} =6.5=r \label{quad/42/eq5}
    \end{align}
and 
\begin{align}
    &\vec{B}=\myvec{0\\0}, \vec{C}=\myvec{5\\0}
\end{align}

\begin{lemma}
\label{quad/42/lemma}
\begin{align}
  & \vec{A} = p \vec{b}  \quad \brak{\because \vec{B}=\myvec{0\\0}} \label{quad/42/eq a}
\\
  & \vec{D} =\vec{C} + r \vec{c} \label{quad/42/eq b}
\end{align}
where 
\begin{align}
 \vec{b} = \myvec{\cos B \\ \sin B } ,\vec{c} = \myvec{\cos C\\ \sin C }
\end{align}
\end{lemma}
Thus, 
\begin{align}
 \vec{A} &=4\myvec{\cos 105 \\\sin 105 }
\\
&=\myvec{-1.03\\3.86}
\end{align}
and 
\begin{align}
\vec{D} &=\myvec{5\\0} + 6.5\myvec{\cos 80 \\\sin 80 }
\\
&=\myvec{6.12\\6.39}
\end{align}
which are then used to plot Fig. \ref{quad/42/fig:Quadrilateral ABCD}	
\begin{figure}[!ht]
\centering
\includegraphics[ width=\columnwidth]{solutions/quad/42/FIGURE.png}
\caption{Quadrilateral ABCD}
\label{quad/42/fig:Quadrilateral ABCD}	
\end{figure}

%
\item Construct $DEAR$ with $DE = 4, EA = 5, AR = 4.5, \angle E = 60 \degree$ and $\angle A = 90 \degree$.
\\
\solution 
Data from the given question is available in Table \ref{constr/circ/61/tab:table1}:
%
\begin{table}[!ht]
\begin{center}
\begin{tabular}{ | m{2cm} | m{1.5cm}| m{2cm} | m{1.5cm} |} 
\hline
& Symbols & Circle1  \\
\hline
Centre & $\vec{C}$ & \myvec{0\\0}  \\ 
\hline
Radius & $r$& 3.4 \\ 
\hline
\end{tabular}
\end{center}
\caption{Input values}
\label{constr/circ/61/tab:table1}
\end{table}
%
Let 
\begin{align}
      \vec{P} &= r \myvec{\cos \theta_1\\ \sin \theta_1} =  \myvec{1.7\\2.9}
      \\
      \vec{Q} &= r \myvec{\cos \theta_2\\ \sin \theta_2} =\myvec{-2.4\\2.4}
 \end{align}
 Then the perpendicular bisector of $PQ$ passes through 
 \begin{align}
    \vec{M} = \frac{\vec{P}+\vec{Q}}{2} 
\end{align}
and has normal vector 
\begin{align}
    \vec{n} = \vec{P}-\vec{Q}
\end{align}
resulting in the equation
\begin{align}
    \brak{\vec{P}-\vec{Q}}^\top\brak{\vec{x}-\frac{\vec{P}+\vec{Q}}{2} } = 0
    \implies \brak{\vec{P}-\vec{Q}}^\top\vec{x} = 0
\end{align}
%
after simplification.  It is obvious that $\vec{O}$ satisfies the above equuation as can be verified in Fig.     \ref{constr/circ/61/fig: perpenicular bisector of the chord passes through the center}.
%
\begin{figure}[ht]
    \centering
    \includegraphics[width=\columnwidth]{solutions/circle/61/FIG.3.png}
    \caption{perpendicular bisector of the chord passes through the center}
    \label{constr/circ/61/fig: perpenicular bisector of the chord passes through the center}
\end{figure}








\item Construct $TRUE$ with $TR = 3.5, RU = 3, UE = 4 \angle R = 75\degree$ and $\angle U = 120\degree$.
\\
\solution 
Data from the given question is available in Table \ref{constr/circ/61/tab:table1}:
%
\begin{table}[!ht]
\begin{center}
\begin{tabular}{ | m{2cm} | m{1.5cm}| m{2cm} | m{1.5cm} |} 
\hline
& Symbols & Circle1  \\
\hline
Centre & $\vec{C}$ & \myvec{0\\0}  \\ 
\hline
Radius & $r$& 3.4 \\ 
\hline
\end{tabular}
\end{center}
\caption{Input values}
\label{constr/circ/61/tab:table1}
\end{table}
%
Let 
\begin{align}
      \vec{P} &= r \myvec{\cos \theta_1\\ \sin \theta_1} =  \myvec{1.7\\2.9}
      \\
      \vec{Q} &= r \myvec{\cos \theta_2\\ \sin \theta_2} =\myvec{-2.4\\2.4}
 \end{align}
 Then the perpendicular bisector of $PQ$ passes through 
 \begin{align}
    \vec{M} = \frac{\vec{P}+\vec{Q}}{2} 
\end{align}
and has normal vector 
\begin{align}
    \vec{n} = \vec{P}-\vec{Q}
\end{align}
resulting in the equation
\begin{align}
    \brak{\vec{P}-\vec{Q}}^\top\brak{\vec{x}-\frac{\vec{P}+\vec{Q}}{2} } = 0
    \implies \brak{\vec{P}-\vec{Q}}^\top\vec{x} = 0
\end{align}
%
after simplification.  It is obvious that $\vec{O}$ satisfies the above equuation as can be verified in Fig.     \ref{constr/circ/61/fig: perpenicular bisector of the chord passes through the center}.
%
\begin{figure}[ht]
    \centering
    \includegraphics[width=\columnwidth]{solutions/circle/61/FIG.3.png}
    \caption{perpendicular bisector of the chord passes through the center}
    \label{constr/circ/61/fig: perpenicular bisector of the chord passes through the center}
\end{figure}










\item Can you construct a rhombus $ABCD$ with $AC = 6$ and $BD = 7$?
\\
\solution 
Data from the given question is available in Table \ref{constr/circ/61/tab:table1}:
%
\begin{table}[!ht]
\begin{center}
\begin{tabular}{ | m{2cm} | m{1.5cm}| m{2cm} | m{1.5cm} |} 
\hline
& Symbols & Circle1  \\
\hline
Centre & $\vec{C}$ & \myvec{0\\0}  \\ 
\hline
Radius & $r$& 3.4 \\ 
\hline
\end{tabular}
\end{center}
\caption{Input values}
\label{constr/circ/61/tab:table1}
\end{table}
%
Let 
\begin{align}
      \vec{P} &= r \myvec{\cos \theta_1\\ \sin \theta_1} =  \myvec{1.7\\2.9}
      \\
      \vec{Q} &= r \myvec{\cos \theta_2\\ \sin \theta_2} =\myvec{-2.4\\2.4}
 \end{align}
 Then the perpendicular bisector of $PQ$ passes through 
 \begin{align}
    \vec{M} = \frac{\vec{P}+\vec{Q}}{2} 
\end{align}
and has normal vector 
\begin{align}
    \vec{n} = \vec{P}-\vec{Q}
\end{align}
resulting in the equation
\begin{align}
    \brak{\vec{P}-\vec{Q}}^\top\brak{\vec{x}-\frac{\vec{P}+\vec{Q}}{2} } = 0
    \implies \brak{\vec{P}-\vec{Q}}^\top\vec{x} = 0
\end{align}
%
after simplification.  It is obvious that $\vec{O}$ satisfies the above equuation as can be verified in Fig.     \ref{constr/circ/61/fig: perpenicular bisector of the chord passes through the center}.
%
\begin{figure}[ht]
    \centering
    \includegraphics[width=\columnwidth]{solutions/circle/61/FIG.3.png}
    \caption{perpendicular bisector of the chord passes through the center}
    \label{constr/circ/61/fig: perpenicular bisector of the chord passes through the center}
\end{figure}








\item Draw a square $READ$ with $RE = 5.1$.
\\
\solution 
The centers and radii of the two circles without any loss of generality are given in Table \ref{constr/55/tab:table1}
%
\begin{table}[!ht]
\begin{center}
\begin{tabular}{ | m{2cm} | m{2cm} | m{2cm} |} 
\hline
 & Circle 1 & Circle 2 \\
\hline
Centre  & $\vec{A}$=\myvec{0\\0} & $\vec{B}$=\myvec{3\\0} \\ 
\hline
Radius & $r_{1}=r_{2}=3$  \\ 
\hline
\end{tabular}
\end{center}
\caption{Input values}
\label{constr/55/tab:table1}

\end{table}

% The choice for $\vec{A}$ and $\vec{B}$ is valid as:
% \begin{align}
% \norm{\vec{B}-\vec{A}} = \norm{\vec{A}-\vec{B}}=\norm{\vec{B}}  = 3 \quad \brak{\because \vec{A}=0}
% \end{align}

Let 
\begin{align}
\vec{u}=\myvec{\cos \theta\\  \sin \theta},  \theta \in \sbrak{0,2\pi}.
\end{align}

Then on Circle 1  and Circle 2 are  given by 
\begin{align}
\vec{x}&=\vec{A}+r\vec{u}
\\
\vec{x}&=\vec{B}+r\vec{u}
\end{align}

Fig. \ref{constr/55/fig:circle} is plotted using the above equations.
%
\begin{figure}[!ht]
\centering
\includegraphics[width=\columnwidth]{solutions/55/Figures/figure3.png}
\caption{Circles with their points of intersection}
\label{constr/55/fig:circle}	
\end{figure}
Fig. \ref{constr/55/fig:circle} 

The general equation of Circle 1 is given by 
\begin{align}
    \norm{\vec{x}-\vec{A}}^2 &= r^2
    \\
\vec{x}^{\top}\vec{x} - 2\vec{A}^{\top}\vec{x} + \norm{\vec{A}}^2 -r_{1}^2 &= 0\label{constr/55/eq:14}
\end{align}
Similarly, for Circle 2,
\begin{align}
\vec{x}^{\top}\vec{x} - 2\vec{B}^{\top}\vec{x} + \norm{\vec{B}}^2 -r_{2}^2 = 0\label{constr/55/eq:15}
\end{align}
Subtracting \eqref{constr/55/eq:15} from \eqref{constr/55/eq:14},
\begin{align}
2\vec{B}^{\top}\vec{x}=\norm{\vec{B}}^2
\\
\myvec{1&0}\vec{x}=\frac{3}{2}
\end{align}
which can be expressed as
\begin{align}
\vec{x} &=\frac{1}{2}\myvec{3\\ 0} + \lambda \myvec{0\\1}\label{constr/55/eq:19}\\
&=\vec{q}+\lambda\vec{m} \text{ where}\label{constr/55/eq:20}\\
\vec{q}&=\myvec{1.5\\ 0},
\vec{m}=\myvec{0\\1}
\end{align}
Substituting \eqref{constr/55/eq:20} in \eqref{constr/55/eq:14}
\begin{align}
\norm{\vec{x}}^2=r^2\quad \brak{\because \vec{A}=0}
\\
\norm{\vec{q}+\lambda\vec{m}}^2=r^2\\
(\vec{q}+\lambda \vec{m})^{\top}(\vec{q}+\lambda \vec{m})=r^2\\
\implies \vec{q}^{\top}(\vec{q}+\lambda \vec{m})+\lambda \vec{m}^{\top}(\vec{q}+\lambda \vec{m})=r^2
\\
\implies \norm{\vec{q}}^2+\lambda\vec{q}^{\top}\vec{m}+\lambda\vec{m}^{\top}\vec{q}+\lambda^2\norm{\vec{m}}^2=r^2
\\
\implies \norm{\vec{q}}^2+2\lambda\vec{q}^{\top}\vec{m}+\lambda^2\norm{\vec{m}}^2=r^2 
% \\
% \implies \lambda(\lambda\norm{\vec{m}}^2+2\vec{q}^{\top}\vec{m})=r^2-\norm{\vec{q}}^2\\
% \implies\lambda^2\norm{\vec{m}}^2=9-\norm{\vec{q}}^2
\\
\implies \lambda=\pm \sqrt{\frac{9-\norm{\vec{q}}^2}{\norm{\vec{m}}^2}} \quad \because \vec{q}^{\top}\vec{m} = 0
% \lambda^2=6.75\\
% \lambda=+\sqrt{6.75},-\sqrt{6.75}
\end{align}
Substituting the value of $\lambda$ in \eqref{constr/55/eq:20},
\begin{align}
%\vec{x}=\vec{q}+\lambda\vec{m}\\
\vec{C}&=\vec{q}+\lambda\vec{m}\\
\vec{D}&=\vec{q}-\lambda\vec{m} \\
\implies (\vec{A}-\vec{B})^{\top}(\vec{C}-\vec{D})
&=2\myvec{-3&0}\myvec{0\\\sqrt{6.75}}
\\
&=0
\\
\implies AB\perp CD
\end{align}


% We have $\vec{C}$ and $\vec{D}$ as points of intersection and $r_{1}=r_{2}$.So,
% \begin{align}
% \norm{\vec{C}-\vec{A}}^2 = \norm{\vec{C}-\vec{B}}^2
% \end{align}
% \begin{align}
% \implies(\vec{C}-\vec{A})^{\top}(\vec{C}-\vec{A})=(\vec{C}-\vec{B})^{\top}(\vec{C}-\vec{B})
% \\
% \implies \vec{A}^{\top}\vec{C}-\vec{B}^{\top}\vec{C}=\vec{C}^{\top}\vec{B}-\vec{C}^{\top}\vec{A}+\norm{\vec{A}}^2-\norm{\vec{B}}^2 
% \end{align}
% \begin{align}
% \implies2\times \vec{A}^{\top}\vec{C}-2\times\vec{B}^{\top}\vec{C}=\norm{\vec{A}}^2-\norm{\vec{B}}^2 \label{constr/55/eq:1}
% \end{align}
% Similarly,using:
% \begin{align}
% \norm{\vec{D}-\vec{A}}^2 = \norm{\vec{D}-\vec{B}}^2
% \end{align}
% We get:
% \begin{align}
% 2\times \vec{A}^{\top}\vec{D}-2\times\vec{B}^{\top}\vec{D}=\norm{\vec{A}}^2-\norm{\vec{B}}^2 \label{constr/55/eq:2}
% \end{align}

% Subtracting equation \ref{constr/55/eq:2} from equation \ref{constr/55/eq:1}:
% \begin{align}
% 2\times(\vec{A}^{\top}-\vec{B}^{\top})(\vec{C}-\vec{D})=0
% \\
% \implies (\vec{A}-\vec{B})^{\top}(\vec{C}-\vec{D})=0
% \\
% \implies AB\perp CD
% \end{align}

\item Draw a rhombus who diagonals are $5.2$ and $6.4$.
\\
\solution 
Data from the given question is available in Table \ref{constr/circ/61/tab:table1}:
%
\begin{table}[!ht]
\begin{center}
\begin{tabular}{ | m{2cm} | m{1.5cm}| m{2cm} | m{1.5cm} |} 
\hline
& Symbols & Circle1  \\
\hline
Centre & $\vec{C}$ & \myvec{0\\0}  \\ 
\hline
Radius & $r$& 3.4 \\ 
\hline
\end{tabular}
\end{center}
\caption{Input values}
\label{constr/circ/61/tab:table1}
\end{table}
%
Let 
\begin{align}
      \vec{P} &= r \myvec{\cos \theta_1\\ \sin \theta_1} =  \myvec{1.7\\2.9}
      \\
      \vec{Q} &= r \myvec{\cos \theta_2\\ \sin \theta_2} =\myvec{-2.4\\2.4}
 \end{align}
 Then the perpendicular bisector of $PQ$ passes through 
 \begin{align}
    \vec{M} = \frac{\vec{P}+\vec{Q}}{2} 
\end{align}
and has normal vector 
\begin{align}
    \vec{n} = \vec{P}-\vec{Q}
\end{align}
resulting in the equation
\begin{align}
    \brak{\vec{P}-\vec{Q}}^\top\brak{\vec{x}-\frac{\vec{P}+\vec{Q}}{2} } = 0
    \implies \brak{\vec{P}-\vec{Q}}^\top\vec{x} = 0
\end{align}
%
after simplification.  It is obvious that $\vec{O}$ satisfies the above equuation as can be verified in Fig.     \ref{constr/circ/61/fig: perpenicular bisector of the chord passes through the center}.
%
\begin{figure}[ht]
    \centering
    \includegraphics[width=\columnwidth]{solutions/circle/61/FIG.3.png}
    \caption{perpendicular bisector of the chord passes through the center}
    \label{constr/circ/61/fig: perpenicular bisector of the chord passes through the center}
\end{figure}








\item Draw a rectangle with adjacent sides $5$ and $4$.
\\
\solution 
Data from the given question is available in Table \ref{constr/circ/61/tab:table1}:
%
\begin{table}[!ht]
\begin{center}
\begin{tabular}{ | m{2cm} | m{1.5cm}| m{2cm} | m{1.5cm} |} 
\hline
& Symbols & Circle1  \\
\hline
Centre & $\vec{C}$ & \myvec{0\\0}  \\ 
\hline
Radius & $r$& 3.4 \\ 
\hline
\end{tabular}
\end{center}
\caption{Input values}
\label{constr/circ/61/tab:table1}
\end{table}
%
Let 
\begin{align}
      \vec{P} &= r \myvec{\cos \theta_1\\ \sin \theta_1} =  \myvec{1.7\\2.9}
      \\
      \vec{Q} &= r \myvec{\cos \theta_2\\ \sin \theta_2} =\myvec{-2.4\\2.4}
 \end{align}
 Then the perpendicular bisector of $PQ$ passes through 
 \begin{align}
    \vec{M} = \frac{\vec{P}+\vec{Q}}{2} 
\end{align}
and has normal vector 
\begin{align}
    \vec{n} = \vec{P}-\vec{Q}
\end{align}
resulting in the equation
\begin{align}
    \brak{\vec{P}-\vec{Q}}^\top\brak{\vec{x}-\frac{\vec{P}+\vec{Q}}{2} } = 0
    \implies \brak{\vec{P}-\vec{Q}}^\top\vec{x} = 0
\end{align}
%
after simplification.  It is obvious that $\vec{O}$ satisfies the above equuation as can be verified in Fig.     \ref{constr/circ/61/fig: perpenicular bisector of the chord passes through the center}.
%
\begin{figure}[ht]
    \centering
    \includegraphics[width=\columnwidth]{solutions/circle/61/FIG.3.png}
    \caption{perpendicular bisector of the chord passes through the center}
    \label{constr/circ/61/fig: perpenicular bisector of the chord passes through the center}
\end{figure}







\item Draw a parallelogram $OKAY$ with $OK = 5.5$ and $KA = 4.2$.
\\
\solution  There are infinite number of parallelograms that can be draw.  For a unique parallelogram, one angle
needs to be specified.

%
%\end{enumerate}
