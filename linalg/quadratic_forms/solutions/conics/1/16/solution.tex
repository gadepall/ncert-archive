
General equation of conics is 
\begin{align}
    \vec{x}^T\vec{V}\vec{x}+ 2\vec{u}^T\vec{x}+f = 0
    \label{eq:solutions/1/16/eq:1}
\end{align}
Comparing with the equation given,
\begin{align}
\vec{V}=\myvec{\frac{1}{9} & 0 \\ 0 & \frac{1}{16}}\\
\vec{u}=\vec{0}\\
f=-1\\
\mydet{\vec{v}}=\mydet{\myvec{\frac{1}{9} & 0 \\ 0 & \frac{1}{16}}}>0
\end{align}
$\because \abs{\vec{V}}>0$, the given equation is of ellipse.\\
a)The tangents are parallel to the x-axis, hence, their direction and normal vectors, $\vec{m_1}$ and $\vec{n_1}$ are respectively,
\begin{align}
\vec{m_1}=\myvec{1\\0}\\
\vec{n_1}=\myvec{0\\1}
\end{align}
For an ellipse, given the normal vector $\vec{n}$, the tangent points of contact to the ellipse are given by
\begin{align}
    \vec{q}=\vec{V}^{-1}(\kappa \vec{n}-\vec{u})
    \label{eq:solutions/1/16/eq:2}
    =\vec{V}^{-1}\kappa \vec{n}
\end{align}
where
\begin{align}
    \kappa=\pm \sqrt{\frac{\vec{u^T}\vec{V}^{-1}\vec{u}-f}{\vec{n^T}\vec{V}^{-1}\vec{n}}}
    \label{eq:solutions/1/16/eq:2.0.9}\\
   =\pm \sqrt{\frac{-f}{\vec{n^T}\vec{V}^{-1}\vec{n}}}\\
    \vec{V}^{-1}=\myvec{9 & 0 \\ 0 & 16}\\
    \kappa_1=\pm \sqrt{\frac{-(-1)}{\myvec{0 & 1}\myvec{9 & 0 \\ 0 & 16} \myvec{0\\1}}}\\
 \implies \kappa_1=\pm \sqrt{\frac{1}{16}}\\
    \implies \kappa_1=\pm \frac{1}{4}      
\end{align}
From \eqref{eq:solutions/1/16/eq:2} , the point of contact $\vec{q_i}$ are,
\begin{align}
    \vec{q_1}=\myvec{9 & 0 \\ 0 & 16}\frac{1}{4}\myvec{0\\1}\\
    =\myvec{9 & 0 \\ 0 & 16}\myvec{0\\\frac{1}{4}}\\
    =\myvec{0\\4}\\
    \vec{q_2}=\myvec{9 & 0 \\ 0 & 16}\left(-\frac{1}{4}\right)\ \myvec{0\\1}\\
    =\myvec{9 & 0 \\ 0 & 16}\myvec{0\\-\frac{1}{4}}\\
    =\myvec{0\\-4}
\end{align}
b) The tangents are parallel to the y-axis, hence, their direction and normal vectors, $\vec{m_2}$ and $\vec{n_2}$ are respectively,
\begin{align}
\vec{m_2}=\myvec{0\\1}\\
\vec{n_2}=\myvec{1\\0}
\end{align}
Using equation \eqref{eq:solutions/1/16/eq:2.0.9}, the values of $\kappa$ for this case are
\begin{align}
     \kappa_2=\pm \sqrt{\frac{-(-1)}{\myvec{1 & 0}\myvec{9 & 0 \\ 0 & 16} \myvec{1\\0}}}\\
 \implies \kappa_2=\pm \sqrt{\frac{1}{9}}\\
    \implies \kappa_2=\pm \frac{1}{3} 
\end{align}
and from \eqref{eq:solutions/1/16/eq:2} , the point of contact $\vec{q_i}$ are,
\begin{align}
\vec{q_3}=\myvec{9 & 0 \\ 0 & 16}\frac{1}{3}\myvec{1\\0}\\
    =\myvec{9 & 0 \\ 0 & 16}\myvec{\frac{1}{3}\\0}\\
    =\myvec{3\\0}\\
\vec{q_4}=\myvec{9 & 0 \\ 0 & 16}\left(-\frac{1}{3}\right)\ \myvec{1\\0}\\
    =\myvec{9 & 0 \\ 0 & 16}\myvec{-\frac{1}{3}\\0}\\
    =\myvec{-3\\0}
\end{align}
 \begin{figure}[h!]
	\centering
	\includegraphics[width=\columnwidth]{./solutions/conics/1/16/ellipse.png}
	\caption{Figure depicting point of contact of tangents of ellipse parallel to x-axis and y-axis}
	\label{eq:solutions/1/16/fig1}
\end{figure}
