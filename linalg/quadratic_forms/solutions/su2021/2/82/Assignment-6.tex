The given equation can be expressed as 
\begin{align}
    \vec{x}^{\top}\myvec{0&0 \\ 0&1}\vec{x}+\myvec{-3&0}\vec{x}+2=0 \label{quadform/82/eq:giveneq}
\end{align}
and 
\begin{align}
\therefore \vec{u}=\myvec{\frac{-3}{2} \\ 0}
\\
\vec{V} =\myvec{0&0 \\ 0&1}
\\
\implies {\mydet{\vec{V}}}=0
\end{align}
Thus the curve is a parabola with eigenvalues 
\begin{align}
 \lambda= 0,1
\end{align}
The corresponding  eigenvectors are 
\begin{align}
    \myvec{0&0 \\ 0&1}\vec{x} = 0\implies \vec{p_{1}}= \myvec{1 \\ 0}
    \\
    \myvec{-1&0 \\ 0&0}\vec{x} = \vec{x}\implies \vec{p_{2}}= \myvec{0 \\ 1}
\end{align}
and 
\begin{align}
    \vec{V} = \vec{P}\vec{D}\vec{P}^{\top}\label{quadform/82/eq:eqn1}
\end{align}
where 
\begin{align}
 \vec{P} = \myvec{\vec{p_{1}} & \vec{p_{2}}} = \myvec{1&0 \\ 0&1}
 \\
 \vec{D}=\myvec{\lambda_{1}&0 \\ 0&\lambda_{2}} = \myvec{0&0 \\ 0&1}
\end{align}
Since the given parallel line equation  is 
\begin{align}
 \myvec{4&-2}\vec{x} + 5 = 0, \label{quadform/82/eq:eqn2}
\end{align}
%Now the tangent to the parabola is parallel to the line equation \eqref{quadform/82/eq:eqn2},
the normal vector $\vec{n}$ and the direction vector $\vec{m}$ of the tangent to the parabola are given by 
\begin{align}
    \vec{n}=\myvec{4 \\ -2} \label{quadform/82/eq:eqn5}
    \\
    \vec{m}=\myvec{-2 \\ -4}
\end{align}
The  point of contact for the parabola is given by,
\begin{align}
    \myvec{\vec{u}+\kappa \vec{n}^{\top} \\ \vec{V} }\vec{q} 
    = \myvec{\vec{-f}\\\kappa \vec{n} -\vec{u}}
    \\
  \implies \kappa =\frac{\vec{p_{1}}^{\top}\vec{u}}{\vec{p_{1}}^{\top}\vec{n}}
    =\frac{\myvec{1&0}\myvec{\frac{-3}{2} \\ 0}}{\myvec{1&0}\myvec{4 \\ 2}}
    =\frac{-3}{8}
    \end{align}
and 
\begin{align}
    \myvec{-3&\frac{3}{4} \\ 0&0 \\ 0&1} \vec{q}&= \myvec{-2 \\ 0 \\ \frac{3}{4}}
\end{align}
% Now solving for $\vec{q}$ by removing the zero row and representing as augmented matrix and then converting it to echelon form.
% \begin{align}
%     \myvec{-3&\frac{3}{4}&-2 \\ 0&1&\frac{3}{4}}\xleftrightarrow{R_1\rightarrow \brak{\frac{-1}{3}}R_1} \myvec{1&\frac{-1}{4}&\frac{2}{3} \\ 0&1&\frac{3}{4}}
%     \\
%     \myvec{1&\frac{-1}{4}&\frac{2}{3} \\ 0&1&\frac{3}{4}}\xleftrightarrow{R_1\rightarrow R_1+\frac{1}{4}R_2}\myvec{1&0&\frac{41}{48} \\ 0&1&\frac{3}{4}} \label{quadform/82/eq:eqn3}
% \end{align}
%Hence from the \eqref{quadform/82/eq:eqn3} we get the point of contact to be 
yielding 
\begin{align}
    \vec{q}=\myvec{\frac{41}{48} \\ \frac{3}{4}}
\end{align}
The desired equation of the line is 
%Now $\vec{q}$ is the point on the tangent.Hence the equation of the line can be expressed as 
\begin{align}
    \mathbf{n}^{\top}(\mathbf{x}-\mathbf{q}) &= 0\label{quadform/82/eq:eqn6}
    \\
% \end{align}
% where,
% \begin{align}
% \vec{n}^{\top}\vec{q}
%     =\myvec{4&-2}\myvec{\frac{41}{48} \\ \frac{3}{4}} 
%     \\
%     = \frac{23}{12} \label{quadform/82/eq:eqn4}
% \end{align}
% Hence the equation of the tangent to the curve \eqref{quadform/82/eq:giveneq} parallel to the line \eqref{quadform/82/eq:eqn2} is given by substituting the value of $\vec{n}^{\top}\vec{q}$ and $\vec{n}$ from \eqref{quadform/82/eq:eqn4} and \eqref{quadform/82/eq:eqn5} respectively to the equation \eqref{quadform/82/eq:eqn6} 
% \begin{align}
\implies     \myvec{4&-2}\vec{x}&=\frac{23}{12} \label{quadform/82/eq:eqn7}
\end{align}
which is verified in Fig. \ref{quadform/82/Plot of curve and the lines}.
% Hence \eqref{quadform/82/eq:eqn7} is the equation of the tangent.
% The plot of the figure is given below
%
\begin{figure}[ht]
\centering
\includegraphics[width=\columnwidth]{solutions/su2021/2/82/Parabola.PNG}
\caption{Plot of curve and the lines}
\label{quadform/82/Plot of curve and the lines}
\end{figure}