\begin{lemma}
The points of intersection of \textbf{Line} $L:\vec{x}=\vec{q}+\mu\vec{m}$ with \textbf{parabola},are given by:
\begin{align}
\vec{x}_i = \vec{q}+\mu_i\vec{m}
\end{align}
%
where,
\begin{multline}
\mu_i = \frac{1}
{
\vec{m}^T\vec{V}\vec{m}
}
\lbrak{-\vec{m}^T\brak{\vec{V}\vec{q}+\vec{u}}}
\\
\pm
{\small
\rbrak{\sqrt{
\sbrak{
\vec{m}^T\brak{\vec{V}\vec{q}+\vec{u}}
}^2
-
\brak
{
\vec{q}^T\vec{V}\vec{q} + 2\vec{u}^T\vec{q} +f
}
\brak{\vec{m}^T\vec{V}\vec{m}}
}
}
}\label{quadform/2/105/eq:mu_i}
\end{multline}
\end{lemma}
 The matrix parameters of the parabola are
\begin{align}
\vec{V}=\myvec{1&0\\0&0},\vec{u}=\myvec{0\\-\frac{2}{3}},f=0 \label{quadform/2/105/eq:v}
\end{align}
with eigen parameters 
\begin{align}
\lambda_1 &= 0 , \lambda_2 = 1
\\
\vec{p_1} &= \myvec{0 \\ 1},
\vec{p_2} = \myvec{1 \\ 0}
\end{align}
The vertex of the parabola can be expressed as
\begin{align} \myvec{\vec{u^T}+\eta\vec{p_1^T}\\\vec{V} }\vec{c} &= \myvec{\vec{-f}\\\eta \vec{p_1} -\vec{u}}
\\
\text{where, }  \eta = \vec{u^T}\vec{p_1} = \frac{-2}{3}
\\
\implies\myvec{0 & \frac{-4}{3}\\1 & 0 \\0 & 0} \vec{c}&= \myvec{0\\0\\0}
\\
\text{or, }
   \vec{c} &= \myvec{0 \\ 0}
\end{align}
From  \eqref{quadform/2/105/eq:mu_i},
\begin{align}
\mu_i &= \frac{1}{4}
\brak{10}
\pm{6}
\\
\implies \mu_1 &= 1, \mu_2 =4
\end{align}
The given line is
\begin{align} 
\myvec{-3&2}\vec{x}&=12
\end{align}
 In parametric form, the given  line can be written as:
\begin{align} 
L: \vec{x}&=\vec{q}+\mu\vec{m}
\\
\implies \vec{x}&=\myvec{-4\\0}+\mu\myvec{2\\3} \label{quadform/2/105/eq:q}
\end{align}
Substitutuing $\mu_1$ and $\mu_2$ in \eqref{quadform/2/105/eq:q},the points of intersection
\begin{align}
 \vec{K}= \myvec{-2\\3},  
\vec{L}&= \myvec{4\\12}
\end{align}

\begin{enumerate}
  \item Thus, from Fig. \ref{quadform/2/105/Plot of the Parabola and line} the area enclosed by parabola and line can be given as
\begin{align}
   A&= \text{Area under line} - \text{Area under parabola}
     \\
   A&= Ar(KLMNK)-Ar(KCLMCNK)
    \\
    A&= A_1 -A_2 \label{quadform/2/105/eqAREA}
    \end{align}
\item Area under the line 2y=3x+12 i.e, $A_1$-
\begin{align}
  A_1&= \int_{-2}^{4} y dx
    \\
    A_1&= \frac{1}{2}\int_{-2}^{4} \brak{3x+12} dx
    \end{align}
    \begin{align}
    A_1= \frac{3}{2}\int_{-2}^{4} x+\frac{1}{2}\int_{-2}^{4}12 dx
    \end{align}
    \begin{align}
    A_1&= \frac{3}{4}\brak{4^2-2^2} +\frac{12}{2}\brak{4+2}
   \\
    A_1&= \frac{3}{4}\brak{12} +\frac{12}{2}\brak{6}
    \\
    A_1&= 9 +36
    \\
    A_1 &=45 \text{ units} \label{quadform/2/105/eqB}
\end{align}
\item Area under the parabola that is $A_2$-
\begin{align}
    A_2&= \int_{-2}^{4} y dx
    \\
    A_2&= \int_{-2}^{4} \frac{3}{4} x^2 dx
    \\
    A_2&= \frac{3}{4} \int_{-2}^{4}  x^2 dx
    \\
    A_2&= \frac{3}{4\times3} \brak{4^3-\brak{-2}^3}
    \\
    A_2&= \frac{1}{4} \brak{64 +8}
    \\
    A_2&= \frac{72}{4} 
    \\
    A_2&= 18 \text{ units} \label{quadform/2/105/eqC}
\end{align}
\item Putting \eqref{quadform/2/105/eqB} and \eqref{quadform/2/105/eqC} in \eqref{quadform/2/105/eqAREA} we get required area A as:
\begin{align}
 A &= A_1 -A_2 
 \\
 A &= 45-18
 \\
 A &= 27 \text{ units}
\end{align}
%
\end{enumerate}
\begin{figure}[ht]
\centering
\includegraphics[width=\columnwidth]{solutions/su2021/2/105/LINE AND PARABOLA.png}
\caption{Plot of the parabola and line}
\label{quadform/2/105/Plot of the Parabola and line}
\end{figure}
