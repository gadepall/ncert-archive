The given curve can be expressed as 
\begin{align}
 x^2-4y &=0 \label{quadform/45/2.0.2}
 \\
%\vec{x}^T\vec{V}\vec{x}+2\vec{u}^T\vec{x}+f=0\\
\implies \vec{V}=\myvec{1 & 0 \\ 0 & 0},\vec{u}=\myvec{0 \\ -2 }, \vec{f} = 0 
\end{align}
$\because$
\begin{align}
 \abs{\vec{V}} &= 0
\end{align}
the given curve \eqref{quadform/45/2.0.2} represents a parabola.
The eigenvalues are given by 
\begin{align}
\lambda_1 = 0 , \lambda_2 = 1
\end{align}
with corresponding eigenvectors
\begin{align}
\myvec{1 & 0 \\ 0 & 0}\vec{x} &=0 \implies \vec{p_1} = \myvec{0 \\ 1}
\\
\myvec{0 & 0 \\ 0 & -1}\vec{x} &=0 \implies \vec{p_2} = \myvec{1 \\ 0}
\end{align}
To find the vertex of the parabola ,
\begin{align} \myvec{\vec{u^T}+\kappa\vec{p_1^T}\\\vec{V} }\vec{c} &= \myvec{\vec{-f}\\\kappa \vec{p_1} -\vec{u}}
\\
\text{where, }  \kappa = \vec{u^T}\vec{p_1} = -2
\\
\implies\myvec{0 & -4\\1 & 0 \\0 & 0} \vec{c}&= \myvec{0\\0\\0} \label{quadform/45/eq:b}
\end{align}
from the above it can be observed that,
\begin{align}    
   \vec{c} &= \myvec{0 \\ 0}
\end{align}
Now to evalute the direction vector m,
\begin{align}
\vec{m}^T(\vec{V}\vec{q} + \vec{u}) &=0
\\
\implies \vec{m}^T\brak{\myvec{1 & 0 \\ 0 & 0}\myvec{1 \\ 2} + \myvec{0 \\ -2}} &=0
\\
\implies \vec{m}^T\myvec{1 \\ -2} &=0
\\
\implies \vec{m} = \myvec{-2 \\ -1}
\end{align}
The normal is obtained as 
\begin{align}
\vec{m}^T(\vec{x} - \vec{q}) &=0 
\\
\myvec{-2 & -1}\brak{\vec{x}-\myvec{1 \\ 2}} &= 0
\\
\myvec{-2 & -1}\vec{x}+ 4 &= 0 
\end{align}
%
The above results are verified in Fig.     \ref{quadform/45/fig:Normal to parabola.}.
\begin{figure}[ht]
    \centering
    \includegraphics[width=\columnwidth]{solutions/su2021/2/45/FIGURE4.png}
    \caption{Normal to Parabola.}
    \label{quadform/45/fig:Normal to parabola.}
\end{figure}    

