
Given curve,
\begin{align}
\vec{x}^T\myvec{\frac{1}{9} & 0 \\ 0 & \frac{1}{16}}\vec{x} = 1 \label{quad/79/2.0.1}
\end{align}
where,
\begin{align}
\vec{V}&=\myvec{\frac{1}{9} & 0 \\ 0 & \frac{1}{16}},\vec{V}^{-1}=\myvec{9&0\\0&16}\vec{u}=0, f = -1 
\end{align}
\begin{align}
\because |\vec{V}| > 0
\end{align}
given curve \eqref{quad/79/2.0.1} is ellipse.
For an ellipse, the point of contact for the tangent is
\begin{align}
\vec{q}&=\vec{V}^{-1}(\kappa\vec{n}-\vec{u})\\
&=\vec{V}^{-1}\kappa\vec{n}\quad\quad\quad(\because \vec{u}=0). \label{quad/79/2.0.5}
\end{align}
where,
\begin{align}
\kappa=&\pm \sqrt{\frac{\vec{u^T}\vec{V}^{-1}\vec{u}-f}{\vec{n^T}\vec{V}^{-1}\vec{n}}}\\
      =&\pm \sqrt{\frac{-f}{\vec{n^T}\vec{V}^{-1}\vec{n}}}\label{quad/79/2.0.7}\quad \quad\quad(\because \vec{u}=0)
\end{align}
\begin{enumerate}
\item For the tangents  parallel to x-axis, then direction and normal vectors are,
\begin{align}
\vec{m_1} = \myvec{1\\0},\vec{n_1} = \myvec{0\\1},
\kappa_1 &=\pm \sqrt{\frac{-f}{\vec{n_1^T}\vec{V}^{-1}\vec{n_1}}} \\
 &=\pm\frac{1}{4}
\end{align}
$\therefore$ By Substituting $\kappa_1,\vec{n_1},\vec{V}^{-1}$ in \eqref{quad/79/2.0.5}
\begin{align}
\vec{q}&=\vec{V}^{-1}\kappa_1\vec{n_1} \\
&=\myvec{0\\\pm4}
\end{align}
% $\therefore$ Point of contact for tangents of ellipse are,
% \begin{align}
%   \vec{q_1}=\myvec{0\\4},\vec{q_2}=\myvec{0\\-4}
% \end{align}
\item For the tangents  parallel to the y-axis, the direction and normal vectors are
\begin{align}
\vec{m_2} = \myvec{0\\1},\vec{n_2} = \myvec{1\\0},
\kappa_2 &=\pm \sqrt{\frac{-f}{\vec{n_2^T}\vec{V}^{-1}\vec{n_2}}} \\
 &=\pm\frac{1}{3}
\end{align}
$\therefore$ substituting $\kappa_2,\vec{n_2},\vec{V}^{-1}$ in \eqref{quad/79/2.0.5}
\begin{align}
\vec{q}&=\vec{V}^{-1}\kappa_2\vec{n_2} \\
&=\myvec{0\\\pm3}
\end{align}
% $\therefore$ Point of contact for tangents of ellipse are,
% \begin{align}
%   \vec{q_3}=\myvec{3\\0},\vec{q_4}=\myvec{-3\\0}
% \end{align}
\end{enumerate}
The above results are verified in Fig.     \ref{quad/79/fig:Tangent to ELLIPSE.}
%
\begin{figure}[!ht]
    \centering
    \includegraphics[width=\columnwidth]{solutions/su2021/2/79/ELLIPSE.png}
    \caption{Tangents to ELLIPSE.}
    \label{quad/79/fig:Tangent to ELLIPSE.}
\end{figure}  

