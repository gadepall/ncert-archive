\renewcommand{\theequation}{\theenumi}
\begin{enumerate}[label=\arabic*.,ref=\thesubsection.\theenumi]
\numberwithin{equation}{enumi}

\item Draw $\triangle ABC$ where $\angle B = 90\degree, a = 4$ and $b = 3$.
\\
\solution The vertices of $\triangle ABC$ are 
\begin{align}
\vec{A} = \myvec{0\\3}, \vec{B} = \myvec{0\\0}, \vec{C} = \myvec{4\\0}
\end{align}
%
The following code plots Fig. \ref{fig:rt_triangle}
\begin{lstlisting}
codes/constructions/rt_triangle.py
\end{lstlisting}
\begin{figure}[!ht]
\includegraphics[width=\columnwidth]{./constructions/figs/rt_triangle.eps}
\caption{}
\label{fig:rt_triangle}
\end{figure}
\item Construct a triangle of sides $a=4$, $b=5$  and $c=6$.  
\label{prob:tri}
\\
\solution Let the vertices of  $\triangle ABC$ be 
\begin{align}
\label{eq:tri_basic}
\vec{A} = \myvec{p\\q}, \vec{B} = \myvec{0\\0}, \vec{C} = \myvec{a\\0}
\end{align}
%
\begin{align}
\label{eq:vec_def}
\vec{A}^T &\define \myvec{p & q}
\\
\norm{\vec{A}}^2 &= \vec{A}^T\vec{A} = \myvec{p & q}\myvec{p \\ q}
\\
&= p\times p + q\times q = p^2+q^2
\end{align}

Then
\begin{align}
\label{eq:c_tricoord}
AB &\define \norm{\vec{A}-\vec{B}}^2 = \norm{\vec{A}}^2  = c^2 \quad \because \vec{B} = \vec{0}
\\
\label{eq:a_tricoord}
BC &= \norm{\vec{C}-\vec{B}}^2 = \norm{\vec{C}}^2  = a^2
\\
AC &= \norm{\vec{A}-\vec{C}}^2 =    b^2
\label{eq:b_tricoord}
\end{align}
%
From \eqref{eq:b_tricoord},
\begin{align}
b^2 &=\norm{\vec{A}-\vec{C}}^2 = \norm{\vec{A}-\vec{C}}^T\norm{\vec{A}-\vec{C}}  
\\
&= \vec{A}^T\vec{A}+\vec{C}^T\vec{C}-\vec{A}^T\vec{C} - \vec{C}^T\vec{A} 
\\
&= \norm{\vec{A}}^2 + \norm{\vec{C}}^2 - 2\vec{A}^T\vec{C} \quad \brak{\because \vec{A}^T\vec{C} = \vec{C}^T\vec{A} } 
\\
&= a^2+c^2-2ap
\end{align}
%
yielding
\begin{align}
p&= \frac{a^2+c^2-b^2}{2a}
\end{align}
%
From \eqref{eq:c_tricoord}, 
\begin{align}
\norm{\vec{A}}^2 &= c^2 = p^2+q^2
\\
\implies q&= \pm \sqrt{c^2-p^2}
\end{align}

The following code plots Fig. \ref{fig:triangle}
\begin{lstlisting}
codes/constructions/draw_triangle.py
\end{lstlisting}
\begin{figure}[!ht]
\includegraphics[width=\columnwidth]{./constructions/figs/triangle.eps}
\caption{}
\label{fig:triangle}
\end{figure}
\item Construct a triangle of sides $a=5$, $b=6$  and $c=7$.  Construct a similar triangle whose sides are $\frac{7}{5}$ times the corresponding sides of the first triangle.
\\
\solution The sides of the similar triangle are $\frac{7}{5}a, \frac{7}{5}b$ and $\frac{7}{5}c$.
\item Construct an isosceles triangle whose base is $a=8$cm and altitude $AD=h=4$cm 
\\
\solution Using Baudhayana's theorem, 
\begin{align}
b = c= \sqrt{h^2+\brak{\frac{a}{2}}^2}
\end{align}
%
\item In $\triangle ABC$,  given that $a+b+c = 11, \angle B = 45^{\degree}$ and $\angle C = 45^{\degree}$, 
find 
$a,b,c$ and sketch the triangle.
\label{prob:const_tr_baudh_cramer}
\\
\solution From the given information, 
\begin{align}
\label{eq:tr_const_ex_sum}
a + b+c = 11
\\
\label{eq:tr_const_ex_isoc}
b = c \quad \brak{\because B = C = 45 \degree}
\\
a^2 = b^2+c^2 \quad \brak{\because A = 90\degree}
\label{eq:tr_const_ex_baudh}
\end{align}
%
From \eqref{eq:tr_const_ex_sum} and \eqref{eq:tr_const_ex_isoc},
\begin{align}
\label{eq:tr_const_ex_sum_ab}
a + 2b = 11
\end{align}
From  \eqref{eq:tr_const_ex_isoc} and \eqref{eq:tr_const_ex_baudh},
\begin{align}
\label{eq:tr_const_ex_sum_ab_baudh}
a^2 = 2b^2 \implies a - b\sqrt{2} =0
\end{align}
\eqref{eq:tr_const_ex_sum_ab} and \eqref{eq:tr_const_ex_sum_ab_baudh}
can be summarized as the matrix equation 
\begin{align}
\label{eq:tr_const_ex_sum_ab_mat}
\myvec{1 & 2\\1 & -\sqrt{2}}\myvec{a\\b} = \myvec{11\\0}
\end{align}
%
which can be solved using Cramer's rule as
\begin{align}
\label{eq:tr_const_ex_sum_ab_mat_sol}
a &= \frac{\mydet{11 & 2\\0 & -\sqrt{2}}}{\mydet{1 & 2\\1 & -\sqrt{2}}} = \frac{11 \times \brak{-\sqrt{2}}-2\times 0}{1\times \brak{-\sqrt{2}} - 2 \times 1} 
\\
&= \frac{11\sqrt{2}}{2+\sqrt{2}}
\\
b &= \frac{\mydet{1 & 11\\1 & 0}}{\mydet{1 & 2\\1 & -\sqrt{2}}} = \frac{11}{2+\sqrt{2}}
\end{align}
%
by expanding the determinants.  The following code may be used to compute $a, b$ and $c$.
\begin{lstlisting}
codes/constructions/triangle_det.py
\end{lstlisting}
\item Repeat Problem \ref{prob:const_tr_baudh_cramer} using a single matrix equation.
\\
\solution The equations 
\begin{align}
\label{eq:tr_const_ex_sum_abc}
a + 2b &= 11
\\
a - b\sqrt{2} &=0
\\
b-c &=0
\end{align}
can be expressed as a single matrix equation
\begin{align}
\label{eq:tr_const_ex_sum_abc_mat}
\myvec{1 & 2 & 0\\1 & -\sqrt{2} & 0\\0 & 1 & -1} \myvec{a\\b\\c} &= \myvec{11\\0\\0}
\end{align}
%
and can be solved using Cramer's rule as
\begin{align}
\label{eq:tr_const_ex_sum_abc_mat_sol}
a &= \frac{\mydet{11 & 2 & 0\\1 & -\sqrt{2} & 0\\0 & 1 & -1}}{\mydet{0 & 2 & 0\\1 & -\sqrt{2} & 0\\0 & 1 & -1}} 
\\
b &= \frac{\mydet{0 & 11 & 0\\1 & 0 & 0\\0 & 0 & -1}}{\mydet{0 & 2 & 0\\1 & -\sqrt{2} & 0\\0 & 1 & -1}} 
\\
c &= \frac{\mydet{0 & 2 & 11\\1 & -\sqrt{2} & 0\\0 & 1 & 0}}{\mydet{0 & 2 & 0\\1 & -\sqrt{2} & 0\\0 & 1 & -1}} 
\end{align}
The determinant
\begin{multline}
\mydet{0 & 2 & 0\\1 & -\sqrt{2} & 0\\0 & 1 & -1} = 0\times \mydet{-\sqrt{2} & 0\\ 1 & -1} 
\\
-2 \times  \mydet{1 &  0\\0 &  -1} + 0 \times \mydet{1 & -\sqrt{2} \\0 & 1 }
\end{multline}
%
The determinant can also be expressed as
\begin{multline}
\mydet{0 & 2 & 0\\1 & -\sqrt{2} & 0\\0 & 1 & -1} = 0\times \mydet{-\sqrt{2} & 0\\ 1 & -1} 
\\
-1 \times  \mydet{2 &  0\\1 &  -1} + 0 \times \mydet{2 & 0\\-\sqrt{2} & 0 }
\end{multline}
%
The determinants of larger matrices can be expressed similarly.
\item Draw $\triangle ABC$ with $a = 6, c = 5$ and $\angle B = 60 \degree$. 
\\
\solution 
In Fig. \eqref{ch2_cosine_formula}, $AD \perp BC$.
\begin{align}
\cos C &= \frac{y}{b},
\\
\cos B &= \frac{x}{b},
\end{align}
%
Thus, 
%
\begin{align}
a=x+y &= b \cos C + c \cos B, \\
b &= c \cos A + a \cos C \\
c &= b \cos A + a \cos B
\end{align}
%
  The above equations can be expressed in matrix form as
%
\begin{align}
\begin{pmatrix}
0 & c & b \\
c & 0 & a \\
b & a & 0
\end{pmatrix}
\begin{pmatrix}
\cos A \\
\cos B \\
\cos C
\end{pmatrix}
= 
\begin{pmatrix}
a\\
b\\
c
\end{pmatrix}
\end{align}
%
Using Cramer's rule and determinants,
%
\begin{align}
\cos A = \frac{
\begin{vmatrix}
a & c & b \\
b & 0 & a \\
c & a & 0
\end{vmatrix}
	}
	{
\begin{vmatrix}
0 & c & b \\
c & 0 & a \\
b & a & 0
\end{vmatrix}
	}
	&=\frac{ab^2 + ac^2 - a^3}{abc + abc} 
\\
&= \frac{b^2 + c^2 - a^2}{2bc}
\label{eq:cosC}
\end{align}
From \eqref{eq:cosC}
%Using the cosine formula, 
\begin{align}
\label{eq:b_cos_form}
b^2 = c^2+a^2-2ca\cos B
\end{align}
which is computed by the following code
\begin{lstlisting}
codes/constructions/cos_form.py
\end{lstlisting}
%
\begin{figure}[!ht]
	\begin{center}
		
		%\includegraphics[width=\columnwidth]{./constructions/figs/ch2_triang_ar}
		%\vspace*{-10cm}
		\resizebox{\columnwidth}{!}{\begin{tikzpicture}
[scale=2,>=stealth,point/.style={draw,circle,fill = black,inner sep=0.5pt},]

\node (D) at (0, 0)[point,label=below :$D$] {};
\node (A) at (0, 3)[point,label=above :$A$]{};
\node (B) at (-3, 0)[point,label=below left:$B$]{};
\node (C) at (3, 0)[point,label=below right:$C$]{};

\draw (D)--(B);
\draw (B)--(A);
\draw (A)--(C);
\draw (C)--(D);
\draw (D)--(A);

\tkzMarkRightAngle[size=.2](A,D,C)

\node [below] at (0,-0.3) {$a$};
\node [below] at (-1.5,0) {$x$};
\node [below] at (1.5, 0) {$y$};
\node [below] at (0.1,1.5) {$h$};
\node [above] at (-1.5,1.5){$c$};
\node [above] at (1.5,1.5){$b$};

\end{tikzpicture}
}
	\end{center}
	\caption{The cosine formula}
	\label{ch2_cosine_formula}	
\end{figure}

\item Draw $\triangle ABC$ with $a = 7, \angle B = 45\degree$ and $\angle A = 105 \degree$. 
\\
\solution In Fig. \eqref{ch2_cosine_formula},	
\begin{align}
\label{eq:sin_form_def}
\sin B &= \frac{h}{c}
\\
\sin C &= \frac{h}{b}
\end{align}
%
which can be used to show that
\begin{align}
\label{eq:sin_form}
\frac{\sin A}{a}=\frac{\sin B}{b}=\frac{\sin C}{c}
\end{align}
%
Thus, 
\begin{align}
%\label{eq:sin_form}
c = \frac{a\sin C}{\sin A}
\end{align}
where
\begin{align}
%\label{eq:sin_form}
C = 180-A-B
\end{align}
\item $\triangle ABC$ is right angled at $\vec{B}$.  If $a = 12$ and $b+c = 18$, find $b,c$ and draw the triangle.
\\
\solution From Baudhayana's theorem, 
\begin{align}
b^2 &= a^2 + c^2
\\
\implies \brak{18-c}^2 &= 12^2 +c^2
\end{align}
which can be simplified to obtain
\begin{align}
 36c -180&= 0
\\
\implies c&=5
\end{align}
%
and $b = 13$
\item Find a simpler solution for  Problem \ref{prob:const_tr_baudh_cramer} 
\\
\solution Use cosine formula.
\item In $\triangle ABC$,  $a = 7, \angle B = 75^{\degree}$ and $b+c = 13$. 
Alternatively, 
\begin{align}
a = b \cos C + c \cos B
\\
b \sin C = c \sin B
\\
a + b+c = 11
\end{align}
%
resulting  in the matrix equation 
\begin{align}
\begin{pmatrix}
1 & -\cos C & - \cos B
\\
0 & \sin C &- \sin B
\\
1 & 1 & 1
\end{pmatrix}
%\myvec{
%}
\myvec{a \\b\\c} = \myvec{0 \\ 0 \\ 11}
\end{align}

Solving the equivalent matrix equation gives the desired answer.

\end{enumerate}
%
