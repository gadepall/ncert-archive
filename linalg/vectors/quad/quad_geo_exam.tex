%\renewcommand{\theequation}{\theenumi}
%\begin{enumerate}[label=\arabic*.,ref=\thesubsection.\theenumi]
%\numberwithin{equation}{enumi}
%

\item Show that the points $\vec{A} = \myvec{1\\7}, \vec{B} = \myvec{4\\2}, \vec{C}=\myvec{-1\\-1},\vec{D}= \myvec{-4\\4} $  are the vertices of a square.
\\
\solution By inspection, 
%
\begin{align}
\frac{\vec{A}+\vec{C}}{2}=\frac{\vec{B}+\vec{D}}{2} = \myvec{0\\3}
\end{align}
%
Hence, the diagonals $AC$ and $BD$ bisect each other.
%
Also, 
\begin{align}
\brak{\vec{A}-\vec{C}}^T
\brak{\vec{B}-\vec{D}} = 0
\end{align}
%
$\implies AC \perp BD $.  Hence $ABCD$ is a square.
\item If the points
$
\vec{A} = \myvec{6\\1}, 
\vec{B} = \myvec{8\\2}, 
\vec{C} = \myvec{9\\4}, 
\vec{D} = \myvec{p\\3}
$
are the vertices of a parallelogram, taken in order, find the value of $p$.
\\
\solution In the parallelogram $ABCD$, $AC$ and $BD$ bisect each other.  This can be used to find $p$.
\item If $\vec{A} = \myvec{-5\\7}, \vec{B} = \myvec{-4\\-5}, \vec{C} = \myvec{-1\\-6}, \vec{D} = \myvec{4\\5}$, find the area of the quadrilateral $ABCD$.
%
\\
\solution The area of  $ABCD$ is the sum of the areas of trianges ABD and CBD and is given by 
\begin{multline}
\frac{1}{2}\norm{\brak{\vec{A}-\vec{B}}\times \brak{\vec{A}-\vec{D}}}
\\
+
\frac{1}{2}\norm{\brak{\vec{C}-\vec{B}}\times \brak{\vec{C}-\vec{D}}}
\end{multline}
\item Show that the points 
$\vec{A} = \myvec{1\\2\\3},
 \vec{B} = \myvec{-1\\-2\\-1},
\vec{C} = \myvec{2\\3\\2},
\vec{D} = \myvec{4\\7\\6}.
$
are the vertices of a parallelogram $ABCD$ but it is not a rectangle.
%
\\
\solution Since the direction vectors
%
\begin{align}
\vec{A}-\vec{B}&= \vec{D}-\vec{C}
\\
\vec{A}-\vec{D}&= \vec{B}-\vec{C}
\end{align}
%
$AB \parallel CD$ and $AD \parallel BC$.  Hence $ABCD$ is a parallelogram.  However, 
%
\begin{align}
\brak{\vec{A}-\vec{B}}^T\brak{ \vec{A}-\vec{D}}\ne 0
\end{align}
%
Hence, it is not a rectangle.
The following code plots Fig. \ref{fig:quad_3d}
%
\begin{lstlisting}
codes/triangle/quad_3d.py
\end{lstlisting}
%
\begin{figure}[!ht]
\includegraphics[width=\columnwidth]{./triangle/figs/quad_3d.eps}
\caption{}
\label{fig:quad_3d}
\end{figure}
%

\item Find the area of a parallelogram whose adjacent sides are given by the vectors \myvec{3\\1\\4} and \myvec{1\\-1\\1}.
%
\\
\solution  The area is given by 
%
\begin{align}
\frac{1}{2}\norm{\myvec{3\\1\\4} \times \myvec{1\\-1\\1}}
\end{align}
%
\item $ABCD$ is a rectangle formed by the points $\vec{A} = \myvec{-1\\-1}, \vec{B} = \myvec{-1\\4}, \vec{C} = \myvec{5\\4}, \vec{D} = \myvec{5\\-1}$. $ \vec{P}, \vec{Q}, \vec{R}, \vec{S}$ are the mid points of $AB, BC, CD, DA$ respectively.  Is the quadrilateral $PQRS$ a 
\begin{enumerate}
\item square?
\item rectangle?
\item rhombus?
\end{enumerate}
\solution
In general, the complex number $\myvec{a_1\\a_2}$ has the matrix representation
\begin{align}
\label{eq:3.4.1_Complex}
\myvec{a_1\\a_2} &= \myvec{a_1 & -a_2\\ a_2 & a_1}\myvec{1\\0}
\\
&= \vec{T}_a\myvec{1\\0}
\\
\implies \myvec{5\\-3}&=\myvec{5&3\\-3&5}\myvec{1\\0}
\end{align}
Then,
\begin{align}
\myvec{5\\-3}^3 &\triangleq\myvec{5&3\\-3&5}^3\myvec{1\\0}
\\
 &= \myvec{-10&198\\-198&-10} \myvec{1\\0}
\\
&=\myvec{-10\\-198}
\end{align}
The python code for above problem is
\begin{lstlisting}
codes/line/comp.py
\end{lstlisting}

\item $ABCD$ is a cyclic quadrilateral with 
\begin{align}
\angle A &= 4y+20
\\
\angle B &= 3y-5
\\
\angle C &= -4x
\\
\angle D &= -7x+5
\end{align}
%
Find its angles.
\\
\solution
In general, the complex number $\myvec{a_1\\a_2}$ has the matrix representation
\begin{align}
\label{eq:3.4.1_Complex}
\myvec{a_1\\a_2} &= \myvec{a_1 & -a_2\\ a_2 & a_1}\myvec{1\\0}
\\
&= \vec{T}_a\myvec{1\\0}
\\
\implies \myvec{5\\-3}&=\myvec{5&3\\-3&5}\myvec{1\\0}
\end{align}
Then,
\begin{align}
\myvec{5\\-3}^3 &\triangleq\myvec{5&3\\-3&5}^3\myvec{1\\0}
\\
 &= \myvec{-10&198\\-198&-10} \myvec{1\\0}
\\
&=\myvec{-10\\-198}
\end{align}
The python code for above problem is
\begin{lstlisting}
codes/line/comp.py
\end{lstlisting}

\item Draw a quadrilateral in the Cartesian plane, whose vertices are \myvec{– 4\\ 5}, \myvec{0\\ 7}, \myvec{5\\ – 5} and \myvec{– 4\\ –2}. Also, find its area.
\\
\solution
In general, the complex number $\myvec{a_1\\a_2}$ has the matrix representation
\begin{align}
\label{eq:3.4.1_Complex}
\myvec{a_1\\a_2} &= \myvec{a_1 & -a_2\\ a_2 & a_1}\myvec{1\\0}
\\
&= \vec{T}_a\myvec{1\\0}
\\
\implies \myvec{5\\-3}&=\myvec{5&3\\-3&5}\myvec{1\\0}
\end{align}
Then,
\begin{align}
\myvec{5\\-3}^3 &\triangleq\myvec{5&3\\-3&5}^3\myvec{1\\0}
\\
 &= \myvec{-10&198\\-198&-10} \myvec{1\\0}
\\
&=\myvec{-10\\-198}
\end{align}
The python code for above problem is
\begin{lstlisting}
codes/line/comp.py
\end{lstlisting}


\item Find the area of a rhombus if its vertices are 
\begin{align}
\vec{P} &= \myvec{3\\0}, \vec{Q} =\myvec{4\\5},
\\
\vec{R} &= \myvec{-1\\4}, \vec{S} = \myvec{-2\\-1} 
\end{align}
taken in order.
\\
\solution
In general, the complex number $\myvec{a_1\\a_2}$ has the matrix representation
\begin{align}
\label{eq:3.4.1_Complex}
\myvec{a_1\\a_2} &= \myvec{a_1 & -a_2\\ a_2 & a_1}\myvec{1\\0}
\\
&= \vec{T}_a\myvec{1\\0}
\\
\implies \myvec{5\\-3}&=\myvec{5&3\\-3&5}\myvec{1\\0}
\end{align}
Then,
\begin{align}
\myvec{5\\-3}^3 &\triangleq\myvec{5&3\\-3&5}^3\myvec{1\\0}
\\
 &= \myvec{-10&198\\-198&-10} \myvec{1\\0}
\\
&=\myvec{-10\\-198}
\end{align}
The python code for above problem is
\begin{lstlisting}
codes/line/comp.py
\end{lstlisting}

\item Without using distance formula, show that points \myvec{– 2\\ – 1}, \myvec{4\\ 0}, \myvec{3\\ 3} and \myvec{–3\\ 2} are the vertices of a parallelogram.
\\
\solution
The following python code plots Fig.\ref{fig:2.2.5_qfour}.
	\begin{lstlisting}
	./solutions/5/codes/quadrilateral/q4.py
	\end{lstlisting}
	

	\begin{align}
\because	\vec{A - B}&=\vec{D - C}
	\\
	\vec{A - D}&=\vec{B - C},
	\end{align}
	
	$AB \parallel CD$ and $AD \parallel BC$.  Hence, $ABCD$ is a $\parallel$gm.
	\begin{figure}[!ht]
	\centering
	\includegraphics[width=\columnwidth]{./solutions/5/figs/quadrilateral/q4.eps}
	\caption{}
	\label{fig:2.2.5_qfour}	
	\end{figure}

\item  Find the area of the quadrilateral whose vertices, taken in order, are 
 \myvec{-4\\2},  \myvec{-3\\-5},  \myvec{3\\-2},  \myvec{2\\3}. 
\\
\solution
In general, the complex number $\myvec{a_1\\a_2}$ has the matrix representation
\begin{align}
\label{eq:3.4.1_Complex}
\myvec{a_1\\a_2} &= \myvec{a_1 & -a_2\\ a_2 & a_1}\myvec{1\\0}
\\
&= \vec{T}_a\myvec{1\\0}
\\
\implies \myvec{5\\-3}&=\myvec{5&3\\-3&5}\myvec{1\\0}
\end{align}
Then,
\begin{align}
\myvec{5\\-3}^3 &\triangleq\myvec{5&3\\-3&5}^3\myvec{1\\0}
\\
 &= \myvec{-10&198\\-198&-10} \myvec{1\\0}
\\
&=\myvec{-10\\-198}
\end{align}
The python code for above problem is
\begin{lstlisting}
codes/line/comp.py
\end{lstlisting}

\item The two opposite vertices of a square are \myvec{-1\\2},  \myvec{3\\2}. Find the coordinates of the other two vertices.
\\
\solution
\renewcommand{\theequation}{\theenumi}
\begin{enumerate}[label=\arabic*.,ref=\thesubsubsection.\theenumi]
\numberwithin{equation}{enumi}
\item In the given question,
\\
The sample size = Total number of possibilities(S)=6
\begin{align}
\myvec{1&2&3&4&5&6}
\end{align}
Event size= Odd number =3
\begin{align}
\myvec{1&3&5}
\end{align}
Probability for this event is = $\frac{1}{2}$
\\
The python code for the distribution of data,
\begin{lstlisting}
prob/codes/prob6_b.py
\end{lstlisting}
This shows the diagrametic representation of dice with the live update of probability with the role of dice.
\end{enumerate}

\item Find the area of a parallelogram whose adjacent sides are given by the vectors \myvec{3\\1\\4} and \myvec{1\\-1\\1}.
\\
\solution
In general, the complex number $\myvec{a_1\\a_2}$ has the matrix representation
\begin{align}
\label{eq:3.4.1_Complex}
\myvec{a_1\\a_2} &= \myvec{a_1 & -a_2\\ a_2 & a_1}\myvec{1\\0}
\\
&= \vec{T}_a\myvec{1\\0}
\\
\implies \myvec{5\\-3}&=\myvec{5&3\\-3&5}\myvec{1\\0}
\end{align}
Then,
\begin{align}
\myvec{5\\-3}^3 &\triangleq\myvec{5&3\\-3&5}^3\myvec{1\\0}
\\
 &= \myvec{-10&198\\-198&-10} \myvec{1\\0}
\\
&=\myvec{-10\\-198}
\end{align}
The python code for above problem is
\begin{lstlisting}
codes/line/comp.py
\end{lstlisting}

\item Find the area of a rectangle $ABCD$ with vertices
$\vec{A} = \myvec{-1\\\frac{1}{2}\\ 4},
 \vec{B} = \myvec{1\\\frac{1}{2}\\ 4},
\vec{C} = \myvec{1\\-\frac{1}{2}\\ 4},
\vec{D} = \myvec{-1\\-\frac{1}{2}\\ 4}.
$
\\
\solution
In general, the complex number $\myvec{a_1\\a_2}$ has the matrix representation
\begin{align}
\label{eq:3.4.1_Complex}
\myvec{a_1\\a_2} &= \myvec{a_1 & -a_2\\ a_2 & a_1}\myvec{1\\0}
\\
&= \vec{T}_a\myvec{1\\0}
\\
\implies \myvec{5\\-3}&=\myvec{5&3\\-3&5}\myvec{1\\0}
\end{align}
Then,
\begin{align}
\myvec{5\\-3}^3 &\triangleq\myvec{5&3\\-3&5}^3\myvec{1\\0}
\\
 &= \myvec{-10&198\\-198&-10} \myvec{1\\0}
\\
&=\myvec{-10\\-198}
\end{align}
The python code for above problem is
\begin{lstlisting}
codes/line/comp.py
\end{lstlisting}


%\end{enumerate}
%
