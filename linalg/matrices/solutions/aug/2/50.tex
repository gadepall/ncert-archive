\begin{align}
    \vec{AC}&=\myvec{0&6&7\\-6&0&8\\7&-8&0}\myvec{2\\-2\\3}\\
  &=\myvec{9\\12\\30}
\end{align}
\begin{align}
    \vec{BC}&=\myvec{0&1&1\\1&0&2\\1&2&0}\myvec{2\\-2\\3}\\
  &=\myvec{1\\8\\-2}
\end{align}
Now,
\begin{align}
    \vec{AC}+\vec{BC}&=\myvec{9\\12\\30}+\myvec{1\\8\\-2}\\
    &=\myvec{10\\20\\28} \label{aug/50/eq-2}
\end{align}
and,
\begin{align}
    (\vec{A}+\vec{B})\vec{C}&=\myvec{0&7&8\\-5&0&10\\8&-6&0}\myvec{2\\-2\\3}\\
&=\myvec{10\\20\\28} \label{aug/50/eq-1}
\end{align}
From \eqref{aug/50/eq-1} and \eqref{aug/50/eq-2},
\begin{align}
    (\vec{A}+\vec{B})\vec{C}=\vec{AC}+\vec{BC}
\end{align}