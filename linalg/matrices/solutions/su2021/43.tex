\begin{align}
A^\top A&=I
\\
\implies \myvec{0 & x & x\\2y & y & -y\\z & -z & z}\myvec{0 & 2y & z\\x & y & -z\\x & -y & z} &= \myvec{1 & 0 & 0\\0 & 1 & 0\\0 & 0 & 1}
\\
\text{or, } \myvec{2x^2 & 0 & 0\\0 & 6y^2 & 0\\0 & 0 & 3z^2} = \myvec{1 & 0 & 0\\0 & 1 & 0\\0 & 0 & 1}
\end{align}
Hence, 
\begin{align}
2x^2 &= 1 
\implies 
x = \pm\frac{1}{\sqrt{2}},
\\
6y^2 &= 1
\implies 
y = \pm\frac{1}{\sqrt{6}}
\\
3z^2 &= 1
\implies 
z = \pm\frac{1}{\sqrt{3}}
\end{align}