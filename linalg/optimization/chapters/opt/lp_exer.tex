%\renewcommand{\theequation}{\theenumi}
%\begin{enumerate}[label=\thesubsection.\arabic*.,ref=\thesubsection.\theenumi]
%%\begin{enumerate}[label=\arabic*.,ref=\thesection.\theenumi]
%\numberwithin{equation}{enumi}

\item Find the maximum and minimum values, if any, of the following functions given by 
%
\begin{enumerate}
\item $f(x) = \brak{2x-1}^2+3$
\\
\solution
In general, the complex number $\myvec{a_1\\a_2}$ has the matrix representation
\begin{align}
\label{eq:3.4.1_Complex}
\myvec{a_1\\a_2} &= \myvec{a_1 & -a_2\\ a_2 & a_1}\myvec{1\\0}
\\
&= \vec{T}_a\myvec{1\\0}
\\
\implies \myvec{5\\-3}&=\myvec{5&3\\-3&5}\myvec{1\\0}
\end{align}
Then,
\begin{align}
\myvec{5\\-3}^3 &\triangleq\myvec{5&3\\-3&5}^3\myvec{1\\0}
\\
 &= \myvec{-10&198\\-198&-10} \myvec{1\\0}
\\
&=\myvec{-10\\-198}
\end{align}
The python code for above problem is
\begin{lstlisting}
codes/line/comp.py
\end{lstlisting}

\item 
\begin{align}
    f(x)=9x^2+12x+2 
\end{align}
%
\solution
In general, the complex number $\myvec{a_1\\a_2}$ has the matrix representation
\begin{align}
\label{eq:3.4.1_Complex}
\myvec{a_1\\a_2} &= \myvec{a_1 & -a_2\\ a_2 & a_1}\myvec{1\\0}
\\
&= \vec{T}_a\myvec{1\\0}
\\
\implies \myvec{5\\-3}&=\myvec{5&3\\-3&5}\myvec{1\\0}
\end{align}
Then,
\begin{align}
\myvec{5\\-3}^3 &\triangleq\myvec{5&3\\-3&5}^3\myvec{1\\0}
\\
 &= \myvec{-10&198\\-198&-10} \myvec{1\\0}
\\
&=\myvec{-10\\-198}
\end{align}
The python code for above problem is
\begin{lstlisting}
codes/line/comp.py
\end{lstlisting}

\item $f(x) = -\brak{x-1}^2+10$
\item $f(x) = x^2$.
\end{enumerate}
\item Solve
\begin{align}
\min_{\vec{x}} Z &= \myvec{3 & 2}\vec{x}
\\
s.t. \quad 
\myvec{
-1 & -1
\\
3 & 5
}
\vec{x} &\preceq \myvec{-8\\15}
\\
\vec{x} &\succeq \vec{0}
\end{align}
\item Solve
\begin{align}
\min_{\vec{x}} Z &= \myvec{200 & 500}\vec{x}
\\
s.t. \quad 
\myvec{
-1 & -2
\\
3 & 4
}
\vec{x} &\preceq \myvec{-10\\24}
\\
\vec{x} &\succeq \vec{0}
\end{align}
\item Maximise Z=3x+4y\\
subject to the constraints : x+y$\leq$4, x$\geq$0, y$\geq$ 0.\\
\item Minimise Z=-3x+4y\\
subject to x+2y$\leq$8, 3x+2y$\leq$12, x$\geq$0, y$\geq$0.\\
\item Maximise Z=5x+3y
subject to 3x+5y$\leq$15, 5x+2y$\leq$10, x$\geq$0, y$\geq$0.\\
\solution
In general, the complex number $\myvec{a_1\\a_2}$ has the matrix representation
\begin{align}
\label{eq:3.4.1_Complex}
\myvec{a_1\\a_2} &= \myvec{a_1 & -a_2\\ a_2 & a_1}\myvec{1\\0}
\\
&= \vec{T}_a\myvec{1\\0}
\\
\implies \myvec{5\\-3}&=\myvec{5&3\\-3&5}\myvec{1\\0}
\end{align}
Then,
\begin{align}
\myvec{5\\-3}^3 &\triangleq\myvec{5&3\\-3&5}^3\myvec{1\\0}
\\
 &= \myvec{-10&198\\-198&-10} \myvec{1\\0}
\\
&=\myvec{-10\\-198}
\end{align}
The python code for above problem is
\begin{lstlisting}
codes/line/comp.py
\end{lstlisting}

\item Minimise Z=3x+5y
such that x+3y$\geq$3, x+y$\geq$2, x,y$\geq$0.\\
\item Maximise Z=3x+2y
subject to x+2y$\leq$10, 3x+y$\leq$15, x,y$\geq$0.\\
\solution
In general, the complex number $\myvec{a_1\\a_2}$ has the matrix representation
\begin{align}
\label{eq:3.4.1_Complex}
\myvec{a_1\\a_2} &= \myvec{a_1 & -a_2\\ a_2 & a_1}\myvec{1\\0}
\\
&= \vec{T}_a\myvec{1\\0}
\\
\implies \myvec{5\\-3}&=\myvec{5&3\\-3&5}\myvec{1\\0}
\end{align}
Then,
\begin{align}
\myvec{5\\-3}^3 &\triangleq\myvec{5&3\\-3&5}^3\myvec{1\\0}
\\
 &= \myvec{-10&198\\-198&-10} \myvec{1\\0}
\\
&=\myvec{-10\\-198}
\end{align}
The python code for above problem is
\begin{lstlisting}
codes/line/comp.py
\end{lstlisting}

\item Minimise Z=x+2y
subject to 2x+y$\geq$3, x+2y$\geq$6, x,y$\geq$0.\\
Show that the minimum of Z occurs at more than two points.\\
\solution
In general, the complex number $\myvec{a_1\\a_2}$ has the matrix representation
\begin{align}
\label{eq:3.4.1_Complex}
\myvec{a_1\\a_2} &= \myvec{a_1 & -a_2\\ a_2 & a_1}\myvec{1\\0}
\\
&= \vec{T}_a\myvec{1\\0}
\\
\implies \myvec{5\\-3}&=\myvec{5&3\\-3&5}\myvec{1\\0}
\end{align}
Then,
\begin{align}
\myvec{5\\-3}^3 &\triangleq\myvec{5&3\\-3&5}^3\myvec{1\\0}
\\
 &= \myvec{-10&198\\-198&-10} \myvec{1\\0}
\\
&=\myvec{-10\\-198}
\end{align}
The python code for above problem is
\begin{lstlisting}
codes/line/comp.py
\end{lstlisting}

\item Solve:
\begin{align}
    \max_{\{x\}} Z=\myvec{4 & 1}\vec{x}\label{eq:solutions/4/38/eq:1}\\
    s.t\quad \myvec{1 & 1\\3 & 1} \leq \myvec{50\\90}\label{eq:solutions/4/38/eq:2}\\
    \Vec{x}\geq 0 \label{eq:solutions/4/38/eq:3}
\end{align}
\solution
In general, the complex number $\myvec{a_1\\a_2}$ has the matrix representation
\begin{align}
\label{eq:3.4.1_Complex}
\myvec{a_1\\a_2} &= \myvec{a_1 & -a_2\\ a_2 & a_1}\myvec{1\\0}
\\
&= \vec{T}_a\myvec{1\\0}
\\
\implies \myvec{5\\-3}&=\myvec{5&3\\-3&5}\myvec{1\\0}
\end{align}
Then,
\begin{align}
\myvec{5\\-3}^3 &\triangleq\myvec{5&3\\-3&5}^3\myvec{1\\0}
\\
 &= \myvec{-10&198\\-198&-10} \myvec{1\\0}
\\
&=\myvec{-10\\-198}
\end{align}
The python code for above problem is
\begin{lstlisting}
codes/line/comp.py
\end{lstlisting}

\item Minimise and Maximise Z=5x+10y
subject to x+2y$\leq$120, x+y$\geq$60, x-2y$\geq$0, x,y$\geq$0.\\
\solution
In general, the complex number $\myvec{a_1\\a_2}$ has the matrix representation
\begin{align}
\label{eq:3.4.1_Complex}
\myvec{a_1\\a_2} &= \myvec{a_1 & -a_2\\ a_2 & a_1}\myvec{1\\0}
\\
&= \vec{T}_a\myvec{1\\0}
\\
\implies \myvec{5\\-3}&=\myvec{5&3\\-3&5}\myvec{1\\0}
\end{align}
Then,
\begin{align}
\myvec{5\\-3}^3 &\triangleq\myvec{5&3\\-3&5}^3\myvec{1\\0}
\\
 &= \myvec{-10&198\\-198&-10} \myvec{1\\0}
\\
&=\myvec{-10\\-198}
\end{align}
The python code for above problem is
\begin{lstlisting}
codes/line/comp.py
\end{lstlisting}

\item Minimise and Maximise Z=x+2y
subject to x+2y$\geq$100, 2x-y$\leq$0, 2x+y$\leq$200; x,y$\geq$0.\\
\solution
In general, the complex number $\myvec{a_1\\a_2}$ has the matrix representation
\begin{align}
\label{eq:3.4.1_Complex}
\myvec{a_1\\a_2} &= \myvec{a_1 & -a_2\\ a_2 & a_1}\myvec{1\\0}
\\
&= \vec{T}_a\myvec{1\\0}
\\
\implies \myvec{5\\-3}&=\myvec{5&3\\-3&5}\myvec{1\\0}
\end{align}
Then,
\begin{align}
\myvec{5\\-3}^3 &\triangleq\myvec{5&3\\-3&5}^3\myvec{1\\0}
\\
 &= \myvec{-10&198\\-198&-10} \myvec{1\\0}
\\
&=\myvec{-10\\-198}
\end{align}
The python code for above problem is
\begin{lstlisting}
codes/line/comp.py
\end{lstlisting}

\item Reshma wishes to mix two types of food P and Q in such a way that the vitamin
contents of the mixture contain at least 8 units of vitamin A and 11 units of
vitamin B. Food P costs Rs 60/kg and Food Q costs Rs 80/kg. Food P contains
3 units/kg of Vitamin A and 5 units/kg of Vitamin B while food Q contains
4 units/kg of Vitamin A and 2 units/kg of vitamin B. Determine the minimum cost
of the mixture.\\
\solution

\begin{table}[!ht]
\centering
\resizebox{\columnwidth}{!}{\begin{tabular}{|c|c|c|c|} 
\hline
Food & Vitamin A & Vitamin B & Cost \\
\hline
P & 3 units/kg & 5 units/kg & 60 Rs/kg \\ 
\hline
Q & 4 units/kg & 2 units/kg & 80 Rs/kg \\ 
\hline
Requirement & 8 units/kg & 11 units/kg & \\ 
\hline
\end{tabular}}
\caption{Food Requirements}
\label{opt/12/tab:table1}
\end{table}
Let the mixture contain $x$ kg of food P and $y$ kg of food Q be $y$  such that 
\begin{align}
    x \geq 0 \\
    y \geq 0 
\end{align}
According to the question,
\begin{align}
    3x+4y &\geq 8 \\
    5x+2y &\geq 11
\end{align}
$\therefore$ Our problem is
\begin{align}
        \min_{\vec{x}} Z &= \myvec{60 & 80}\vec{x}\\
        s.t. \quad 
        \myvec{3 & 4 \\ 5 & 2 }\vec{x} &\preceq \myvec{8\\11} \\
        \vec{x} &\preceq \vec{0}
\end{align}
Lagrangian function is given by
\begin{equation}
\begin{aligned}
    &L(\vec{x},\boldsymbol{\lambda}) \\ &= \myvec{60 & 80}\vec{x}+\lcbrak{\sbrak{\myvec{3 & 4}\vec{x}-8}} \\ &+ \sbrak{\myvec{5 & 2}\vec{x}-11} \\ &+ \sbrak{\myvec{-1 & 0}\vec{x}} +\rcbrak{\sbrak{\myvec{0 & -1}\vec{x}}}\boldsymbol{\lambda}
\end{aligned}
\end{equation}
where,
\begin{align}
    \boldsymbol{\lambda} &= \myvec{\lambda_1 \\ \lambda_2 \\ \lambda_3 \\ \lambda_4}
\end{align}
Now,
\begin{align}
    \nabla L(\vec{x},\boldsymbol{\lambda}) &= \myvec{60+ \myvec{3 & 5 & -1 & 0 }\boldsymbol{\lambda}\\ 80+\myvec{4 & 2 & 0 & -1}\boldsymbol{\lambda} \\ \myvec{3 & 4}\vec{x}-8 \\ \myvec{5 & 2}\vec{x}-11 \\ \myvec{-1 & 0}\vec{x} \\ \myvec{0 & -1}\vec{x}}
\end{align}
$\therefore$ Lagrangian matrix is given by
\begin{align}
    \myvec{0 & 0 & 3 & 5 & -1 & 0 \\ 0 & 0 & 4 & 2 & 0 & -1 \\ 3 & 4 & 0 & 0 & 0 & 0 \\ 5 & 2 & 0 & 0 & 0 & 0 \\ -1 & 0 & 0 & 0 & 0 & 0 \\ 0 & -1 & 0 & 0 & 0 & 0}\myvec{\vec{x} \\ \boldsymbol{\lambda} } &= \myvec{-60 \\ -80 \\ 8 \\ 11 \\ 0 \\0 }
\end{align}
Considering $\lambda_1,\lambda_2$ as only active multiplier,
\begin{align}
    \myvec{0 & 0 & 3 & 5 \\ 0 & 0 & 4 & 2 \\ 3 & 4 & 0 & 0 \\ 5 & 2 & 0 & 0}\myvec{\vec{x}\\ \boldsymbol{\lambda}} &= \myvec{-60 \\ -80 \\ 8 \\ 11}
\end{align}
resulting in,
\begin{align}
    \myvec{\vec{x} \\ \boldsymbol{\lambda}} &= \myvec{0 & 0 & 3 & 5 \\ 0 & 0 & 4 & 2 \\ 3 & 4 & 0 & 0 \\ 5 & 2 & 0 & 0}^{-1}\myvec{-60 \\-80 \\ 8 \\ 11}
    \\
    \implies   \myvec{\vec{x} \\ \boldsymbol{\lambda}} &= \myvec{0 & 0 & \frac{-28}{196} & \frac{56}{196} \\ 0 & 0 & \frac{70}{196} & \frac{-42}{196} \\ \frac{-28}{196} & \frac{70}{196} & 0 & 0 \\ \frac{56}{196} & \frac{-42}{196} & 0 & 0}\myvec{-60 \\-80 \\ 8 \\ 11}
    \\
    \implies \myvec{\vec{x} \\ \boldsymbol{\lambda}} &= \myvec{2 \\ \frac{1}{2} \\ -20 \\ 0 }
\end{align}
$\because \boldsymbol{\lambda}=\myvec{-20 \\ 0} \preceq \vec{0} $
\\
$\therefore$ Optimal solution is given by
\begin{align}
    \vec{x} &= \myvec{2\\ \frac{1}{2}} \\
    Z &= \myvec{60 & 80}\vec{x} \\
    &= \myvec{60 & 80}\myvec{2 \\ \frac{1}{2}} \\
    &= 160
\end{align}
By using cvxpy in python,
\begin{align}
    \vec{x}=\myvec{2.11436237\\0.41422822}\\
    Z = 159.99999999
\end{align}
Fig. \ref{opt/12/fig:Graphical Solution}	verifies this result.
%
\begin{figure}[!ht]
\centering
\includegraphics[width=\columnwidth]{solutions/su2021/2/12/Graphical Solution_10.png}
\caption{Graphical Solution}
\label{opt/12/fig:Graphical Solution}	
\end{figure}




\item One kind of cake requires 200g of flour and 25g of fat, and another kind of cake
requires 100g of flour and 50g of fat. Find the maximum number of cakes which
can be made from 5kg of flour and 1 kg of fat assuming that there is no shortage
of the other ingredients used in making the cakes.\\
\solution
Data from the given question is available in Table \ref{constr/circ/61/tab:table1}:
%
\begin{table}[!ht]
\begin{center}
\begin{tabular}{ | m{2cm} | m{1.5cm}| m{2cm} | m{1.5cm} |} 
\hline
& Symbols & Circle1  \\
\hline
Centre & $\vec{C}$ & \myvec{0\\0}  \\ 
\hline
Radius & $r$& 3.4 \\ 
\hline
\end{tabular}
\end{center}
\caption{Input values}
\label{constr/circ/61/tab:table1}
\end{table}
%
Let 
\begin{align}
      \vec{P} &= r \myvec{\cos \theta_1\\ \sin \theta_1} =  \myvec{1.7\\2.9}
      \\
      \vec{Q} &= r \myvec{\cos \theta_2\\ \sin \theta_2} =\myvec{-2.4\\2.4}
 \end{align}
 Then the perpendicular bisector of $PQ$ passes through 
 \begin{align}
    \vec{M} = \frac{\vec{P}+\vec{Q}}{2} 
\end{align}
and has normal vector 
\begin{align}
    \vec{n} = \vec{P}-\vec{Q}
\end{align}
resulting in the equation
\begin{align}
    \brak{\vec{P}-\vec{Q}}^\top\brak{\vec{x}-\frac{\vec{P}+\vec{Q}}{2} } = 0
    \implies \brak{\vec{P}-\vec{Q}}^\top\vec{x} = 0
\end{align}
%
after simplification.  It is obvious that $\vec{O}$ satisfies the above equuation as can be verified in Fig.     \ref{constr/circ/61/fig: perpenicular bisector of the chord passes through the center}.
%
\begin{figure}[ht]
    \centering
    \includegraphics[width=\columnwidth]{solutions/circle/61/FIG.3.png}
    \caption{perpendicular bisector of the chord passes through the center}
    \label{constr/circ/61/fig: perpenicular bisector of the chord passes through the center}
\end{figure}








\item A factory makes tennis rackets and cricket bats. A tennis racket takes 1.5 hours
of machine time and 3 hours of craftman’s time in its making while a cricket bat
takes 3 hour of machine time and 1 hour of craftman’s time. In a day, the factory
has the availability of not more than 42 hours of machine time and 24 hours of
craftsman’s time.\\
(i) What number of rackets and bats must be made if the factory is to work
at full capacity?\\
(ii)If the profit on a racket and on a bat is Rs 20 and Rs 10 respectively, find
the maximum profit of the factory when it works at full capacity.\\
\solution

\begin{table}[!ht]
\centering
\resizebox{\columnwidth}{!}{\begin{tabular}{|c|c|c|c|} 
\hline
item & Machine hours & Craftman's hours & profit \\
\hline
Tennis Racket & 1.5 & 3  &  20  \\ 
\hline
Cricket Bats & 3 & 1 &  10  \\ 
\hline
Maximum time Available & 42 & 24 & \\ 
\hline
\end{tabular}}
\caption{factory Requirements}
\label{opt/14/tab:table1}
\end{table}
Let the number of Tennis Rackets  be $x$ and the number of cricket bats be $y$  such that 
\begin{align}
x \geq 0 \\
y \geq 0 
\end{align}
According to the question,
\begin{align}
1.5x+3y &\leq 42 \\
\implies 3x+6y &\leq 84 \\
\implies x+2y &\leq 28 
\end{align}
and,
\begin{align}
3x+y &\leq 24 
\end{align}
$\therefore$ Our problem is
\begin{align}
\max_{\vec{x}} Z &= \myvec{20 & 10}\vec{x}\\
s.t. \quad \myvec{1 & 2 \\ 3 & 1}\vec{x} &\preceq \myvec{28\\24} 
\end{align}
Lagrangian function is given by
\begin{equation}
\begin{aligned}
&L(\vec{x},\boldsymbol{\lambda}) \\ &= \myvec{20 & 10}\vec{x}+\lcbrak{\sbrak{\myvec{1 & 2}\vec{x}-28}} \\ &+ \sbrak{\myvec{3 & 1}\vec{x}-24}\\ &+ \sbrak{\myvec{-1 & 0}\vec{x}} +\rcbrak{\sbrak{\myvec{0 & -1}\vec{x}}}\boldsymbol{\lambda}
\end{aligned}
\end{equation}
where,
\begin{align}
\boldsymbol{\lambda} &= \myvec{\lambda_1 \\ \lambda_2 \\ \lambda_3 \\ \lambda_4 \\ \lambda_5 \\ \lambda_6}
\end{align}
Now,
\begin{align}
\nabla L(\vec{x},\boldsymbol{\lambda}) &= \myvec{20+ \myvec{1 & 3  & -1 & 0 }\boldsymbol{\lambda}\\ 10+\myvec{2 & 1 & 0 & -1}\boldsymbol{\lambda} \\ \myvec{1 & 2}\vec{x}-28 \\ \myvec{3 & 1}\vec{x}-24 \\  \myvec{-1 & 0}\vec{x} \\ \myvec{0 & -1}\vec{x}}
\end{align}
$\therefore$ Lagrangian matrix is given by
\begin{align}
\myvec{0 & 0 & 1 & 3 & -1 & 0 \\ 0 & 0 & 2 & 1  & 0 & -1 \\ 1 & 2 & 0 & 0 & 0 & 0 \\ 3 & 1 & 0 & 0 & 0 & 0  \\ -1 & 0 & 0 & 0 & 0 & 0  \\ 0 & -1 & 0 & 0 & 0 & 0 }\myvec{\vec{x} \\ \boldsymbol{\lambda} } &= \myvec{-20 \\ -10 \\ 28 \\ 24 \\ 0 \\0 }
\end{align}
Considering $\lambda_1,\lambda_2$ as only active multiplier,
\begin{align}
\myvec{0 & 0 & 1 & 3 \\ 0 & 0 & 2 & 1 \\ 1 & 2 & 0 & 0 \\ 3 & 1 & 0 & 0}\myvec{\vec{x}\\ \boldsymbol{\lambda}} &= \myvec{-20 \\ -10 \\ 28 \\ 24}
\end{align}
resulting in,
\begin{align}
\myvec{\vec{x} \\ \boldsymbol{\lambda}} &= \myvec{0 & 0 & 1 & 3 \\ 0 & 0 & 2 & 1 \\ 1 & 2 & 0 & 0 \\ 3 & 1 & 0 & 0}^{-1}\myvec{-20 \\ -10 \\ 28 \\ 24}
\\
\implies   \myvec{\vec{x} \\ \boldsymbol{\lambda}} &= \myvec{0 & 0 & \frac{-1}{5} & \frac{2}{5} \\ 0 & 0 & \frac{3}{5} & \frac{-1}{5} \\ \frac{-1}{5} & \frac{3}{5} & 0 & 0 \\ \frac{2}{5} & \frac{-1}{5} & 0 & 0}\myvec{-20 \\ -10 \\ 28 \\ 24}
\\
\implies \myvec{\vec{x} \\ \boldsymbol{\lambda}} &= \myvec{4 \\ 12 \\ -2 \\ -6 }
\end{align}
$\because \boldsymbol{\lambda}=\myvec{-2 \\ -6} \succ \vec{0} $
\\
$\therefore$ Optimal solution is given by
\begin{align}
\vec{x} &= \myvec{4\\12} \\
Z &= \myvec{20 & 10}\vec{x} \\
&= \myvec{20 & 10}\myvec{4 \\ 12} \\
&= 200
\end{align}
By using cvxpy in python ,
\begin{align}
\vec{x}=\myvec{3.99999998\\12.0000000}\\
Z = 199.99999964
\end{align}
Hence ,\boxed{x=4} Tennis Rackets and \boxed{y=12} Cricket Bats should be used to maximum time Available profit \boxed{Z=200} as can be
verified from Fig. \ref{opt/14/fig: Graphical Solution}.	
\begin{enumerate}
\item 4 Tennis Rackets and 12 Cricket Bats must be made so that factory runs at full capacity.
\item Maximum profit is Rs 200, When 4 Tennis Bats and 12 Cricket Bats are produced.
\end{enumerate}
%
\begin{figure}[!ht]
\centering
\includegraphics[width=\columnwidth]{solutions/su2021/2/14/download.png}
\caption{Graphical Solution}
\label{opt/14/fig: Graphical Solution}	
\end{figure}



\item A manufacturer produces nuts and bolts. It takes 1 hour of work on machine A
and 3 hours on machine B to produce a package of nuts. It takes 3 hours on
machine A and 1 hour on machine B to produce a package of bolts. He earns a
profit of Rs17.50 per package on nuts and Rs 7.00 per package on bolts. How
many packages of each should be produced each day so as to maximise his
profit, if he operates his machines for at the most 12 hours a day?\\
\solution
Data from the given question is available in Table \ref{constr/circ/61/tab:table1}:
%
\begin{table}[!ht]
\begin{center}
\begin{tabular}{ | m{2cm} | m{1.5cm}| m{2cm} | m{1.5cm} |} 
\hline
& Symbols & Circle1  \\
\hline
Centre & $\vec{C}$ & \myvec{0\\0}  \\ 
\hline
Radius & $r$& 3.4 \\ 
\hline
\end{tabular}
\end{center}
\caption{Input values}
\label{constr/circ/61/tab:table1}
\end{table}
%
Let 
\begin{align}
      \vec{P} &= r \myvec{\cos \theta_1\\ \sin \theta_1} =  \myvec{1.7\\2.9}
      \\
      \vec{Q} &= r \myvec{\cos \theta_2\\ \sin \theta_2} =\myvec{-2.4\\2.4}
 \end{align}
 Then the perpendicular bisector of $PQ$ passes through 
 \begin{align}
    \vec{M} = \frac{\vec{P}+\vec{Q}}{2} 
\end{align}
and has normal vector 
\begin{align}
    \vec{n} = \vec{P}-\vec{Q}
\end{align}
resulting in the equation
\begin{align}
    \brak{\vec{P}-\vec{Q}}^\top\brak{\vec{x}-\frac{\vec{P}+\vec{Q}}{2} } = 0
    \implies \brak{\vec{P}-\vec{Q}}^\top\vec{x} = 0
\end{align}
%
after simplification.  It is obvious that $\vec{O}$ satisfies the above equuation as can be verified in Fig.     \ref{constr/circ/61/fig: perpenicular bisector of the chord passes through the center}.
%
\begin{figure}[ht]
    \centering
    \includegraphics[width=\columnwidth]{solutions/circle/61/FIG.3.png}
    \caption{perpendicular bisector of the chord passes through the center}
    \label{constr/circ/61/fig: perpenicular bisector of the chord passes through the center}
\end{figure}








\item A factory manufactures two types of screws, A and B. Each type of screw
requires the use of two machines, an automatic and a hand operated. It takes
4 minutes on the automatic and 6 minutes on hand operated machines to
manufacture a package of screws A, while it takes 6 minutes on automatic and
3 minutes on the hand operated machines to manufacture a package of screws
B. Each machine is available for at the most 4 hours on any day. The manufacturer
can sell a package of screws A at a profit of Rs 7 and screws B at a profit of
Rs 10. Assuming that he can sell all the screws he manufactures, how many
packages of each type should the factory owner produce in a day in order to
maximise his profit? Determine the maximum profit.\\
\solution

\begin{table}[!ht]
\centering
\resizebox{\columnwidth}{!}{\begin{tabular}{|c|c|c|c|c|} 
\hline
Item & Number & Machine A & Machine B & Profit \\
\hline
Screw A & x & 4 minutes & 6 minutes & Rs 7 \\
\hline
Screw B & y & 6 minutes & 3 minutes & Rs 10 
\\
\hline
Max Available Time &  & 4hours  =240minutes & 4hours =240minutes &
\\
\hline
\end{tabular}}
\caption{Screw Requirements}
\label{opt/16/tab:table1}
\end{table}
Let the number of packages of screw  A be $x$ and the number of packages of screw B be $y$  such that 
\begin{align}
    x \geq 0 \\
    y \geq 0 
\end{align}
According to the question,
\begin{align}
    4x+6y &\leq 240 \\
 \implies 2x+3y &\leq 120
\end{align}
     and,
\begin{align}
    6x+3y &\leq 240 \\
 \implies 2x+y &\leq 80
\end{align}
$\therefore$ Our problem is
\begin{align}
        \max_{\vec{x}} Z &= \myvec{7 & 10}\vec{x}\\
        s.t. \quad 
        \myvec{2 & 3 \\ 2 & 1 }\vec{x} &\preceq \myvec{120\\80} 
\end{align}
Lagrangian function is given by
\begin{equation}
\begin{aligned}
    &L(\vec{x},\boldsymbol{\lambda}) \\ &= \myvec{7 & 10}\vec{x}+\lcbrak{\sbrak{\myvec{2 & 3}\vec{x}+120}} \\ &+ \sbrak{\myvec{2 & 1}\vec{x}+80} \\ &+ \sbrak{\myvec{-1 & 0}\vec{x}} +\rcbrak{\sbrak{\myvec{0 & -1}\vec{x}}}\boldsymbol{\lambda}
\end{aligned}
\end{equation}
where,
\begin{align}
    \boldsymbol{\lambda} &= \myvec{\lambda_1 \\ \lambda_2 \\ \lambda_3 \\ \lambda_4 \\ \lambda_5 \\ \lambda_6}
\end{align}
Now,
\begin{align}
    \nabla L(\vec{x},\boldsymbol{\lambda}) &= \myvec{7+ \myvec{2 & 3 & -1 & 0 }\boldsymbol{\lambda}\\ 10+\myvec{2 & 1 & 0 & -1}\boldsymbol{\lambda} \\ \myvec{2 & 3}\vec{x}+120 \\ \myvec{2 & 1}\vec{x}+80 \\ \myvec{-1 & 0}\vec{x} \\ \myvec{0 & -1}\vec{x}}
\end{align}
$\therefore$ Lagrangian matrix is given by
\begin{align}
    \myvec{0 & 0 & 2 & 3 & -1 & 0 \\ 0 & 0 & 2 & 1 & 0 & -1 \\ 2 & 3 & 0 & 0 & 0 & 0  \\ 2 & 1 & 0 & 0 & 0 & 0  \\ -1 & 0 & 0 & 0 & 0 & 0  \\ 0 & -1 & 0 & 0 & 0 & 0 }\myvec{\vec{x} \\ \boldsymbol{\lambda} } &= \myvec{-5 \\ -6 \\ 200 \\ 240 \\ 0 \\0 }
\end{align}
Considering $\lambda_1,\lambda_2$ as only active multiplier,
\begin{align}
    \myvec{0 & 0 & 2 & 3 \\ 0 & 0 & 2 & 1 \\ 2 & 3 & 0 & 0 \\ 2 & 1 & 0 & 0}\myvec{\vec{x}\\ \boldsymbol{\lambda}} &=\myvec{-7 \\ -10 \\ 120 \\ 80}
\end{align}
resulting in,
\begin{align}
    \myvec{\vec{x} \\ \boldsymbol{\lambda}} &= \myvec{0 & 0 & 2 & 3 \\ 0 & 0 & 2 & 1 \\ 2 & 3 & 0 & 0 \\ 2 & 1 & 0 & 0}^{-1}\myvec{-7 \\ -10 \\ 120 \\ 80}
    \\
    \implies   \myvec{\vec{x} \\ \boldsymbol{\lambda}} &= \myvec{0 & 0 & \frac{-1}{4} & \frac{3}{4} \\ 0 & 0 & \frac{1}{2} & \frac{-1}{2} \\ \frac{-1}{4} & \frac{1}{2} & 0 & 0 \\ \frac{1}{2} & \frac{-1}{2} & 0 & 0}\myvec{-7 \\ -10 \\ 120 \\ 80}
    \\
    \implies \myvec{\vec{x} \\ \boldsymbol{\lambda}} &= \myvec{30 \\ 20 \\ \frac{-13}{4} \\ \frac{3}{2} }
\end{align}
$\because \boldsymbol{\lambda}=\myvec{\frac{-13}{4} \\ \frac{3}{2}} \succ \vec{0} $ 
\\
$\therefore$ Optimal solution is given by
\begin{align}
    \vec{x} &= \myvec{30\\20} \\
    Z &= \myvec{7 & 10}\vec{x} \\
    &= \myvec{7 & 10}\myvec{30 \\ 20} \\
    &= 410
\end{align}
By using cvxpy in python ,
\begin{align}
    \vec{x}=\myvec{30.00000000\\20.00000000}\\
    Z = 410.00000000
\end{align}
Hence ,\boxed{x=30} packages of screw A and \boxed{y=20} packages of screw B should be the factory owner produce in a day in order to maximise his  profit is \boxed{Z=410}.  This
is verified in Fig. \ref{opt/16/fig: graphical solution}.	
%
\begin{figure}[!ht]
\centering
\includegraphics[width=\columnwidth]{solutions/su2021/2/16/Figure 11.png}
\caption{graphical solution}
\label{opt/16/fig: graphical solution}	
\end{figure}


\item A cottage industry manufactures pedestal lamps and wooden shades, each
requiring the use of a grinding/cutting machine and a sprayer. It takes 2 hours on
grinding/cutting machine and 3 hours on the sprayer to manufacture a pedestal
lamp. It takes 1 hour on the grinding/cutting machine and 2 hours on the sprayer
to manufacture a shade. On any day, the sprayer is available for at the most 20
hours and the grinding/cutting machine for at the most 12 hours. The profit from
the sale of a lamp is Rs 5 and that from a shade is Rs 3. Assuming that the
manufacturer can sell all the lamps and shades that he produces, how should he
schedule his daily production in order to maximise his profit?\\
\solution
Data from the given question is available in Table \ref{constr/circ/61/tab:table1}:
%
\begin{table}[!ht]
\begin{center}
\begin{tabular}{ | m{2cm} | m{1.5cm}| m{2cm} | m{1.5cm} |} 
\hline
& Symbols & Circle1  \\
\hline
Centre & $\vec{C}$ & \myvec{0\\0}  \\ 
\hline
Radius & $r$& 3.4 \\ 
\hline
\end{tabular}
\end{center}
\caption{Input values}
\label{constr/circ/61/tab:table1}
\end{table}
%
Let 
\begin{align}
      \vec{P} &= r \myvec{\cos \theta_1\\ \sin \theta_1} =  \myvec{1.7\\2.9}
      \\
      \vec{Q} &= r \myvec{\cos \theta_2\\ \sin \theta_2} =\myvec{-2.4\\2.4}
 \end{align}
 Then the perpendicular bisector of $PQ$ passes through 
 \begin{align}
    \vec{M} = \frac{\vec{P}+\vec{Q}}{2} 
\end{align}
and has normal vector 
\begin{align}
    \vec{n} = \vec{P}-\vec{Q}
\end{align}
resulting in the equation
\begin{align}
    \brak{\vec{P}-\vec{Q}}^\top\brak{\vec{x}-\frac{\vec{P}+\vec{Q}}{2} } = 0
    \implies \brak{\vec{P}-\vec{Q}}^\top\vec{x} = 0
\end{align}
%
after simplification.  It is obvious that $\vec{O}$ satisfies the above equuation as can be verified in Fig.     \ref{constr/circ/61/fig: perpenicular bisector of the chord passes through the center}.
%
\begin{figure}[ht]
    \centering
    \includegraphics[width=\columnwidth]{solutions/circle/61/FIG.3.png}
    \caption{perpendicular bisector of the chord passes through the center}
    \label{constr/circ/61/fig: perpenicular bisector of the chord passes through the center}
\end{figure}








\item A company manufactures two types of novelty souvenirs made of plywood.
Souvenirs of type A require 5 minutes each for cutting and 10 minutes each for
assembling. Souvenirs of type B require 8 minutes each for cutting and 8 minutes
each for assembling. There are 3 hours 20 minutes available for cutting and 4
hours for assembling. The profit is Rs 5 each for type A and Rs 6 each for type
B souvenirs. How many souvenirs of each type should the company manufacture
in order to maximise the profit?\\
\solution
Data from the given question is available in Table \ref{constr/circ/61/tab:table1}:
%
\begin{table}[!ht]
\begin{center}
\begin{tabular}{ | m{2cm} | m{1.5cm}| m{2cm} | m{1.5cm} |} 
\hline
& Symbols & Circle1  \\
\hline
Centre & $\vec{C}$ & \myvec{0\\0}  \\ 
\hline
Radius & $r$& 3.4 \\ 
\hline
\end{tabular}
\end{center}
\caption{Input values}
\label{constr/circ/61/tab:table1}
\end{table}
%
Let 
\begin{align}
      \vec{P} &= r \myvec{\cos \theta_1\\ \sin \theta_1} =  \myvec{1.7\\2.9}
      \\
      \vec{Q} &= r \myvec{\cos \theta_2\\ \sin \theta_2} =\myvec{-2.4\\2.4}
 \end{align}
 Then the perpendicular bisector of $PQ$ passes through 
 \begin{align}
    \vec{M} = \frac{\vec{P}+\vec{Q}}{2} 
\end{align}
and has normal vector 
\begin{align}
    \vec{n} = \vec{P}-\vec{Q}
\end{align}
resulting in the equation
\begin{align}
    \brak{\vec{P}-\vec{Q}}^\top\brak{\vec{x}-\frac{\vec{P}+\vec{Q}}{2} } = 0
    \implies \brak{\vec{P}-\vec{Q}}^\top\vec{x} = 0
\end{align}
%
after simplification.  It is obvious that $\vec{O}$ satisfies the above equuation as can be verified in Fig.     \ref{constr/circ/61/fig: perpenicular bisector of the chord passes through the center}.
%
\begin{figure}[ht]
    \centering
    \includegraphics[width=\columnwidth]{solutions/circle/61/FIG.3.png}
    \caption{perpendicular bisector of the chord passes through the center}
    \label{constr/circ/61/fig: perpenicular bisector of the chord passes through the center}
\end{figure}







\item A manufacturer makes two types of toys A and B. Three machines are needed
for this purpose and the time (in minutes) required for each toy on the machines
is given below:\\
\begin{table}[!ht]
    \begin{center}
    \begin{tabular}{|l|l|l|l|} \hline
    \multicolumn{4}{|c|} {Machines} \\ \hline
    Types of toys & I & II & III \\ \hline
    A & 12 & 18 & 6\\ \hline
    B & 6 & 0 & 9\\ \hline
    \end{tabular}
    \end{center}
    \caption{Toys table}
    \label{opt/26/opt/26/tab:table1}
    \end{table}


Each machine is available for a maximum of 6 hours per day. If the profit on
each toy of type A is Rs 7.50 and that on each toy of type B is Rs 5, show that 15
toys of type A and 30 of type B should be manufactured in a day to get maximum
profit.\\
\\
\solution 
Data from the given question is available in Table \ref{constr/circ/61/tab:table1}:
%
\begin{table}[!ht]
\begin{center}
\begin{tabular}{ | m{2cm} | m{1.5cm}| m{2cm} | m{1.5cm} |} 
\hline
& Symbols & Circle1  \\
\hline
Centre & $\vec{C}$ & \myvec{0\\0}  \\ 
\hline
Radius & $r$& 3.4 \\ 
\hline
\end{tabular}
\end{center}
\caption{Input values}
\label{constr/circ/61/tab:table1}
\end{table}
%
Let 
\begin{align}
      \vec{P} &= r \myvec{\cos \theta_1\\ \sin \theta_1} =  \myvec{1.7\\2.9}
      \\
      \vec{Q} &= r \myvec{\cos \theta_2\\ \sin \theta_2} =\myvec{-2.4\\2.4}
 \end{align}
 Then the perpendicular bisector of $PQ$ passes through 
 \begin{align}
    \vec{M} = \frac{\vec{P}+\vec{Q}}{2} 
\end{align}
and has normal vector 
\begin{align}
    \vec{n} = \vec{P}-\vec{Q}
\end{align}
resulting in the equation
\begin{align}
    \brak{\vec{P}-\vec{Q}}^\top\brak{\vec{x}-\frac{\vec{P}+\vec{Q}}{2} } = 0
    \implies \brak{\vec{P}-\vec{Q}}^\top\vec{x} = 0
\end{align}
%
after simplification.  It is obvious that $\vec{O}$ satisfies the above equuation as can be verified in Fig.     \ref{constr/circ/61/fig: perpenicular bisector of the chord passes through the center}.
%
\begin{figure}[ht]
    \centering
    \includegraphics[width=\columnwidth]{solutions/circle/61/FIG.3.png}
    \caption{perpendicular bisector of the chord passes through the center}
    \label{constr/circ/61/fig: perpenicular bisector of the chord passes through the center}
\end{figure}








\item Two godowns A and B have grain capacity of 100 quintals and 50 quintals respectively.They supply to 3 ration shops, D, E and F whose requirements are 60, 50 and 40 quintals respectively. The cost of transportation per quintal from the godowns to the shops are given in the following table:
\numberwithin{table}{section}
\begin{table}[!ht]
\begin{center}
\begin{tabular}{ |l|l|l|}
\hline
\multicolumn{3}{ |c| }{Transportation cost per quintal (in rupees)} \\
\hline
From/To & A & B \\ \hline
D & 6 & 4  \\ \hline
E & 3 & 2 \\ \hline
F & 2.50 & 3 \\ \hline
\end{tabular}
\end{center}
\caption{Transportation table}
\label{opt/28/tab:table1}
\end{table}
How should the supplies be transported in order that the transportation cost is minimum? What is the minimum cost?
\\
\solution 
\input{solutions/su2021/2/28/Assignment-12.tex}
\item A fruit grower can use two types of fertilizer in his garden, brand P and brand Q.The amounts(in kg) of nitrogen, phosphoric acid,potash and chlorine in a bag of each brand are given in the table.Tests indicate that garden needs atleast 240 kg of phosphoric acid,atleast 270 kg of potash and atmost 310 kg of chlorine. If the grower wants to minimise the amount of nitrogen added to garden, how many bags of each brand should be used?What is the minimum amount of nitrogen added in the ground?

\begin{table}[!ht]
\centering
\resizebox{\columnwidth}{!}{\begin{tabular}{|c|c|c|} 
\hline
 & \textbf{Brand P }& \textbf{Brand Q }\\
\hline
Nitrogen & 3  & 3.5  \\ 
\hline
Phosphoric Acid & 1 & 2 \\ 
\hline
Potash & 3 & 1.5\\ 
\hline
Chlorine & 1.5 & 2 \\ 
\hline
\end{tabular}}
\caption{kg per bag}
\label{opt/30/tab:table1}
\end{table}
%
\solution
\begin{itemize}
\item All the data can be tabularised as:
\numberwithin{table}{section}
\begin{table}[!ht]
\centering
\resizebox{\columnwidth}{!}{\begin{tabular}{|c|c|c|c|} 
\hline
 & \textbf{Brand P }& \textbf{Brand Q } &Amounts Required\\
\hline
Nitrogen & 3  & 3.5 & $?$ \\ 
\hline
Phosphoric Acid & 1 & 2 &$\geq 240$ kg \\ 
\hline
Potash & 3 & 1.5 &$\geq 270$ kg\\ 
\hline
Chlorine & 1.5 & 2 &$\leq 310$ kg\\ 
\hline
\end{tabular}}
\caption{Requirements of fertilizers}
\label{opt/30/tab:table2}
\end{table}
\item Let the number of bags of Brand P be $x$ $\And$
\item The number of bags of Brand Q be $y$ such that : 
\begin{align}
    x \geq 0 
    \\
    y \geq 0 
\end{align}
\item From the data given we have:
\begin{align}
    x+2y &\geq 240 \\
   \implies   -x-2y &\leq -240
\end{align}
and,
\begin{align}
    3x+1.5y &\geq 270 \\
    \implies -x-0.5y &\leq -90
\end{align}
and,
\begin{align}
     1.5x+2y &\leq 310 \\
\end{align}
$\therefore$ The minimizing function is:
\begin{align}
        \min_{\vec{x}} Z &= \myvec{3& 3.5}\vec{x}\\
        s.t. \quad 
        \myvec{-1 & -2\\ -1 & -0.5 \\ 1.5 & 2 }\vec{x} &\preceq \myvec{-240\\-90\\310} \\
        \vec{-x} &\preceq \vec{0}
\end{align}
\item The Lagrangian function can be given as:
\begin{equation}
\begin{aligned}
    &L(\vec{x},\boldsymbol{\lambda}) \\ &= \myvec{3 & 3.5}\vec{x}+\lcbrak{\sbrak{\myvec{-1 & -2}\vec{x}+240}} \\ &+ \sbrak{\myvec{-1 & -0.5}\vec{x}+90} +\sbrak{\myvec{1.5 & 2}\vec{x}-310} \\ &+ \sbrak{\myvec{-1 & 0}\vec{x}} +\rcbrak{\sbrak{\myvec{0 & -1}\vec{x}}}\boldsymbol{\lambda}
\end{aligned}
\end{equation}
where,
\begin{align}
    \boldsymbol{\lambda} &= \myvec{\lambda_1 \\ \lambda_2 \\ \lambda_3 \\ \lambda_4 \\ \lambda_5 \\ \lambda_6}
\end{align}
\item Now, we have
\begin{align}
    \nabla L(\vec{x},\boldsymbol{\lambda}) &= \myvec{3+ \myvec{-1 & -1 & 1.5 & -1 & 0 }\boldsymbol{\lambda}\\ 3.5+\myvec{-2 & -0.5 & 2 & 0 & -1}\boldsymbol{\lambda} \\ \myvec{-1 & -2}\vec{x}+240 \\ \myvec{-1 & -0.5}\vec{x}+90 \\ \myvec{1.5 & 2}\vec{x}-310 \\ \myvec{-1 & 0}\vec{x} \\ \myvec{0 & -1}\vec{x}}
\end{align}
$\therefore$ The Lagrangian matrix is given by:-
\begin{align}
  \small{\myvec{0 & 0 & -1 & -1 & 1.5 & -1 & 0 \\ 0 & 0 & -2 & -0.5 & 2 & 0 & -1 \\ -1 & -2 & 0 & 0 & 0 & 0 & 0 \\ -1 & -0.5 & 0 & 0 & 0 & 0 & 0 \\ 1.5 & 2 & 0 & 0 & 0 & 0 & 0 \\ -1 & 0 & 0 & 0 & 0 & 0 & 0 \\ 0 & -1 & 0 & 0 & 0 & 0 & 0 }\myvec{\vec{x} \\ \boldsymbol{\lambda} }}= \small{\myvec{-3 \\ -3.5 \\ -240 \\ -90 \\ 310 \\ 0 \\0 }}
\end{align}
\item Considering $\lambda_1,\lambda_2$ as only active multiplier,
\begin{align}
    \myvec{0 & 0 & -1 & -1  \\ 0 & 0 & -2 & -0.5 \\ -1 & -2 & 0 & 0 \\-1 & -0.5 & 0 & 0}\myvec{\vec{x}\\ \boldsymbol{\lambda}} &= \myvec{-3 \\ -3.5 \\ -240 \\ -90}
\end{align}
\begin{align}
 \implies   \myvec{\vec{x} \\ \boldsymbol{\lambda}} &=  \myvec{0 & 0 & -1 & -1  \\ 0 & 0 & -2 & -0.5 \\ -1 & -2 & 0 & 0 \\-1 & -0.5 & 0 & 0}^{-1}\myvec{-3 \\ -3.5 \\ -240 \\ -90}
    \\
    \implies   \myvec{\vec{x} \\ \boldsymbol{\lambda}} &= \myvec{0 & 0 & \frac{1}{3} & \frac{4}{3} \\ 0 & 0 & \frac{-2}{3} & \frac{2}{3} \\ \frac{1}{3} & \frac{-2}{3} & 0 & 0 \\ \frac{-4}{3} & \frac{2}{3} & 0 & 0}\myvec{-3 \\ -3.5 \\ -240 \\ -90}
    \\
    \implies \myvec{\vec{x} \\ \boldsymbol{\lambda}} &= \myvec{40 \\100 \\ \frac{4}{3} \\ \frac{5}{3} }
\end{align}
$\because \boldsymbol{\lambda}=\myvec{\frac{4}{3} \\ \frac{5}{3}} \succ \vec{0}$
\\
\item The Optimal solution is given by:
\begin{align}
    \vec{x} &= \myvec{40\\100} \\
    Z &= \myvec{3&3.5}\vec{x} \\
   Z &= \myvec{3&3.5}\myvec{40 \\ 100} \\
    Z&= 470 \text{ units}
\end{align}
\item So, we get
\\
 Bags of brand \textbf{P} as \boxed{x=40} $\And$
 \\
 Bags of brand \textbf{Q} as  \boxed{y=100} so as to minimise the amount of nitrogen added.
\item The minimum amount of nitrogen required is \boxed{Z=470 \text{ units}} .  Fig. \ref{opt/30/fig}
verifies this result.
%
\begin{figure}[!ht]
\centering
\includegraphics[width=\columnwidth]{solutions/su2021/2/30/Graphical_Solution_2.30.png}
\caption{Graphical Solution}
\label{opt/30/fig}
\end{figure}

\end{itemize}


\item A toy company manufactures two types of dolls, A and B. Market research and
available resources have indicated that the combined production level should not
exceed 1200 dolls per week and the demand for dolls of type B is at most half of that
for dolls of type A. Further, the production level of dolls of type A can exceed three
times the production of dolls of other type by at most 600 units. If the company
makes profit of Rs 12 and Rs 16 per doll respectively on dolls A and B, how many of
each should be produced weekly in order to maximise the profit?
%
\solution
Data from the given question is available in Table \ref{constr/circ/61/tab:table1}:
%
\begin{table}[!ht]
\begin{center}
\begin{tabular}{ | m{2cm} | m{1.5cm}| m{2cm} | m{1.5cm} |} 
\hline
& Symbols & Circle1  \\
\hline
Centre & $\vec{C}$ & \myvec{0\\0}  \\ 
\hline
Radius & $r$& 3.4 \\ 
\hline
\end{tabular}
\end{center}
\caption{Input values}
\label{constr/circ/61/tab:table1}
\end{table}
%
Let 
\begin{align}
      \vec{P} &= r \myvec{\cos \theta_1\\ \sin \theta_1} =  \myvec{1.7\\2.9}
      \\
      \vec{Q} &= r \myvec{\cos \theta_2\\ \sin \theta_2} =\myvec{-2.4\\2.4}
 \end{align}
 Then the perpendicular bisector of $PQ$ passes through 
 \begin{align}
    \vec{M} = \frac{\vec{P}+\vec{Q}}{2} 
\end{align}
and has normal vector 
\begin{align}
    \vec{n} = \vec{P}-\vec{Q}
\end{align}
resulting in the equation
\begin{align}
    \brak{\vec{P}-\vec{Q}}^\top\brak{\vec{x}-\frac{\vec{P}+\vec{Q}}{2} } = 0
    \implies \brak{\vec{P}-\vec{Q}}^\top\vec{x} = 0
\end{align}
%
after simplification.  It is obvious that $\vec{O}$ satisfies the above equuation as can be verified in Fig.     \ref{constr/circ/61/fig: perpenicular bisector of the chord passes through the center}.
%
\begin{figure}[ht]
    \centering
    \includegraphics[width=\columnwidth]{solutions/circle/61/FIG.3.png}
    \caption{perpendicular bisector of the chord passes through the center}
    \label{constr/circ/61/fig: perpenicular bisector of the chord passes through the center}
\end{figure}








\item  \textbf{(Manufacturing problem)} A manufacturing company makes two models
A and B of a product. Each piece of Model A requires 9 labour hours for fabricating
and 1 labour hour for finishing. Each piece of Model B requires 12 labour hours for
fabricating and 3 labour hours for finishing. For fabricating and finishing, the maximum
labour hours available are 180 and 30 respectively. The company makes a profit of
Rs 8000 on each piece of model A and Rs 12000 on each piece of Model B. How many
pieces of Model A and Model B should be manufactured per week to realise a maximum
profit? What is the maximum profit per week?\\
\solution
\begin{itemize}
\item All the data can be tabularised as:
%
\begin{table}[!ht]
\centering
\resizebox{\columnwidth}{!}{\begin{tabular}{|c|c|c|c|} 
\hline
 & Fabricating&Finishing &Profit\\
\hline
\textbf{Model A} & 9  & 1 & 8000 \\ 
\hline
\textbf{Model B} & 12  & 3 & 12000 \\ 
\hline
\text{Max Hours} & $\leq180$&$\leq30$&\\
\hline
\end{tabular}}
\caption{Labour Hours and Profit for each piece}
\label{opt/2/35/tab:table1}
\end{table}
\item Let the number of pieces of model A manufactured be $x$ and
the number of pieces of model B manufactured be $y$ such that : 
\begin{align}
    x \geq 0 
    \\
    y \geq 0 
\end{align}
\item From the data given we have:
\begin{align}
    9x+12y &\leq 180 \\
    \implies 3x+4y&\leq60
\end{align}
and,
\begin{align}
    x+3y &\leq 30 
\end{align}

$\therefore$ The maximizing function is:
\begin{align}
        \max Z &= \myvec{8000& 12000}\vec{x}\\
        s.t. \quad 
        \myvec{3 & 4\\ 1 & 3 }\vec{x} &\preceq \myvec{60\\30} \\
        \vec{-x} &\preceq \vec{0}
\end{align}
\item The Lagrangian function can be given as:
\begin{equation}
\begin{aligned}
    &L(\vec{x},\boldsymbol{\lambda}) \\ &= \myvec{8000 & 12000}\vec{x}+\lcbrak{\sbrak{\myvec{3 & 4}\vec{x}-60}} \\ &+ \sbrak{\myvec{1 & 3}\vec{x}-30} \\ &+ \sbrak{\myvec{-1 & 0}\vec{x}} +\rcbrak{\sbrak{\myvec{0 & -1}\vec{x}}}\boldsymbol{\lambda}
\end{aligned}
\end{equation}
where,
\begin{align}
    \boldsymbol{\lambda} &= \myvec{\lambda_1 \\ \lambda_2 \\ \lambda_3 \\ \lambda_4}
\end{align}
\item Now, we have
\begin{align}
    \nabla L(\vec{x},\boldsymbol{\lambda}) &= \myvec{8000+ \myvec{3 & 1 & -1 & 0 }\boldsymbol{\lambda}\\ 12000+\myvec{4 & 3 &0 & -1}\boldsymbol{\lambda} \\ \myvec{3 & 4}\vec{x}-60 \\ \myvec{1 & 3}\vec{x}-30 \\ \myvec{-1 & 0}\vec{x} \\ \myvec{0 & -1}\vec{x}}
\end{align}
$\therefore$ The Lagrangian matrix is given by:-
\begin{align}
  \small{\myvec{0 & 0 & 3 & 1 & -1 & 0 \\ 0 & 0 & 4 & 3 &0 & -1 \\ 3 & 4 & 0 & 0 & 0 & 0 \\ 1 & 3 & 0 & 0 & 0 & 0 \\ -1 & 0 & 0 & 0 & 0 & 0 \\ 0 & -1 & 0 & 0 & 0 & 0 }\myvec{\vec{x} \\ \boldsymbol{\lambda} }}= \small{\myvec{-8000 \\ -12000 \\ 60 \\ 30 \\ 0 \\0 }}
\end{align}
\item Considering $\lambda_1,\lambda_2$ as only active multiplier,
\begin{align}
    \myvec{0 & 0 & 3 & 1  \\ 0 & 0 & 4 & 3 \\ 3 & 4 & 0 & 0 \\1 & 3 & 0 & 0}\myvec{\vec{x}\\ \boldsymbol{\lambda}} &= \myvec{-8000 \\ -12000 \\ 60 \\ 30}
\end{align}
\begin{align}
 \implies   \myvec{\vec{x} \\ \boldsymbol{\lambda}} &=  \myvec{0 & 0 & 3 & 1  \\ 0 & 0 & 4 & 3 \\ 3 & 4 & 0 & 0 \\1 & 3 & 0 & 0} ^{-1}\myvec{-8000 \\ -12000 \\ 60 \\ 30}
    \\
    \implies   \myvec{\vec{x} \\ \boldsymbol{\lambda}} &= \myvec{0 & 0 & \frac{3}{5} & \frac{-4}{5} \\ 0 & 0 & \frac{-1}{5} & \frac{3}{5} \\ \frac{3}{5} & \frac{-1}{5} & 0 & 0 \\ \frac{-4}{5} & \frac{3}{5} & 0 & 0}\myvec{-8000 \\ -12000 \\ 60 \\ 30}
    \\
    \implies \myvec{\vec{x} \\ \boldsymbol{\lambda}} &= \myvec{12 \\6 \\ -2400 \\ -800 }
\end{align}
$\because \boldsymbol{\lambda}=\myvec{-2400 \\ -800} \prec \vec{0}$
\\
\item The Optimal solution is given by:
\begin{align}
    \vec{x} &= \myvec{12\\6} \\
    Z &= \myvec{8000&12000}\vec{x} \\
   Z &= \myvec{8000&12000}\myvec{12 \\ 6} \\
    Z&= \text{Rs} 168000
\end{align}
\item So, to maximise profit
\\
 Pieces of model \textbf{A} manufactured is \boxed{x=12} and
 \\
  Pieces of model \textbf{B} manufactured is \boxed{y=6}.
\item The maximum profit per week is \boxed{Z=\text{Rs} 168000} .

\begin{figure}[!ht]
\centering
\includegraphics[width=\columnwidth]{solutions/su2021/2/35/Figure 10_1.png}
\caption{Graphical Representataion}
\end{figure}
\end{itemize}


\item \textbf {(Diet problem)} A dietician has to develop a special diet using two foods
P and Q. Each packet (containing 30 g) of food P contains 12 units of calcium, 4 units
of iron, 6 units of cholesterol and 6 units of vitamin A. Each packet of the same quantity
of food Q contains 3 units of calcium, 20 units of iron, 4 units of cholesterol and 3 units
of vitamin A. The diet requires atleast 240 units of calcium, atleast 460 units of iron and
at most 300 units of cholesterol. How many packets of each food should be used to
minimise the amount of vitamin A in the diet? What is the minimum amount of vitamin A?\\
\\
\solution
\begin{table}[!ht]
    \centering
    \resizebox{\columnwidth}{!}{\begin{tabular}{|c|c|c|c|} 
    \hline
    Component & P & Q & Requirement \\
    \hline
    Calcium & 12 units & 3 units & $\geq 240$ units \\ 
    \hline
    Iron & 4 units & 20 units & $\geq 460$ units \\ 
    \hline
    Cholesterol & 6 units & 4 units & $\leq 300$ units\\ 
    \hline
    Vitamin A & 6 units & 3 units & \\ 
    \hline
    \end{tabular}}
    \caption{Diet Requirements}
    \label{opt/36/tab:table1}
    \end{table}
    Let the number of packets of food P be $x$ and the number of packets of food Q be $y$  such that 
    \begin{align}
        x \geq 0 \\
        y \geq 0 
    \end{align}
    According to the question,
    \begin{align}
        12x+3y &\geq 240 \\
        \implies -4x-y &\leq -80 \\
    \end{align}
    and,
    \begin{align}
        4x+20y &\geq 460 \\
        \implies -x-5y &\leq -115 
    \end{align}
    and,
    \begin{align}
         6x+4y &\leq 300 \\
        \implies 3x+2y &\leq 150 
    \end{align}
    $\therefore$ Our problem is
    \begin{align}
            \min_{\vec{x}} Z &= \myvec{6 & 3}\vec{x}\\
            s.t. \quad 
            \myvec{-4 & -1 \\ -1 & -5 \\ 3 & 2 }\vec{x} &\preceq \myvec{-80\\-115\\150} \\
            \vec{-x} &\preceq \vec{0}
    \end{align}
    Lagrangian function is given by
    \begin{equation}
    \begin{aligned}
        &L(\vec{x},\boldsymbol{\lambda}) \\ &= \myvec{6 & 3}\vec{x}+\lcbrak{\sbrak{\myvec{-4 & -1}\vec{x}+80}} \\ &+ \sbrak{\myvec{-1 & -5}\vec{x}+115} +\sbrak{\myvec{3 & 2}\vec{x}-150} \\ &+ \sbrak{\myvec{-1 & 0}\vec{x}} +\rcbrak{\sbrak{\myvec{0 & -1}\vec{x}}}\boldsymbol{\lambda}
    \end{aligned}
    \end{equation}
    where,
    \begin{align}
        \boldsymbol{\lambda} &= \myvec{\lambda_1 \\ \lambda_2 \\ \lambda_3 \\ \lambda_4 \\ \lambda_5 \\ \lambda_6}
    \end{align}
    Now,
    \begin{align}
        \nabla L(\vec{x},\boldsymbol{\lambda}) &= \myvec{6+ \myvec{-4 & -1 & 3 & -1 & 0 }\boldsymbol{\lambda}\\ 3+\myvec{-1 & -5 & 2 & 0 & -1}\boldsymbol{\lambda} \\ \myvec{-4 & -1}\vec{x}+80 \\ \myvec{-1 & -5}\vec{x}+115 \\ \myvec{3 & 2}\vec{x}-150 \\ \myvec{-1 & 0}\vec{x} \\ \myvec{0 & -1}\vec{x}}
    \end{align}
    $\therefore$ Lagrangian matrix is given by
    \begin{align}
        \myvec{0 & 0 & -4 & -1 & 3 & -1 & 0 \\ 0 & 0 & -1 & -5 & 2 & 0 & -1 \\ -4 & -1 & 0 & 0 & 0 & 0 & 0 \\ -1 & -5 & 0 & 0 & 0 & 0 & 0 \\ 3 & 2 & 0 & 0 & 0 & 0 & 0 \\ -1 & 0 & 0 & 0 & 0 & 0 & 0 \\ 0 & -1 & 0 & 0 & 0 & 0 & 0 }\myvec{\vec{x} \\ \boldsymbol{\lambda} } &= \myvec{-6 \\ -3 \\ -80 \\ -115 \\ 150 \\ 0 \\0 }
    \end{align}
    Considering $\lambda_1,\lambda_2$ as only active multiplier,
    \begin{align}
        \myvec{0 & 0 & -4 & -1 \\ 0 & 0 & -1 & -5 \\ -4 & -1 & 0 & 0 \\ -1 & -5 & 0 & 0}\myvec{\vec{x}\\ \boldsymbol{\lambda}} &= \myvec{-6 \\ -3 \\ -80 \\ -115}
    \end{align}
    resulting in,
    \begin{align}
        \myvec{\vec{x} \\ \boldsymbol{\lambda}} &= \myvec{0 & 0 & -4 & -1 \\ 0 & 0 & -1 & -5 \\ -4 & -1 & 0 & 0 \\ -1 & -5 & 0 & 0}^{-1}\myvec{-6 \\ -3 \\ -80 \\ -115}
        \\
        \implies   \myvec{\vec{x} \\ \boldsymbol{\lambda}} &= \myvec{0 & 0 & \frac{-5}{19} & \frac{1}{19} \\ 0 & 0 & \frac{1}{19} & \frac{-4}{19} \\ \frac{-5}{19} & \frac{1}{19} & 0 & 0 \\ \frac{1}{19} & \frac{-4}{19} & 0 & 0}\myvec{-6 \\ -3 \\ -80 \\ -115}
        \\
        \implies \myvec{\vec{x} \\ \boldsymbol{\lambda}} &= \myvec{15 \\ 20 \\ \frac{27}{19} \\ \frac{6}{19} }
    \end{align}
    $\because \boldsymbol{\lambda}=\myvec{\frac{27}{19} \\ \frac{6}{19}} \succ \vec{0} $
    \\
    $\therefore$ Optimal solution is given by
    \begin{align}
        \vec{x} &= \myvec{15\\20} \\
        Z &= \myvec{6 & 3}\vec{x} \\
        &= \myvec{6 & 3}\myvec{15 \\ 20} \\
        &= 150
    \end{align}
    By using cvxpy in python ,
    \begin{align}
        \vec{x}=\myvec{14.99999999\\20.00000001}\\
        Z = 150.00000001
    \end{align}
    Hence ,\boxed{x=15} packets of food P and \boxed{y=20} packets of food Q should be used to minimise the amount of vitamin A in the diet and the minimum amount of vitamin A is \boxed{Z=150} units.  This is verified in Fig. \ref{opt/36/fig:diet problem}	
    %
    \begin{figure}[!ht]
    \centering
    \includegraphics[width=\columnwidth]{solutions/su2021/2/36/Figure12.png}
    \caption{Diet Problem}
    \label{opt/36/fig:diet problem}	
    \end{figure}

%\end{enumerate}
%\end{document}
