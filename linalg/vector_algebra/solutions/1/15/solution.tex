
\begin{figure}[!ht]
\centering
\resizebox{\columnwidth}{!}{\tikzset{every picture/.style={line width=0.75pt}}         

\begin{tikzpicture}[x=0.75pt,y=0.75pt,yscale=-1,xscale=1]


\draw   (338.25,31.17) -- (576.5,288.17) -- (100,288.17) -- cycle ;

\draw    (338.25,31.17) -- (337.5,287.17) ;

\draw   (337.5,272.17) -- (352.5,272.17) -- (352.5,287.17) ;


\draw (330,5) node [anchor=north west][inner sep=0.75pt]   [align=left] {{\Large A}};

\draw (81,280) node [anchor=north west][inner sep=0.75pt]   [align=left] {{\Large B}};

\draw (584,279) node [anchor=north west][inner sep=0.75pt]   [align=left] {{\Large C}};

\draw (332,292) node [anchor=north west][inner sep=0.75pt]   [align=left] {{\Large D}};

\draw (358,256) node [anchor=north west][inner sep=0.75pt]   [align=left] {90 \degree};

\end{tikzpicture}
}
\caption{Isosceles Triangle with $\vec{AD}$  $\perp$ $\vec{BC}$}
\label{eq:solutions/1/15/fig:tri_right_angle}
\end{figure}
Given,
line AD is perpendicular to line BC which implies the inner product is zero  	
\begin{align}\label{eq:solutions/1/15/my20}	
(\vec{B}-\vec{D})^T (\vec{A}-\vec{D}) &= (\vec{D}-\vec{A
})^T (\vec{B}-\vec{D})&= 0
\end{align}
\begin{align}\label{eq:solutions/1/15/my21}
(\vec{C}-\vec{D})^T (\vec{A}-\vec{D}) &= (\vec{D}-\vec{A})^T (\vec{C}-\vec{D})&= 0	
\end{align}
Consider$\triangle{BAD}$ and $\triangle{CAD}$ ;\\	
Taking inner product of sides BA and AD	
\begin{align}\label{eq:solutions/1/15/my22}	
(\vec{B}-\vec{A})^T(\vec{A}-\vec{D}) =	
\norm{\vec{B}-\vec{A}}\norm{\vec{A}-\vec{D}}\cos{BAD}	
\end{align}
The angle BAD from the  above equation is:	
\begin{align}\label{eq:solutions/1/15/my23}	
     \cos{BAD} = \frac{(\vec{B}-\vec{A})^T(\vec{A}-\vec{D})}{\norm{\vec{B}-\vec{A}}\norm{\vec{A}-\vec{D}}} 
\end{align}
Taking inner product of sides CA and AD	
\begin{align}\label{eq:solutions/1/15/my24}	
(\vec{C}-\vec{A})^T(\vec{A}-\vec{D}) =	
\norm{\vec{C}-\vec{A}}\norm{\vec{A}-\vec{D}}\cos{CAD}	
\end{align}
The angle CAD from the  above equation is:	
\begin{align}\label{eq:solutions/1/15/my25}	
     \cos{CAD} = \frac{(\vec{C}-\vec{A})^T(\vec{A}-\vec{D})}{\norm{\vec{C}-\vec{A}}\norm{\vec{A}-\vec{D}}} 
\end{align}
from equation (\ref{eq:solutions/1/15/my23}) and (\ref{eq:solutions/1/15/my25})\\	
\begin{align}
\angle {BAD}=\angle {CAD}	
\end{align}
Now using pythagorus law;
\begin{align}\label{eq:solutions/1/15/my26}
     \norm{\vec{B-A}}^2 = \norm{\vec{A-D}}^2+ \norm{\vec{B-D}}^2
\end{align}
\begin{align}\label{eq:solutions/1/15/my27}
     \norm{\vec{C-A}}^2 = \norm{\vec{A-D}}^2+ \norm{\vec{C-D}}^2
\end{align}
using (\ref{eq:solutions/1/15/my20}) and (\ref{eq:solutions/1/15/my21}) in above equation we can conclude ;
\begin{align}\label{eq:solutions/1/15/my28}
     \norm{\vec{B-A}} = \norm{\vec{C-A}} 
\end{align}
Thus, $\triangle{ABC}$ is isosceles triangle. \\
Hence proved.
