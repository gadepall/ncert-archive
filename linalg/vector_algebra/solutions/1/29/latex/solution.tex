\begin{figure}[!ht]
\begin{center}
\resizebox{\columnwidth}{!}{\begin{tikzpicture}
    \draw (0,0) node[anchor=north]{$B$}-- (4,6) node[anchor=south]{$A$}-- (8,0) node[anchor=north]{$C$}-- cycle;
    \draw (4,6) -- (4,0) node[anchor=north]{$D$};
    \draw (4,0.5) -- (3.5,0.5) -- (3.5,0);
    \begin{scope}[shift={(4,0)}]
        \draw (135:0.75cm) node[anchor=east]{$90^\circ$};
    \end{scope}
\end{tikzpicture}
}
\end{center}
\caption{}
\label{fig:es17btech11009_fig1}
\end{figure}
Consider the above $\Delta$ABC. Given that AD is the perpendicular bisector of BC. So, BD = DC and $\angle$ADB = $\angle$ADC = 90.Since D is the midpoint of BC
\begin{multline}
    \vec{D} = \brak{\vec{B+C}}/2\\
    2\vec{D} = \brak{\vec{B+C}}\\
    \brak{\vec{B-D}} = \brak{\vec{D-C}}\\
    \norm{\vec{B}-\vec{D}}=\norm{\vec{D}-\vec{C}}\label{eq:solutions/1/29/es17btech11009_1}
\end{multline}
Since, AD is the perpendicular bisector of BC
\begin{multline}
    \brak{\vec{A-D}}^T\brak{\vec{B-C}} =\brak{\vec{B-C}}^T\brak{\vec{A-D}} = 0\\
    \brak{\vec{A-D}}^T\brak{\vec{B-D+D-C}} = 0\\
    \brak{\vec{A-D}}^T\brak{\vec{B-D}} + \brak{\vec{A-D}}^T\brak{\vec{D-C}} = 0\\
    \implies\brak{\vec{A-D}}^T\brak{\vec{B-D}}=\brak{\vec{A-D}}^T\brak{\vec{D-C}} = 0\label{eq:solutions/1/29/es17btech11009_2}
\end{multline}
\begin{multline}
    \brak{\vec{B-C}}^T\brak{\vec{A-D}} = 0\\
    \brak{\vec{B-D+D-C}}^T\brak{\vec{A-D}} = 0\\
    [\brak{\vec{B-D}}^T + \brak{\vec{D-C}}^T]\brak{\vec{A-D}} = 0\\
    \brak{\vec{B-D}}^T\brak{\vec{A-D}} + \brak{\vec{D-C}}^T\brak{\vec{A-D}} = 0\\
    \implies\brak{\vec{B-D}}^T\brak{\vec{A-D}}=\brak{\vec{D-C}}^T\brak{\vec{A-D}}= 0\label{eq:solutions/1/29/es17btech11009_3}
\end{multline}
Find the length of AB,
\begin{multline}
 \norm{\vec{A-B}}^2=\brak{\vec{A-B}}^{T}\brak{\vec{A-B}}\\ =\brak{\vec{A-D+D-B}}^{T}\brak{\vec{A-D+D-B}}\\
 = [\brak{\vec{A-D}}^{T}+\brak{\vec{D-B}}^{T}][\brak{\vec{A-D}}+\brak{\vec{D-B}}]\\
 =\brak{\vec{A-D}}^{T}\brak{\vec{A-D}} + \brak{\vec{A-D}}^{T}\brak{\vec{D-B}} +\\ \brak{\vec{D-B}}^{T}\brak{\vec{A-D}} + \brak{\vec{D-B}}^{T}\brak{\vec{D-B}}
\end{multline}
From Eq \eqref{eq:solutions/1/29/es17btech11009_2} and Eq \eqref{eq:solutions/1/29/es17btech11009_3},
\begin{multline}
    \brak{\vec{A-B}}^{T}\brak{\vec{A-B}}\\
    = \brak{\vec{A-D}}^{T}\brak{\vec{A-D}} + \brak{\vec{D-B}}^{T}\brak{\vec{D-B}}\\
    \implies\norm{\vec{A-B}}^2=\norm{\vec{A-D}}^2 + \norm{\vec{D-B}}^2 \label{eq:solutions/1/29/es17btech11009_4}
\end{multline}
Similarily, find the length of AC
\begin{multline}
 \norm{\vec{A-C}}^2=\brak{\vec{A-C}}^{T}\brak{\vec{A-C}}\\ =\brak{\vec{A-D+D-C}}^{T}\brak{\vec{A-D+D-C}}\\
 = [\brak{\vec{A-D}}^{T}+\brak{\vec{D-C}}^{T}][\brak{\vec{A-D}}+\brak{\vec{D-C}}]\\
 =\brak{\vec{A-D}}^{T}\brak{\vec{A-D}} + \brak{\vec{A-D}}^{T}\brak{\vec{D-C}} +\\ \brak{\vec{D-C}}^{T}\brak{\vec{A-D}} + \brak{\vec{D-C}}^{T}\brak{\vec{D-C}}
\end{multline}
From Eq \eqref{eq:solutions/1/29/es17btech11009_2} and Eq \eqref{eq:solutions/1/29/es17btech11009_3}
\begin{multline}
    \brak{\vec{A-C}}^{T}\brak{\vec{A-C}}\\
     = \brak{\vec{A-D}}^{T}\brak{\vec{A-D}} + \brak{\vec{D-C}}^{T}\brak{\vec{D-C}}\\
    \implies\norm{\vec{A-C}}^2=\norm{\vec{C-D}}^2 + \norm{\vec{D-A}}^2\label{eq:solutions/1/29/es17btech11009_5}
\end{multline}
Eq \eqref{eq:solutions/1/29/es17btech11009_1}, Eq \eqref{eq:solutions/1/29/es17btech11009_4} and Eq \eqref{eq:solutions/1/29/es17btech11009_5}
\begin{align}
      \norm{\vec{A-B}}^2=\norm{\vec{A-C}}^2 
\end{align}
Since the lengths AB and AC are equal, We can conclude that the $\Delta$ABC is an isosceles triangle with AB = AC.

