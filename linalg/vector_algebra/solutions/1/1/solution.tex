See Fig. \ref{fig:solutions/1/1tri_right_angle}

\begin{figure}[!ht]
\centering
\resizebox{\columnwidth}{!}{\begin{tikzpicture}
    \draw (0, 0) coordinate (B) 
        -- node[below] {$k$} (8,0) coordinate (C) 
        -- node[above right] {$k$} (4,6.9282) coordinate (A) 
        -- node[above left] {$k$}  (0, 0);
        \node at (A)[anchor=south] {$A$};
        \node at (B)[anchor=north] {$B$};
        \node at (C)[anchor=north] {$C$};
\end{tikzpicture}
}
\caption{Equilateral $\triangle{ABC}$ with A,B and C as vertices}
\label{fig:solutions/1/1tri_right_angle}
\end{figure}
Considering A,B and C as the vertices of triangle:
\begin{align}
   A=
\myvec{
a_1\\
a_2
}
B=
\myvec{
0\\
0
}
C=
\myvec{
c_1\\
c_2
} \nonumber
\end{align}
In equilateral triangle all sides are equal.Hence,
\begin{align}\label{eq:solutions/1/1/11}
\norm{\vec{A-B}} = \norm{\vec{B-C}} = \norm{\vec{A-C}}
\end{align}
Putting $\vec{B}$= 0 in (\ref {eq:solutions/1/1/11}) we have,
\begin{align}
\label{eq:solutions/1/1/12}
\norm{\vec{A}}=\norm{\vec{C}}
\end{align}
\begin{align}
\label{eq:solutions/1/1/13}
\norm{\vec{A}}=\norm{\vec{A-C}}
\end{align}
Squaring  equation (\ref{eq:solutions/1/1/12})
 \begin{align}
\label{eq:solutions/1/1/17}
\norm{\vec{A}}^2=\norm{\vec{C}}^2
\end{align}
Squaring  equation (\ref{eq:solutions/1/1/13})
 \begin{align}
\label{eq:solutions/1/1/14}
\norm{\vec{A}}^2=\norm{\vec{A}}^2 - 2(\vec{A}^T )(\vec{C}) +\norm{\vec{C}}^2 \nonumber \\
\implies \norm{\vec{A}}^2=2(\vec{A}^T )(\vec{C})
\end{align}
Taking the inner product of sides $AB$,$BC$ we have:
\begin{align}
    (\vec{A}-\vec{B})^T(\vec{B}-\vec{C}) =
    \norm{\vec{A}-\vec{B}}\norm{\vec{B}-\vec{C}}\cos{ABC}
\end{align}
The angle ABC from the  above equation is:
\begin{align}\label{eq:solutions/1/1/my}
     \cos{ABC} = \frac{(\vec{A}-\vec{B})^T(\vec{B}-\vec{C})}{\norm{\vec{A}-\vec{B}}\norm{\vec{B}-\vec{C}}} 
\end{align}
Substituting value in (\ref{eq:solutions/1/1/my}) and putting we have:
\begin{align}\label{eq:solutions/1/1/my1}
     \cos{ABC} = \frac{(\vec{A})^T(\vec{C})}{\norm{\vec{A}}^2}
\end{align}
From (\ref{eq:solutions/1/1/14}) we have:
\begin{align}\label{eq:solutions/1/1/my22}
     \cos{ABC} = \frac{(\vec{A})^T(\vec{C})}{2(\vec{A})^T (\vec{C})}\nonumber\\
     \implies \cos{ABC} = 1/2\nonumber\\
\implies \angle {ABC =60 \degree}
\end{align}
Taking the inner product of sides $AB$,$AC$ we have:
\begin{align}
    (\vec{B}-\vec{A})^T(\vec{A}-\vec{C}) =
    \norm{\vec{B}-\vec{A}}\norm{\vec{A}-\vec{C}}\cos{BAC}
\end{align}
The angle BAC from the  above equation is:
\begin{align}\label{eq:solutions/1/1/my70}
     \cos{BAC} = \frac{(\vec{B}-\vec{A})^T(\vec{A}-\vec{C})}{\norm{\vec{B}-\vec{A}}\norm{\vec{A}-\vec{C}}} 
\end{align}
Substituting value in (\ref{eq:solutions/1/1/my70}) and putting we have:
\begin{align}\label{eq:solutions/1/1/my71}
     \cos{BAC} = \frac{(\vec{A})^T(\vec{A-C})}{\norm{\vec{A}}^2}\\
     \implies \frac{(\vec{A})^T(\vec{A})-(\vec{A})^T(\vec{C})}{\norm{\vec{A}}^2}
\end{align}
We know $(\vec{A})^T (\vec{A})= {\norm{\vec{A}}^2}$\\
From  equation (\ref{eq:solutions/1/1/14}) we have:
$ (\vec{A})^T(\vec{C})= \frac{1}{2}{\norm{\vec{A}}^2}$\\
Substituting values in (\ref{eq:solutions/1/1/my71}) we have:
\begin{align}\label{eq:solutions/1/1/my72}
\cos{BAC} =
    \frac{{\frac{1}{2}\norm{\vec{A}}^2}}{\norm{\vec{A}}^2}\nonumber\\
     \implies \cos{BAC} = 1/2\nonumber\\
\implies \angle {BAC =60 \degree}
\end{align}
Taking the inner product of sides $AC$,$BC$ we have:
\begin{align}
    (\vec{C}-\vec{A})^T(\vec{C}-\vec{B}) =
    \norm{\vec{C}-\vec{A}}\norm{\vec{C}-\vec{B}}\cos{ACB}
\end{align}
The angle ACB from the  above equation is:
\begin{align}
\label{eq:solutions/1/1/my75}
     \cos{ACB} = \frac{(\vec{C}-\vec{A})^T(\vec{C}-\vec{B})}{\norm{\vec{C}-\vec{A}}\norm{\vec{C}-\vec{B}}} 
\end{align}
Substituting value in (\ref{eq:solutions/1/1/my75}) and putting we have:
\begin{align}\label{eq:solutions/1/1/my76}
     \cos{ACB} = \frac{(\vec{C-A})^T(\vec{C})}{\norm{\vec{A}}^2}\nonumber\\
     \implies \frac{(\vec{C})^T(\vec{C})-(\vec{A})^T(\vec{C})}{\norm{\vec{A}}^2}
\end{align}
We know $(\vec{C})^T (\vec{C})= {\norm{\vec{C}}^2}$\\
From   (\ref{eq:solutions/1/1/17})  and (\ref{eq:solutions/1/1/14})  we have:\\
$(\vec{A})^T(\vec{C})= \frac{1}{2}{\norm{\vec{A}}^2}=\frac{1}{2}{\norm{\vec{C}}^2}$\\
Substituting values in (\ref{eq:solutions/1/1/my76}) we have:
\begin{align}
%\label{eq:solutions/1/1/my75}
\cos {ACB}=
     \frac{{\frac{1}{2}\norm{\vec{C}}^2}}{\norm{\vec{C}}^2}\nonumber\\
     \implies \cos{ACB} = 1/2\nonumber\\
\implies \angle {ACB = 60} \degree
\end{align}
Hence from  equation (\ref{eq:solutions/1/1/my22}),(\ref{eq:solutions/1/1/my72}) and(\ref{eq:solutions/1/1/my75})
\begin{align}
\angle {ABC} =\angle {BAC}=\angle {ACB}=60\degree
    \end{align}

	
	
	
