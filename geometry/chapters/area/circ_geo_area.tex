%
%
\renewcommand{\theequation}{\theenumi}
\begin{enumerate}[label=\arabic*.,ref=\thesubsection.\theenumi]
\numberwithin{equation}{enumi}
%
%
\item
	The diameter of a circle is the chord that divides the circle into two equal parts. The diameter is equal to twice the radius
%
%	%
\item The ratio of the perimeter of a circle to its diameter is $\pi$.
\label{prob:circ_peri_dia}
\item {\em Radian} is a another unit of the angle defined by
%
\begin{align}
\pi \text{ radians} = 180 \degree
\end{align}
%
\item
	In Fig. \ref{ch5_polygon_def}, 6 congruent triangles are arranged in a circular fashion.  Such a figure is known as a regular hexagon.  In general, $n$ number of traingles can be arranged to form a regular polygon.
\begin{figure}[!ht]
	\begin{center}
		
		%\includegraphics[width=\columnwidth]{./figs/ch5_polygon_def}
		%\vspace*{-10cm}
		\resizebox{\columnwidth}{!}{\input{./figs/circle/circ_polygon.tex}}
	\end{center}
	\caption{Polygon Definition}
	\label{ch5_polygon_def}	
\end{figure}
%
\item
The angle formed by each of the congruent triangles at the centre of a regular polygon of $n$ sides is $\frac{2\pi}{n} = \frac{2\pi}{n}$ rad.
%
\item 	The triangle that forms a polygon of $n$ sides is given in Fig. \ref{ch5_polygon_area}. Show that 
%
\begin{equation}
BC = 2r \sin\frac{\pi}{n}
\label{eq:circ_chord_len}
\end{equation}
%

\begin{figure}[!ht]
	\begin{center}
		
		%\includegraphics[width=\columnwidth]{./figs/ch5_polygon_area}
		%\vspace*{-10cm}
		\resizebox{\columnwidth}{!}{\begin{tikzpicture}
[scale =2,>=stealth,point/.style = {draw, circle, fill = black, inner sep = 1pt},]

\node (A) at (0,3)[point,label=above :$A$] {};
\node (B) at (-3,0)[point,label=below :$B$] {};
\node (C) at (3,0)[point,label=below :$C$] {};
\draw (A)--(B);
\draw (C)--(B);
\draw (A)--(C);

\tkzMarkAngle[size=.2](B,A,C);
\node [above] at (0.05,2.5){$\frac{2\pi}{n}$};
\node [above] at (-1.6,1.5){$r$};
\node [above] at (1.6,1.5){$r$};
\end{tikzpicture}
}
	\end{center}
	\caption{Triangle that forms a polygon}
	\label{ch5_polygon_area}	
\end{figure}
%
\solution From  \eqref{eq:tri_crad_R}.
%
\begin{equation}
BC = 2r \sin \frac{A}{2} = 2r\sin\frac{\pi}{n}
\end{equation}
%%
\item
Show that the perimeter of a regular polygon is given by 
%
\begin{equation}
\label{eq:peri_poly_n}
2rn \sin\frac{\pi}{n}
\end{equation}
%
\item
Show that the area of a regular polygon is given by 
%
\begin{equation}
\frac{n}{2}r^{2}\sin\frac{2\pi}{n}
\end{equation}
%
\solution  From Fig. 	\ref{ch5_polygon_area}	

%
\begin{equation}
\begin{split}
ar\brak{polygon} &= n \times ar\brak{\Delta ABC} \\
&= \frac{n}{2}r^{2}\sin\frac{2\pi}{n}
\end{split}
\end{equation}
%
using \eqref{eq:circ_area_chord}

\item
	Using Fig. \ref{fig:circ_poly_squeeze}, show that
%
\begin{equation}
\label{fig:circ_poly_squeeze_eq}
\frac{n}{2}r^{2}\sin\frac{2\pi}{n} < \text{ area of circle } < nr^{2}\tan\frac{\pi}{n}
\end{equation}
%
The portion of the circle visible in Fig. \ref{fig:circ_poly_squeeze} is defined to be a sector of the circle.

\begin{figure}[!ht]
	\begin{center}
		
		%\includegraphics[width=\columnwidth]{./figs/fig:circ_poly_squeeze}
		%\vspace*{-10cm}
		\resizebox{\columnwidth}{!}{\input{./figs/circle/circ_poly_squeeze.tex}}
	\end{center}
	\caption{Circle Area in between Area of Two Polygons}
	\label{fig:circ_poly_squeeze}	
\end{figure}
%

\solution Note that the circle is squeezed between the inner and outer regular polygons.  As we can see from Fig. \ref{fig:circ_poly_squeeze}, the area of the circle should be in between the areas of the inner and outer polygons.  Since
%
\begin{align}
ar \brak{\Delta OAB} &= \frac{1}{2}r^{2}\sin\frac{2\pi}{n} \\
ar \brak{\Delta OPQ} &= 2 \times \frac{1}{2} \times r \tan\frac{2\pi/n}{2} \times r \\
&= r^{2}\tan\frac{\pi}{n},
\end{align}
%
we obtain \eqref{fig:circ_poly_squeeze_eq}.
%
\item
Show that
	%
	\begin{equation}
	\label{fig:circ_poly_squeeze_simple}
\cos^2\frac{\pi}{n} < \frac{\text{ area of circle }}{nr^{2}\tan\frac{\pi}{n}} < 1	\end{equation}
	%

\solution From \eqref{fig:circ_poly_squeeze_eq} and \eqref{eq:sin2x},
	%
{\small
	\begin{align}
	\frac{n}{2}r^{2}\sin\frac{2\pi}{n} < \text{ area of circle } 
%	\\
	< nr^{2}\tan\frac{\pi}{n} 
	\\
\Rightarrow 	
	{n}r^{2}\sin\frac{\pi}{n}\cos\frac{\pi}{n} < \text{ area of circle } 
%	\\
	< nr^{2}\tan\frac{\pi}{n} 
	\end{align}
%
}
which yields 	\eqref{fig:circ_poly_squeeze_simple} upon making use of the fact that 
%
\begin{align}
\frac{\sin \theta}{\cos \theta} = \tan \theta
\end{align}
%

%
\item
	Show that 
	\begin{equation}
	\label{ch5_cos_zero}
	\cos 0^{\degree} = 1
	\end{equation}

\solution Follows from the fact that $\cos 0 \degree = \sin \brak{90\degree -0\degree} = \sin \brak{90\degree }=1$ using \eqref{eq:sin90} and \eqref{eq:tri_90-ang}.



\item
	Show that 
	\begin{equation}
	\label{ch5_sin_zero}
	\sin 0^{\degree} = 0
	\end{equation}
%\solution Follows from the fact that $\sin 0 = 0$ and \eqref{eq:tri_sin_cos_id}.

\item
	Show that for large values of $n$
	%
	\begin{equation}
\label{eq:cos_zero_lim}
	%
\cos^2\frac{\pi}{n} = 1
%
	\end{equation}	
	% 

%
\solution  As $n \to \infty, \frac{\pi}{n} \to 0$. From \eqref{ch5_cos_zero}, this yields \eqref{eq:cos_zero_lim}.

%
\item  \eqref{eq:cos_zero_lim} is a {\em limit} and 
	 expressed as 
%
\begin{equation}
\label{eq:cos_zero_lim_def}
\lim_{n \rightarrow \infty}\cos^2\frac{\pi}{n} = 1
\end{equation}
%	

\item
	Show that 
	%
\begin{equation}
\label{fig:circ_poly_squeeze_tan_n}
\text{ area of circle } = r^2\lim_{n \rightarrow \infty}
{n\tan\frac{\pi}{n}} 
	%
\end{equation}	
	% 
\solution From \eqref{fig:circ_poly_squeeze_simple} and \eqref{eq:cos_zero_lim_def}, 
	%
	\begin{align}
%	\label{fig:circ_poly_squeeze_simple}
\lim_{n\to \infty}\cos^2\frac{\pi}{n} < \lim_{n\to \infty} \frac{\text{ area of circle }}{nr^{2}\tan\frac{\pi}{n}} < 1	
\\
1 = \lim_{n\to \infty} \frac{\text{ area of circle }}{nr^{2}\tan\frac{\pi}{n}} < 1	
\end{align}
	%
resulting in \eqref{fig:circ_poly_squeeze_tan_n}.
%
\item Show that 

	\begin{align}
\label{eq:peri_poly_sin_n}
 \lim_{n\to \infty}n \sin\frac{\pi}{n} &= \pi 
\end{align}
%
\solution From \eqref{prob:circ_peri_dia} and \eqref{eq:peri_poly_n}, the perimeter of the circle is 
%
\begin{align}
%\label{eq:peri_poly_sin_n}
\lim_{n\to \infty}2rn \sin\frac{\pi}{n} &= 2\pi r
\implies \lim_{n\to \infty}n \sin\frac{\pi}{n} &= \pi 
\end{align} 
\item Show that 

	\begin{equation}
\label{eq:peri_poly_tan_n}
	\pi = \lim_{n \rightarrow \infty}
	{n\tan\frac{\pi}{n}}
	\end{equation}
\solution 
%
From Fig. \eqref{fig:circ_poly_squeeze}, using the fact that the inner and outer polygons converge into a circle for large $n$,
\begin{align}
%\label{eq:peri_poly_n}
\lim_{n\to \infty} nCD -nAB &= 0
\\
\implies \lim_{n\to \infty} 2r n\tan\frac{\pi}{n}-2r n\sin\frac{\pi}{n} &= 0
\end{align}
%
from which, we obtain \eqref{eq:peri_poly_tan_n}
by substituting from \eqref{eq:peri_poly_sin_n}.


\item Show that the area of a circle is $\pi r^2$.
\\
\solution Use \eqref{eq:peri_poly_tan_n} in \eqref{fig:circ_poly_squeeze_tan_n}.

%\item
%	The radian is a unit of angle defined by
%\begin{equation}
%	1 \text{ radian} = \frac{2\pi}{2\pi}
%\end{equation}

%
%\item
%	Show that the circumference of a circle is $2 \pi r$.
\item
	Show that
	\begin{equation}
	\lim_{\theta \rightarrow 0} \frac{\sin\theta}{\theta} =
	\lim_{\theta \rightarrow 0} \frac{\tan\theta}{\theta} = 1
	\end{equation}

\item
	Show that the area of a sector with angle $\theta$ in radians is $\frac{1}{2}r^2\theta$.


\end{enumerate}
