\begin{figure}[!ht]
\centering
\resizebox{\columnwidth}{!}{\begin{tikzpicture}[scale = 1.5,>=stealth,point/.style={draw,circle,fill=black, inner sep=0.5pt},]

\node (A) at (4, 1)[point,label=above right:$A$] {};
\node (B) at (6, 5)[point,label=above right:$B$] {};
\node (O) at (3, 4)[point,label=above:$O$] {};

\draw (3,4) circle(3.162277cm);

\draw[dotted] (O) -- node[below=5pt]{}(A);
\draw[dotted] (O) -- node[below=5pt]{}(B);

\end{tikzpicture}}
\caption{}
\label{fig:8.5.40_circle_1}	
\end{figure}

\item  {\em Construction: } See Fig. \ref{fig:8.5.40_circle_1}.	 The input parameters are
\begin{align}
\vec{O} &= \myvec{0\\0},
\vec{A} &= r_1\myvec{\cos \theta_1\\ \sin \theta_1}
\vec{D} &= r_2\myvec{\cos \theta_2\\ \sin \theta_2}
\end{align}
The two concentric circles  with centre $\vec{O}$ and radii $r_1$ and $r_2$ are drawn.

\subitem The equation of $AD$ is 
\begin{align}
\label{eq:8.1.40_line}
\vec{x} = \vec{A} + \lambda \brak{\vec{A}-\vec{D}}
\end{align}
Points $\vec{B}$ and $\vec{D}$ are obtained as the intersection of $AB$ and the circle with equation
\begin{align}
\label{eq:8.1.40_circle}
\norm{\vec{x}}^2 = r_2^2
\end{align}
From \eqref{eq:8.1.40_circle}
and \eqref{eq:8.1.40_circle}, 
\begin{align}
\norm{\vec{A} + \lambda \brak{\vec{A}-\vec{D}}}^2 = r_2^2
\end{align}
\begin{multline}
\lambda^2\norm{\vec{A}-\vec{D}}^2 + 2\lambda \vec{A}^T\brak{\vec{A}-\vec{D}} 
\\
+ \norm{\vec{A}}^2 -r_2^2 =0
\label{eq:8.1.40_quad}
\end{multline}
Solving the above quadratic equation yields two values for $\lambda$ yielding $\vec{B}$ and $\vec{C}$.  
\begin{align}
\vec{M} = \frac{\vec{B}+\vec{C}}{2}
\end{align}

