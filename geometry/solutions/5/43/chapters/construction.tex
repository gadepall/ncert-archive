
\begin{figure}[!ht]
\centering
\resizebox{\columnwidth}{!}{\input{./solutions/5/43/figs/trapezium_altitude_fig}}
\caption{}
\label{fig:8.5.43_trapezium}	
\end{figure}
%
\item {\em Construction: }See Fig. \ref{fig:8.5.43_trapezium}
The input parameters are
\begin{align}
\vec{A} &=\myvec{0\\0},
\vec{B} &= \myvec{b\\0}, \label{eq:8.5.43_constr_b}
\\
\vec{C} &= \myvec{b - h\cot{\theta}\\h}\label{eq:8.5.43_constr_c}\\ 
\vec{D} &= h\myvec{\cot{\theta}\\ 1}\label{eq:8.5.43_constr_d}	\end{align}
%
which are sufficient to draw the trapezium.  The circumcircle of $\triangle ABC$ is then drawn.  This circle passes through $\vec{D}$.
