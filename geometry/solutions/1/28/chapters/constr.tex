%\renewcommand{\theequation}{\theenumi}
%\begin{enumerate}[label=\thesection.\arabic*.,ref=\thesection.\theenumi]
%\numberwithin{equation}{enumi}
%\item 
\item {\em Construction:} 
%with angles $\phase{ A},\phase{ C}$ and $\phase{ B}$ and sides $a, b$ and $c$.  The unique feature of this triangle is $\phase{ C}$ which is defined to be $90\degree$.%
%\renewcommand{\thefigure}{\theenumi.\arabic{figure}}
From the given information, 
%$\triangle ABC$ are 
\begin{align}
\label{eq:8.1.28_constr_a}
\vec{A} &= \myvec{0\\b} 
\\
 \vec{C} &= \myvec{0\\0}, 
\label{eq:8.1.28_constr_c}
\\
\vec{B} &= \myvec{a\\0}
\label{eq:8.1.28_constr_b}
\end{align}
$\because \vec{M}$ is the midpoint of $AB$,
\begin{align}
\vec{M}= \frac{\vec{A}+\vec{B}}{2} = \frac{1}{2}\myvec{a\\b}
\label{eq:8.1.28_constr_m}
\end{align}
%
Also, $\vec{M}$ is given to be the midpoint of $C$.  Hence, 
\begin{align}
\vec{M}&= \frac{\vec{C}+\vec{D}}{2}
\\
\implies \vec{D} &= 2 \vec{M} - \vec{C} = \myvec{a\\b}
\label{eq:8.1.28_constr_d}
\end{align}
\begin{figure}[!ht]
\centering
\resizebox{\columnwidth}{!}{
\begin{tikzpicture}

\draw (0,0) node[anchor=north]{$B$}
  -- (4,8) node[anchor=west]{$A$}
  -- (8,0) node[anchor=north]{$C$}
  -- cycle;
\draw (4,0) node[anchor=north]{$E$}
  -- (2,4) node[anchor=east]{$D$}
  -- (6,4) node[anchor=west]{$F$}
  -- cycle;
\end{tikzpicture}

}
\caption{}
\label{fig:8.1.28}	
\end{figure}
%
%
%\renewcommand{\thefigure}{\theenumi}
%
%\item List the design parameters for construction
%\label{const:table1}
%\\
%\solution See Table. \ref{table:table1}. 
%%
%\begin{table}[ht!]
%\centering
%%\begin{tabular}{ |p{3cm}|p{3cm}|  }
%%\hline
%% \multicolumn{2}{|c|}{Initial Input Values.} \\
%%\hline
%%a & 4\\
%%\hline
%%b & 3\\
%%\hline
%%$\phase{(ACB)$ & $90^{\circ}$ \\
%%\hline
%%\end{tabular}
%\input{./tables/inp.tex}
%\caption{To construct $\triangle ACB$}
%\label{table:table1}	
%\end{table}
%\item
%	For simplicity, let the greek letter $\alpha = \phase{ B$.  We have the following definitions.
%\begin{equation}
%\label{eq:8.1.28_tri_trig_defs}
%\begin{matrix}
	%\sin \theta = \frac{b}{c} & 	\cos \theta = \frac{a}{c} \\
	%\tan \theta = \frac{c}{a} & \cot \theta = \frac{1}{\tan \theta} \\
	%\csc \theta = \frac{1}{\sin \theta} & \sec \theta = \frac{1}{\cos \theta}
	%\end{equation}
%
%\item Find the coordinates of the various points in Fig. \ref{fig:tri_right_angle}
%\label{const:tri_right_angle}
%\\
%
%\solution 
%
%The values are listed in 
%%\item List the  derived values.
%%\label{const:table2}
%%\\
%%\solution See  
%Table. \ref{table:table2} 
%\begin{table}[ht!]
%%\centering
%%\begin{tabular}{ |p{3cm}|p{3cm}|  }
%%\hline
%% \multicolumn{2}{|c|}{Derived Values.} \\
%%\hline
%%$\vec{M}$ & $$\begin{pmatrix}2\\1.5\end{pmatrix}$$\\						
%%\hline
%%$\vec{D}$ & $$\begin{pmatrix}4\\3\end{pmatrix} $$\\
%%\hline
%\input{./tables/inp1.tex}
%\caption{To construct $\triangle DBC$}
%\label{table:table2}
%
%\end{table}
%
%\item Draw Fig. \ref{fig:tri_right_angle}.	
%\\
%\solution The  following Python code generates Fig. \ref{fig:tri_sss_py}
%%
%\begin{lstlisting}
%codes/triangle.py
%\end{lstlisting}
%\begin{figure}[!ht]
%\centering
%\includegraphics[width=\columnwidth]{./figs/triangle.eps}
%\caption{Triangle generated using python}
%\label{fig:tri_sss_py}
%\end{figure}
%
%%
%and the equivalent latex-tikz code generating Fig. \ref{fig:tri_right_angle} is 
%\begin{lstlisting}
%figs/triangle.tex
%\end{lstlisting}
%
%The above latex code can be compiled as a standalone document as
%\begin{lstlisting}
%figs/triangle_fig.tex
%\end{lstlisting}

%

%

%
%

%\end{enumerate}
